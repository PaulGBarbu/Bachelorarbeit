\subsection{Lizenzen}

% TODO: Weitere Quelle einbauen!
\todoBox{
    \textbf{Weitere Quelle einbauen:}

    \cite{stewartImpactsLicenseChoice2006a} Seite 140 (16 in PDF) sagt aus, 
    dass nicht-restriktive Lizenzen sich positiv auf ein Projekt auswirken (Hypothese 1A), 
    indem mehr User angezogen werden.
    
    Gleichzeitig heißt es das es \textit{keinen} Zusammenhang gibt zwischen restriktive Lizenzen
    und anziehen von mehr Entwicklern für das Projekt.
}


% Lizenzen haben signifikaten Einfluss auf Erfolg
Laut Subramaniam et. al. haben Lizenzen einen \textbf{signifikanten} Einfluss auf den Erfolg von 
Open Source Software vor allem dann, wenn die Zielgruppe Entwickler sind.
Freie Lizenzen wie MIT oder BSD haben einen positiven Einfluss auf Software Entwickler die OSS nutzten,
während restriktive Lizenzen wie GPL sich negativ auf den Erfolg von OSS auswirken.
Die Erklärung von Subramaniam ist, dass Entwickler die OSS nutzen es tun, um diese zu
modifizieren, in eigene Projekte einzubauen und weiterzuverbreiten. Das ist mit
restriktiven Lizenzen meist nicht bedingungslos umsetzbar. Wenn die Software allerdings an Endnutzer
gerichtet ist, wie zum Beispiel die Chat-App Telegram, spielt die Lizenz eine weniger wichtige Rolle, 
da Weiterverbreitung und Modifizierung
für den normalen Nutzer keine Rolle spielen \cite{subramaniamDeterminantsOpenSource2009}.

% Widerspruch von Midha et. al.
Midha und Palvia widerspricht allerdings dieser Aussage, laut ihnen spielt die Lizenz nur zu Beginn
des Projektes eine Rolle, da sobald ein Projekt beliebt ist, die Beliebtheit höher gewichtet wird als
die Lizenz, so behaupten sie.
Es heißt allerdings auch, dass restriktive Lizenzen sich im späteren verlauf eines Projektes positiv
auf Entwickler auswirkt \cite{midhaFactorsAffectingSuccess2012}. % Kapitel 6.2
Die Stichprobengröße von Midha et. al. lag allerdings nur bei 283, % Kapitel 4.1 (midhaFactorsAffectingSuccess2012)
während die Stichprobengröße von Subramaniam et. al. bei 
8627 lag \cite{subramaniamDeterminantsOpenSource2009}. % Kapitel 4.2

% ! Hidden Quote
% \begin{quote}
%     \begin{tcolorbox}[colback=black!5!white,colframe=white!75!black,title=Direkt Zitat aus \cite{midhaFactorsAffectingSuccess2012} Kapitel 6.2]
%         The insignificance may be because consumers
%         do not wish to be bothered by the license choice when information on other extrinsic attributes is readily
%         available. This, in a way, agrees with vox populi that most of the end users do not read the
%         licensing agreements.
%     \end{tcolorbox}
% \end{quote}

Wie in \ref{ssec:Eine Community Aufbauen} später genauer erläutert wird, ist eine Community ein 
essenzieller Bestandteil für ein Open Source Projekt.
Abhängig der Lizenz zieht man unterschiedliche Personengruppen an.
Offene Lizenzen wie MIT lädt vor allem X an...
Mit eingeschränkten Lizenzen wie GPL zieht man weniger Leute/Unternehmen etc. an und hindert 
somit das Wachstum der eigenen Community \citationNeeded{\cite{stewartImpactsLicenseChoice2006a} PDF S. 16}

Unternehmen nutzten die Software, 
eingie improven die die Software und ein Teil davon gibt zur OSS Community auch wieder zurück 
\cite{bangerthWhatMakesComputational2013} % Kapitel 3.5

\bigskip

% --------------------------------- Hypothese -------------------------------- %
\begin{hypothesis}
    Lizenzen haben einen signifikanten Einfluss sowohl auf den technischen
    als auch den Markterfolg. 
    Wobei offene Lizenzen sich positiv auswirken, während restriktive einen negativen
    Einfluss auf den Erfolg haben
\end{hypothesis}
% ---------------------------------------------------------------------------- %