\subsection{Lizenzen}
\rawidea

\noindent
Laut Midha und Palvia spielt die Lizenz eine \textbf{insignifikante} Rolle beim Erfolg eines Projektes.
Lediglich zu Beginn eines Projektes spielt die Lizenz eine Rolle,
da für den Nutzer wenige alternative Metriken zur Entscheidung stehen. 
Die Begründung hierfür sei das, wenn ein Projekt schon beliebt ist die Beliebtheit höher
gewichtet wird als die Lizenz. Gleichzeitig heißt es aber auch, dass GPL und ähnlich restriktive
Lizenzen einen positiven Effekt auf die Entwickler haben. Vor allem im späteren Verlauf der 
Entwicklung \cite{midhaFactorsAffectingSuccess2012} % Kapitel 6.2 

% ! Hidden Quote
% \begin{quote}
%     \begin{tcolorbox}[colback=black!5!white,colframe=white!75!black,title=Direkt Zitat aus \cite{midhaFactorsAffectingSuccess2012} Kapitel 6.2]
%         The insignificance may be because consumers
%         do not wish to be bothered by the license choice when information on other extrinsic attributes is readily
%         available. This, in a way, agrees with vox populi that most of the end users do not read the
%         licensing agreements.
%     \end{tcolorbox}
% \end{quote}

\cite{subramaniamDeterminantsOpenSource2009} widerspricht zum Teil
\cite{midhaFactorsAffectingSuccess2012}, laut Subramaniam et. al.
haben Lizenzen einen signifikanten Einfluss auf den Erfolg von Open Source vor allem dann,
wenn die Zielgruppe Entwickler sind.
Restriktive Lizenzen haben einen negativen Effekt auf den Erfolg von OSS.
Die Erklärung von Subramaniam \todoo{ist/sei}, dass Entwickler OSS nutzen mit der Intention es zu
modifizieren, in eigene Projekte einzubauen und weiterzuverbreiten, dies ist mit
restriktiven Lizenzen meist nicht bedingungslos umsetzbar. Wenn die Software allerdings an Endnutzer
gerichtet ist, spielt die Lizenz eine insignifikante Rolle, da Weiterverbreitung und Modifizierung
für den normalen Nutzer keine Rolle spielen \cite{subramaniamDeterminantsOpenSource2009}.

\todo{Kommenden Absatz schöner Formulieren}

Wie in \ref{ssec:Eine Community Aufbauen} später genauer erläutert wird, ist eine Community wichtig, 
abhängig der Lizenz kann man sich hier entweder das Leben schwer machen oder nicht, 
eine offene Lizenz wie MIT lädt alle dazu ein beizutragen. Unternehmen nutzten die Software, 
eingie improven die die Software und ein Teil davon gibt zur OSS Community auch wieder zurück 
\cite{bangerthWhatMakesComputational2013} % Kapitel 3.5
Mit eingeschränkten Lizenzen wie GPL zieht man weniger Leute/Unternehmen etc. an und hindert 
somit das Wachstum der eigenen Community 

\todoo{Der Wechsel von Docker könnte ggf. Interssant sein sich anzuschauen.}
\todoo{Hier \cite{deloneDeLoneMcLeanModel2003} hab ich was gelesen und markiert, aber bisher als Quelle noch nicht eingebaut}

\begin{hypothesis}
    Lizenzen haben einen signifikanten Einfluss auf den technischen Erfolg. 
    Wobei offene Lizenzen wie MIT o.ä. einen positiven,
    während restriktive Lizenzen einen eher negativen Einfluss haben.
\end{hypothesis}

\noindent
Die Datenerhebung wird mittels Datenanalyse und Teil der Umfrage sein. 