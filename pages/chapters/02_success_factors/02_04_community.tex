\subsection{Eine Community Aufbauen} \label{ssec:Eine Community Aufbauen}

\rawidea

\todoo{There are two types of community, User Community and Developer Community...}

\noindent
Ein Open Source Projekt braucht eine Community.
Eine Community von Benutzern und eine Community von Contributor. Ohne eine Community kann ein Projekt
nicht wachsen. Contributer werden gebraucht um das Projekt kontinuierlich zu verbessern, Benutzer um
es natürlich zu nutzten (aka Goal of the " success ") aber auch um Bugs zu finden und zu reporten,
dies muss allerdings auch aktive encouraged werden. \cite{bangerthWhatMakesComputational2013} 
Sprich die Entwickler müssen sich um die
Community kümmern, bzw aktiv dafür sorgen, dass die Community wächst. 

\todoBox{
    Weitere Quellen: \cite{midhaFactorsAffectingSuccess2012}

    Mögliche weitere Quelle \link{How do Firms Make Use of Open Source Communities}{https://www.sciencedirect.com/science/article/abs/pii/S0024630108000836}
}


% \begin{hypothesis}
%     Bemühungen darin eine Community aufzubauen führen zu einem hohen Markt Erfolg.
% \end{hypothesis}