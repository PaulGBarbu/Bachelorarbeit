\subsection{Eine Community Aufbauen} \label{ssec:Eine Community Aufbauen}
\todoo{Braucht es dieses Kapitel zwingend? Sieh Kapitel 2.3. in \cite{bangerthWhatMakesComputational2013}}

Ein Open Source Projekt braucht eine Community.
Eine Community von Benutzern und eine Community von Contributor. Ohne eine Community kann ein Projekt
nicht wachsen. Contributer werden gebraucht um das Projekt kontinuierlich zu verbessern, Benutzer um
es natürlich zu nutzten (aka Goal of the "success") aber auch um Bugs zu finden und zu reporte,
dies muss allerdings auch aktive encouraged werden. \cite{bangerthWhatMakesComputational2013}

\begin{hypothesis}
    Bemühungen darin eine Community aufzubauen führen zu einem hohen Markt Erfolg.
\end{hypothesis}