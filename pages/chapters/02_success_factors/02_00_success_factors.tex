\chapter{Erfolgsfaktoren}

% ---------------------
\textcolor{red}{\rule{\textwidth}{3pt}}


\bigskip
\noindent
\textbf{\todoo{Wie soll man jetzt in diesem Kapitel die Faktoren unterteilen?}}

\noindent
Aus den ganzen Erfolgsfaktoren sollen folgende Punkte spezifisch bearbeitet werden. Weitere Punkte können zusätzlich auf Basis von
Literatur angesprochen werden. Sieh Liste weiter unten

\begin{enumerate}
    \item Lizenzen haben einen Einfluss auf den Erfolg der Projekte
    \item Hohe Qualität der Software  => positiver Einfluss aufs Projekt
    \item Gute Dokumentation => positiver Einfluss aufs Projekt
    \item Wer sich um die Community kümmert (devs / userbase) => positiver Einfluss aufs Projekt
\end{enumerate}

\noindent
Diese Punkte können angesprochen werden, sind aber etwas schwer zu erfassen mit einer Umfrage oder Crawler.

\begin{itemize}
    \item Network Effekt, an der Community kann man zwar arbeiten den Network Effekt aber nicht so richtig triggern (I guess)
    \item Responsibility Assignment (Bezug auf Community / Qualität)
    \item Timing
    \item Modularität
    \item Komplexität
\end{itemize}


\todoo{BTW}
Wenn es aussieht als ob die Arbeit zu kurz wird, kann man noch einzelne Projekte ganz genau unter die Lupe nehmen I guess
z.B. Linux, oder React for that matter weil JavaScript etc.

React gegenüberstellen zu Vue. Das eine ist ja aus dem Hause Facebook, das andere völlig Open Source...
Weitere mögliche Kapitel also \textit{Case Study: React vs. Vue} oder so ähnlich


\noindent
\textcolor{red}{\rule{\textwidth}{3pt}}

\todo{Infos ergänzen nachdem, die Unterkapitel geschrieben sind}

\noindent
Im Artikel von Wolfgang Bangerth und Timo Heister werden die Erfolgsfaktoren in zwei Klassen unterteilt.
Eine Klasse mit beeinflussbaren Faktoren und eine für nicht beeinflussbare Faktoren
\cite{bangerthWhatMakesComputational2013}.
Im Folgenden sollen diese Faktoren erklärt werden und wie sie in diese Arbeit eingebunden werden.



% --------------------------------------------------------------------------------------------------------

\section{Beeinflussbare Faktoren}

% TODO: Braucht es überhaupt die Unterteilung in Beinflussbar vs nicht beeinflussbar?
% I Guess höchstens wenn man so etwas sagen will wie "Wir fokussieren uns auf das was die Entwickler in der Hand haben..."  + mehr Text

Beeinflussbare Faktoren sind die, die OSS Entwickler selbst in der Hand haben. Zu diesen Faktoren gehört
die Lizenz, Qualität der Software sowie der Dokumentation und der \todoo{Aufbau} einer Community.

\subsection{Lizenzen}

% TODO: Weitere Quelle einbauen!
\todoBox{
    \textbf{Weitere Quelle einbauen:}

    \cite{stewartImpactsLicenseChoice2006a} Seite 140 (16 in PDF) sagt aus, 
    dass nicht-restriktive Lizenzen sich positiv auf ein Projekt auswirken (Hypothese 1A), 
    indem mehr User angezogen werden.
    
    Gleichzeitig heißt es das es \textit{keinen} Zusammenhang gibt zwischen restriktive Lizenzen
    und anziehen von mehr Entwicklern für das Projekt.
}


% Lizenzen haben signifikaten Einfluss auf Erfolg
Laut Subramaniam et. al. haben Lizenzen einen \textbf{signifikanten} Einfluss auf den Erfolg von 
Open Source Software vor allem dann, wenn die Zielgruppe Entwickler sind.
Freie Lizenzen wie MIT oder BSD haben einen positiven Einfluss auf Software Entwickler die OSS nutzten,
während restriktive Lizenzen wie GPL sich negativ auf den Erfolg von OSS auswirken.
Die Erklärung von Subramaniam ist, dass Entwickler die OSS nutzen es tun, um diese zu
modifizieren, in eigene Projekte einzubauen und weiterzuverbreiten. Das ist mit
restriktiven Lizenzen meist nicht bedingungslos umsetzbar. Wenn die Software allerdings an Endnutzer
gerichtet ist, wie zum Beispiel die Chat-App Telegram, spielt die Lizenz eine weniger wichtige Rolle, 
da Weiterverbreitung und Modifizierung
für den normalen Nutzer keine Rolle spielen \cite{subramaniamDeterminantsOpenSource2009}.

% Widerspruch von Midha et. al.
Midha und Palvia widerspricht allerdings dieser Aussage, laut ihnen spielt die Lizenz nur zu Beginn
des Projektes eine Rolle, da sobald ein Projekt beliebt ist, die Beliebtheit höher gewichtet wird als
die Lizenz, so behaupten sie.
Es heißt allerdings auch, dass restriktive Lizenzen sich im späteren verlauf eines Projektes positiv
auf Entwickler auswirkt \cite{midhaFactorsAffectingSuccess2012}. % Kapitel 6.2
Die Stichprobengröße von Midha et. al. lag allerdings nur bei 283, % Kapitel 4.1 (midhaFactorsAffectingSuccess2012)
während die Stichprobengröße von Subramaniam et. al. bei 
8627 lag \cite{subramaniamDeterminantsOpenSource2009}. % Kapitel 4.2

% ! Hidden Quote
% \begin{quote}
%     \begin{tcolorbox}[colback=black!5!white,colframe=white!75!black,title=Direkt Zitat aus \cite{midhaFactorsAffectingSuccess2012} Kapitel 6.2]
%         The insignificance may be because consumers
%         do not wish to be bothered by the license choice when information on other extrinsic attributes is readily
%         available. This, in a way, agrees with vox populi that most of the end users do not read the
%         licensing agreements.
%     \end{tcolorbox}
% \end{quote}

Wie in \ref{ssec:Eine Community Aufbauen} später genauer erläutert wird, ist eine Community ein 
essenzieller Bestandteil für ein Open Source Projekt.
Abhängig der Lizenz zieht man unterschiedliche Personengruppen an.
Offene Lizenzen wie MIT lädt vor allem X an...
Mit eingeschränkten Lizenzen wie GPL zieht man weniger Leute/Unternehmen etc. an und hindert 
somit das Wachstum der eigenen Community \citationNeeded{\cite{stewartImpactsLicenseChoice2006a} PDF S. 16}

Unternehmen nutzten die Software, 
eingie improven die die Software und ein Teil davon gibt zur OSS Community auch wieder zurück 
\cite{bangerthWhatMakesComputational2013} % Kapitel 3.5

\bigskip

% --------------------------------- Hypothese -------------------------------- %
\begin{hypothesis}
    Lizenzen haben einen signifikanten Einfluss sowohl auf den technischen
    als auch den Markterfolg. 
    Wobei offene Lizenzen sich positiv auswirken, während restriktive einen negativen
    Einfluss auf den Erfolg haben
\end{hypothesis}
% ---------------------------------------------------------------------------- %
\subsection{Qualität}

\rawidea
Ein weiterer Punkt worüber Entwickler Einfluss haben ist die Qualität der Software.
Produkts als auch die User Experience (UX) beziehungsweise Developer Experience (DX).
Sprich, gibt es eine gute Dokumentation, ist das Package etc. Customizable nach den Wünschen
des Nutzers etc. Zu der Qualität gehört auch Software die "Bug-Frei" läuft, oder zumindest die Funktionalität
die Bugs überwiegt \cite{bangerthWhatMakesComputational2013}. % Kapitel 2.1. Utility and Quality

\begin{quote}
    \begin{tcolorbox}[colback=black!5!white,colframe=white!75!black,title=Direkt Zitat aus \cite{bangerthWhatMakesComputational2013} Kapitel 2.1]
        [W]ithout a focus on fixing bugs as soon as they are
        identified will never be of high quality.
        Thus, quality needs to be an important aspect of
        development from the start.
    \end{tcolorbox}
\end{quote}

Die \textit{First Time Experience} spielt eine wichtige Rolle, ist ein Tool
schwer aufzusetzten / installieren (bei npm trifft das nicht ganz zu?)
beziehungsweise die Dokumentation nicht schlüssig genug, gibt es meist eine (Hand voll/Menge...) alternative Tools
die ein User stattdessen einfach hernehmen kann, statt sich mit Tool \textit{X} herumzuschlagen
\cite{bangerthWhatMakesComputational2013}. % Kapitel 2.1 Utility and Quality

\todoo{
    Als Beispiel, könnte man hierfür sowas nehmen wie eine JS Bibliothek die 10 verschiedene Alternativen hat.
    Beispielsweise hat das Tool X mit 300k Downloades 5 alternativen, die zwar nicht soviele Download zahlen haben,
    aber wenn man das Tool nicht zum laufen kriegt, dann schlägt man sich damit auch nicht weiter rum und geht einfach
    zu Y, Z oder A.
}

\begin{hypothesis}
    Die Qualität spielt eine sehr wichtige Rolle, vor allem dann, wenn es Alternativen gibt. 
    (Gute Qualität => Guter Technischer/Markt Erfolg) 
\end{hypothesis}
\subsection{Gute Dokumentation}

\rawidea
Dokumentationen spielen eine entscheidende Rolle beim Erfolg eines Projekts.
Ohne eine gute Dokumentation ist die Software schwerer zugänglich für die Benutzer und damit teils
unbrauchbar, ausgenommen Projekte mit intuitiven User Interfaces.
Mailing Listen und StackOverflow können eine gute Ergänzung zur Dokumentation sein, allerdings kann
diese dadurch nicht ersetzt werden.
Mit einem Crawler ist es schwer zu beurteilen, ob eine Dokumentation gut ist oder nicht oder ob
eine Dokumentation überhaupt existiert, da sich diese häufig auch auf der Homepage des Projekts befinden.
Man kann aber Dokumentation mit als Punkt in die Umfrage mit aufnehmen.
\citationNeeded{Könnte aus \cite{bangerthWhatMakesComputational2013} stammen}

\bigskip

\todoo{
    \noindent
    \textbf{"Wie wichtig ist eine gute Dokumentation bei der Auswahl einer OSS für Sie?"} eignet
    sich als hervorragende Frage in der Umfrage.\\
    Alternative könnte man diese Daten auch erfassen, allerdings nur von Hand.
    Da zum einen die Dokumentationen nicht immer in der README.md sind, sondern auf anderen Website
    und der Crawler nicht beurteilen kann, ob eine Dokumentation gut ist oder nicht. Daher könnte man
    quasi eine Liste zum Abhacken durchgehen
    Beispielsweise wie folgt:
    \begin{itemize}
        \item Hat das Projekt eine Dokumentation?
        \item Hat die Doku Anwendungsbeispiele?
        \item Ist ein Sandbox-Modus für dieses Projekt möglich? Wenn ja, gibt es einen?
    \end{itemize}
}

\begin{hypothesis}
    Dokumentation ist wichtig vorallem wenn die Zielgruppe Entwickler sind.
    (Gute Doku => Hoher Technischer Erfolg)
\end{hypothesis}
\subsection{Eine Community Aufbauen} \label{ssec:Eine Community Aufbauen}
\todoo{Braucht es dieses Kapitel zwingend? Sieh Kapitel 2.3. in \cite{bangerthWhatMakesComputational2013}}

Ein Open Source Projekt braucht eine Community.
Eine Community von Benutzern und eine Community von Contributor. Ohne eine Community kann ein Projekt
nicht wachsen. Contributer werden gebraucht um das Projekt kontinuierlich zu verbessern, Benutzer um
es natürlich zu nutzten (aka Goal of the "success") aber auch um Bugs zu finden und zu reporte,
dies muss allerdings auch aktive encouraged werden. \cite{bangerthWhatMakesComputational2013}

\begin{hypothesis}
    Bemühungen darin eine Community aufzubauen führen zu einem hohen Markt Erfolg.
\end{hypothesis}


\section{Nicht beeinflussbare Faktoren}

Nicht beeinflussbare Faktoren sind Faktoren über die OSS Entwickler wenig bis gar kein Einfluss haben.

\subsection{Schneeball Effekt / Network Effekt}


\todo{Network Effekt recherchieren}

Eine zweite Quelle bestätigt das Vox Populi. In \cite{subramaniamDeterminantsOpenSource2009} % Kapitel 5.1. 
wird vom sogenannten \textit{Network Effekt} gesprochen. Dieser wirkt sich laut
Subraminam et. al. positiv auf den Erfolg von OSS aus. \todoo{Network Effekt in 2. Worten erklären... den Aktiven als auch den Passiven }

\begin{quote}
    \begin{tcolorbox}[colback=black!5!white,colframe=white!75!black,title=Direkt Zitat aus \cite{subramaniamDeterminantsOpenSource2009} Kapitel 5.1.]
        The results from our study show the important role played by
        network effects of OSS
    \end{tcolorbox}
\end{quote}

\begin{hypothesis}
    Erfolgreiche Projekte werden noch erfolgreicher. => neue Projekte müssen raus stechen
    sonst haben sie keine Chance gegen die bestehenden OSS Alternativen
\end{hypothesis}
\subsection{Der richtige Zeitpunkt \colorbox{yellow}{WIP}}


%\txodoo{Zitat aus Kapitel 3.1 hernehmen und Paraphrasieren/einbauen, siehe auch Schluss von 3.1}

\begin{quote}
    \begin{tcolorbox}[colback=black!5!white,colframe=white!75!black,title=Direkt Zitat aus \cite{bangerthWhatMakesComputational2013} Kapitel 3.1]
        An interesting point made in Malcolm Gladwells book Outliers: The Story of Success [24] is that people
        are successful if their skills support products in a marketplace that is just maturing and where there is,
        consequently, still little competition. The same is certainly true for open source software projects as well:
        Projects that pick up a trend too late will have a difficult time thriving in a market that already supports other,
        large and mature projects
    \end{tcolorbox}
\end{quote}

