\subsection{Gute Dokumentation}

\rawidea
Dokumentationen spielen eine entscheidende Rolle beim Erfolg eines Projekts.
Ohne eine gute Dokumentation ist die Software schwerer zugänglich für die Benutzer und damit teils
unbrauchbar, ausgenommen Projekte mit intuitiven User Interfaces.
Mailing Listen und StackOverflow können eine gute Ergänzung zur Dokumentation sein, allerdings kann
diese dadurch nicht ersetzt werden.
Mit einem Crawler ist es schwer zu beurteilen, ob eine Dokumentation gut ist oder nicht oder ob
eine Dokumentation überhaupt existiert, da sich diese häufig auch auf der Homepage des Projekts befinden.
Man kann aber Dokumentation mit als Punkt in die Umfrage mit aufnehmen.
\citationNeeded{Könnte aus \cite{bangerthWhatMakesComputational2013} stammen}

\bigskip

\todoo{
    \noindent
    \textbf{"Wie wichtig ist eine gute Dokumentation bei der Auswahl einer OSS für Sie?"} eignet
    sich als hervorragende Frage in der Umfrage.\\
    Alternative könnte man diese Daten auch erfassen, allerdings nur von Hand.
    Da zum einen die Dokumentationen nicht immer in der README.md sind, sondern auf anderen Website
    und der Crawler nicht beurteilen kann, ob eine Dokumentation gut ist oder nicht. Daher könnte man
    quasi eine Liste zum Abhacken durchgehen
    Beispielsweise wie folgt:
    \begin{itemize}
        \item Hat das Projekt eine Dokumentation?
        \item Hat die Doku Anwendungsbeispiele?
        \item Ist ein Sandbox-Modus für dieses Projekt möglich? Wenn ja, gibt es einen?
    \end{itemize}
}

\begin{hypothesis}
    Dokumentation ist wichtig vorallem wenn die Zielgruppe Entwickler sind.
    (Gute Doku => Hoher Technischer Erfolg)
\end{hypothesis}