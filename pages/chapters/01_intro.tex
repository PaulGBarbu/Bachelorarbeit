\chapter{Einleitung}

\todo{Wird im Laufe der Arbeit noch genauer spezifizieren.}

% ? Einschränken nur auf Packages. Grund: 
%?    a) npm 
%?    b) "used by" Kategorie auf GitHub  
%?    c) kleinerer Umfang für BA

% Intro
In der IT ist Open Source mittlerweile ein fester Bestandsteil der gesamten Infrastruktur.
Mehr als die Hälfte aller Web Server laufen unter Open-Source-Lizenzen \cite{W3Techs_WebServer}.
Die meistgenutzten Frontend Frameworks sind ebenfalls alle Open Source. \cite{StackOverflowSurvey2021}


% Open Source Erklärung
Der Begriff Open Source ist den meisten Softwareentwicklern wahrscheinlich bekannt,
aber was genau steckt dahinter? Die Antwort ist weitaus mehr als \textit{nur} quelloffener Code
und kostenlose Software.
Die \textit{Open Source Initiative} hat eine klare Definition für Open Source.
Wie der Name schon sagt, muss der Quellcode offen liegen, des Weiteren gelten allerdings auch
Voraussetzungen, wie beispielsweise, dass Nutzer den Quellcode verändern und weitergeben
dürfen \cite{OpenSourceDefinition}.


% Ziel der Arbeit
Mit dieser Arbeit soll, basierend auf ausgewählten Open-Source-Projekten und einer Umfrage,
herausgefunden werden welche Faktoren entscheidend zum Erfolg eines Projektes beitragen.
Hierbei wird hauptsächlich von der Nutzerperspektive ausgegangen, wobei mit Nutzer nicht nur die
Endnutzer der Software, sondern auch Softwareentwickler gemeint sind, die Open Source Produkte wie
Bibliotheken etc. in eigenen Projekten einbauen.

% Abgrenzung
Ein zentraler Punkt dieser Ausarbeitung sind die extrinsischen sowie intrinsischen Anreize,
die Nutzer zur Auswahl eines Produktes motivieren \cite{midhaFactorsAffectingSuccess2012}. % Chapter 3.1 & 3.2
Aspekte wie die interne Führung und Organisation der Projekte wird hierbei nicht thematisiert.

% Approach
% Mithilfe eines Webcrawlers werden zunächst Daten gesammelt und analysiert.
% Zusätzlich dazu soll es eine standardisierte Umfrage geben mit dem Ziel herauszufinden welche 
% Faktoren für die Nutzer wichtig sind. Anschließend sollen die Ergebnisse gegenübergestellt werden, 
% sodass gegebenenfalls Schlussfolgerungen gezogen werden kann.



\section{Erfolg definieren}

\todo{Erfolg genauer definieren}
\bigskip

\noindent
\todoo{$\bullet$ 
    Eine weitere Erfolgsmetrik ist die Anzahl an Nutzern als auch die Anzahl an Contributorn.
    Warum? Naja wenn viele es nutzten ist es ja obviously Beliebt/Erfolgreiche.
    Und ein unerfolgreiches Projekt würde ja nicht ein haufen Contributor attracten...
}

\bigskip

\noindent
\todoo{$\bullet$ \textbf{Sponsoren}: Erfolgreiche Projekte haben Sponsoren, kann man als Indikator für Erfolg hernehmen
    aber auch als Faktor was es teils braucht um Erfolgreich zu werden. (Mehr Geld => Mehr Zeit fürs Projekt)
    Beziehungsweise wenn die Software aus einem Unternehmen kommt React => Facebook / Google => Tensorflow etc.
    Das ist zwar kein Garant für Erfolg, allerdings könnte man sich dann anschauen
    \textit{"Are Company backed OSS more successful?"}
    bzw
    \textit{"Are Sponsor backed OSS more sucessful?"} mehr in \cite{stewartImpactsLicenseChoice2006a} Hypothese 2A}

% \todoBox{
%     Erfolg muss genauer definiert werden (Sieh Mail vom 16.02.). 
%     Sieh \cite{midhaFactorsAffectingSuccess2012} Kapitel 2.1, 
%     sowie die zitierten Quellen in diesem Kapitel. 
% }

\bigskip

\noindent
\todoo{$\bullet$ \textbf{Bezüglich Erfolgskategoiren} Basically, Marktefolg = User Activity / Technsicher Erfolg = Developer Activity}

\noindent
Um Erfolg in Open Source Software (OSS) messen zu können muss Erfolg zunächst definiert werden.
Nach Midha lässt sich Erfolg in zwei Klassen unterteilen, den Markterfolg und den technischen Erfolg.
Für ein vollständiges Gesamtbild des Erfolges müssen und werden beide Klassen miteinbezogen
\cite{midhaFactorsAffectingSuccess2012}. % Kapitel 2.1 Success

\subsection{Markterfolg}

% \bigskip
% \noindent
Markterfolg wird durch Charakteristika wie Beliebtheit gekennzeichnet.
Diese Eigenschaft spiegelt sich beispielsweise in der Anzahl der Nutzer, GitHub Sterne oder
Downloads wider. Einige dieser Metriken finden sich auf den GitHub Seiten
der jeweiligen Projekte wieder und werden vom Webcrawler erfasst.
\cite{midhaFactorsAffectingSuccess2012}. % Kapitel 2.1 Success

\subsection{Technischer Erfolg}

% \bigskip
% \noindent
Unter dem technischen Erfolg zählen Eigenschaften wie Frequenz von Updates,
sowie Anzahl an Commits und Mitwirkenden. Diese Daten lassen sich ebenfalls auf GitHub erfassen.
\cite{midhaFactorsAffectingSuccess2012}. % Kapitel 2.1 Success

% \subsection{Ökosystem}
% Ein weiterer Indikator für Erfolg ist, wenn ein Projekt ein Ökosystem, um sich entwickelt.
% Beispielsweise hat das Web-Framework React eine große Community die eine Vielzahl von Erweiterungen
% unabhängig aber für React entwickeln.
% % React-Router compared to React / vs VueJS or Angular
