\documentclass[
    twoside=false, %  doppelseitiger Druck
    DIV=15,% DIV Faktor für Satzspiegelberechnung, sie Doku zu KOMA Script
    BCOR=15mm, % Bindekorrektur
    chapterprefix=false,
    headinclude=true,
    footinclude=false,
    pagesize,%         write pagesize to DVI or PDF
    fontsize=11pt,%             use this font size
    paper=a4,%          use ISO A4
    bibliography=totoc,%         write bibliography-chapter to table of contents
    index=totoc,%         write index-chapter to table of contents
    cleardoublepage=plain,% \cleardoublepage generates pages with pagestyle empty
    headings=big,%       A4/B5
    listof=flat,%        improved list of tables
    numbers=noenddot
]{scrbook}

\usepackage[anythingbreaks]{breakurl}

\usepackage[utf8]{inputenc}
\usepackage{makeidx}
\usepackage{amsfonts}
\usepackage[slantedGreek,sc]{mathpazo}  % Schriftart Palatino
% \usepackage{lmodern}    % statt mathpazo, falls CM Fonts verwendet werden sollen
%\usepackage{mathptmx}    % statt mathpazo, falls Times  verwendet werden soll
\usepackage[scaled=.95]{helvet}
\usepackage{courier}
\usepackage[T1]{fontenc}
\usepackage{textcomp}
\usepackage{amsmath}            % standard math notation (vectors/sets/...)
\usepackage{bm}        % standard math notation (fonts)
\usepackage{fixmath}        % standard math notation (fonts)
\usepackage{graphicx}
\usepackage[facing=yes]{floatrow}       % mehrere Gleitobjekte nebeneinander/caption neben Bild/Tabelle
\usepackage[labelfont=bf,sf,font=small,labelsep=space,format=plain]{caption}
\usepackage{subcaption}
\usepackage{scrlayer-scrpage}
% \usepackage{pstool}  % einbinden falls psfrag verwendet werden soll
\usepackage{epstopdf}
\usepackage[ngerman]{babel}
\usepackage{ellipsis}  % Korrigiert den Weißraum um Auslassungspunkte
\usepackage{microtype}  % optischer Randausgleich etc.

\usepackage{xcolor}         % z.B. für schattierte Boxen
\usepackage{framed}			% shaded Umgebung
\definecolor{shadecolor}{gray}{.85}%


% ------------------------------ Extra Packages ------------------------------ %
\usepackage{acronym}
\usepackage{tcolorbox}

% Multi Columns
% \usepackage[a4paper, left=1.5cm, right=1.5cm, top=3cm]{geometry}
% \usepackage{multicol}
% \setlength{\columnsep}{0.75cm}

% Hypothesen Notes
\usepackage{amsthm}
\theoremstyle{hypothesis}
\newtheorem{hypothesis}{H}
\newtheorem{subhypothesis}{H}[hypothesis]
\renewcommand{\thesubhypothesis}{\thehypothesis\alph{subhypothesis}}

% Font for Code
\usepackage{inconsolata}

% Links im PDF
\usepackage[colorlinks=false,
    pdfborder={0 0 0},
    breaklinks=true]
{hyperref}



% ------------------------------- Aus der Vorlage idk what this is ------------------------------- %

% Einstellungen für Bild-/Tabellenbeschriftung neben dem Bild
\floatsetup[figure]{capbesideposition={inside,top}}
\floatsetup[table]{capbesideposition={inside,top},style=plaintop}
\renewfloatcommand{fcapside}{figure}[\capbeside][\FBwidth]
\newfloatcommand{tcapside}{table}[\capbeside][\FBwidth]


\selectlanguage{ngerman}


\deffootnote{1em}{1em}{%
    \makebox[1em][l]{\thefootnotemark}}

\makeindex

\newcommand{\real}{\mathord{\mathrm{I\!R}}}


\selectlanguage{ngerman}

\frontmatter

\pagestyle{scrplain}
\pagestyle{empty}


% ------------------------------------------------------------------------------------------------ %

% Set Macros/Variables (kind of a shortcut to figure/tables folder)
\def\figdir{figures}
\def\tabledir{tables}