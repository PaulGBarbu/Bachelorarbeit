\chapter{Crawler}

% Beispiel wie Datenanalyse aussehen könnte: 
% https://fossa.com/blog/which-open-source-license-is-the-best-for-commercialization/
% https://docs.github.com/en/code-security/supply-chain-security/understanding-your-software-supply-chain/about-the-dependency-graph#supported-package-ecosystems

% TODO: Formulierungen + Erklären Warum GitHub

% Introduction für den Crawler

Mithilfe des Crawlers kann sowohl der Markterfolg als auch der technische Erfolg gemessen werden.
Über Metriken wie GitHub Sterne, Forks oder Downloads lässt sich der Markterfolg quantifizieren.
Der technische Erfolg findet sich wieder in Daten wie Anzahl an Commits, Frequenz von Commits oder Mitwirkende.

Bei der ersten Recherche wurde festgestellt, dass die Repositories mit den meisten Github Sternen
keine Software direkt sind, sondern Repositories mit Lerninhalten.
Zum Zeitpunkt des Schreibens (13. Februar 2022) werden Platz eins, drei und vier von
\begin{itemize}
    \item freeCodeCamp \textit{(340k Sterne)},
    \item free-programming-books \textit{(222k Sterne)} und
    \item coding-interview-university \textit{(209k Sterne)}
\end{itemize}
belegt.

\bigskip

\noindent
Insgesamt sind unter den Top 10 Repositories auf GitHub nur drei IT Projekte im eigentlichen
Sinne dabei \cite{TopGitHubProjects}.
\begin{itemize}
    \item Vue \textit{(193k Sterne)},
    \item React \textit{(182k Sterne)} und
    \item TensorFlow \textit{(163k Sterne)}.
\end{itemize}


\bigskip
\noindent
Es wird daher nicht möglich sein die Top 100 Projekte für die Datensammlung zu verwenden.
Stattdessen werden Projekte von Hand ausgewählt, um zu garantieren, dass es sich nur um IT Projekte handelt.


% TODO: Jones. 1986 finden und Zitieren, andernfalls muss man schon wieder MID12 Zitieren...
%Zudem wird der Fokus auf JavaScript Projekte liegen. Der Grund hierfür liegt darin, dass
%unterschiedliche Programmiersprachen sich schwerer miteinander vergleichen lassen \todoo{(Jones, 1986)}

% -----------------------------------------------------------------------------------------------------------------


\subsection{Der Crawler}

Hilfreiche Ressourcen:

\begin{itemize}
    \item Welche Daten könnte man erfassen und wie? Sieh \link{npms.io}{https://npms.io/about}.
    \item Hilfreiches Tool: \link{npmtrends.com}{https://www.npmtrends.com/}
    \item API um an NPM Downloads ran zu kommen:
          \link{GitHub registry}{https://github.com/npm/registry/blob/master/docs/download-counts.md}
          bzw.
          \link{Deprecated registry}{https://github.com/npm/download-counts}
\end{itemize}



\subsection{Disclaimer}

npm downloadzahlen sind manipulierbar, ich geh zwar davon aus, dass es allgemein nicht häufig gemacht wird
und da ja sowieso vielmehr in die rechnung rein geht als nur npm downloads. Werden diese daten trotzdem mit erfasst.
\link{How to exploit NPM}{https://dev.to/andyrichardsonn/how-i-exploited-npm-downloads-and-why-you-shouldn-t-trust-them-4bme}.

In einem 
\link{Blogpost}{https://blog.npmjs.org/post/92574016600/numeric-precision-matters-how-npm-download-counts-work.html}
erklärt NPM wie es zu den Download zahlen zustande kommt.


\section{GitHub API}
GitHub stellt eine API bereit welche man zum Analysieren von Daten verwenden kann, dies wird auch um
einiges stabiler funktionieren als ein Web Crawler. Hier sind einige Endpoints die sehr Praktisch sein könnten.
Mehr gibt es \link{hier}{https://docs.github.com/en/rest/reference}. Es wäre wahrscheinlich, sich
einen \link{Access Token}{https://docs.github.com/en/authentication/keeping-your-account-and-data-secure/creating-a-personal-access-token}
zu besorgen, ohne sind 60 Aufrufe pro Stunde möglich, mit Token 5000/h.

Als Alternative zur REST API gibt es auch eine \link{GraphQL API}{https://docs.github.com/en/graphql}. Diese ist etwas präziser und sollte auch
z.B. Stars over time geben können.

\begin{itemize}
    \item \link{Lizenzen}{https://docs.github.com/en/rest/reference/licenses\#get-the-license-for-a-repository}
    \item \link{Project Health und mehr}{https://docs.github.com/en/rest/reference/metrics\#get-community-profile-metrics}
    \item \link{Page Views}{https://docs.github.com/en/rest/reference/metrics\#get-page-views}
    \item \link{Contributers}{https://stackoverflow.com/a/36462386/16940103} \textsubscript{1}
    \item \link{Commits}{https://docs.github.com/en/rest/reference/commits\#list-commits} \textsuperscript{2}
    \item \link{Stars}{https://docs.github.com/en/rest/reference/activity\#list-stargazers}
\end{itemize}

\noindent
\textsuperscript{1} Um an die gesamtzahl der Contributer ran zu kommen ist es ein bisschen tricky, sieh Stackoverflow link
ich habe dazu auch nicht den Link in der Doku gefunden aber das wird da schon irgendwo sein. Mehr dazu auch
\link{hier}{https://docs.github.com/en/rest/guides/traversing-with-pagination}

\bigskip

\noindent
\textsuperscript{2} Hier muss man für Gesamt Commits den gleichen Trick anwenden nehme ich an wie bei den Contributorn.
zusätzlich dazu lässt sich auch eine Zeitspanne bestimmen!

\bigskip

Zusätzlich oder alternativ lässt sich auch mit den \link{GraphQL Endpoints}{https://docs.github.com/en/graphql/overview/explorer} 
arbeiten. Hier gibt es zwei Beispiele wie andere es gemacht haben. \link{star-history}{https://star-history.com/}
und \link{vesoft-inc}{https://vesoft-inc.github.io/github-statistics/}. (Wäre praktische diese wahrscheinlich zu Zitieren,
die Idee ist ja nicht so ganz meine I guess so ya know). Wie das erste funktioniert weiß ich nicht genau, aber das zweite
Tool verwendet GraphQL. Es konnte eine Grafik für React anzeigen von Anfang bis heute mit Sternen, allerdings hat der Request
auch irgendwie idk 10min gedauert, aber bei 187k Sternen und damit dem meisten ist das schon durchaus verkraftbar.
Die Commits lagen nur ein Jahr zurück, könnte aber eine Implementierungssache sein, da beide Projekte Open Source sind
kann man in den Code reinschauen und sich was daraus gönnen. Außerdem gibt es noch folgendes Tool: \link{starline}{https://github.com/mawrkus/starline}

% TODO: Frage
% \frage{
%     Soll ich einzelne Projekte unter die Lupen nehmen, wenn Ja
%     1. Bei der Analyse würde ich auf folgende Punkte eingehen...(*) "reicht das?"
%     2. Wie viele solcher Analysen sollte ich durchführen
% }

% ? Beschränkung auf JS
% \todoo{Ggf. könnte man sich nur auf JavaScript beschränken... Warum?
%     Sieh \cite{midhaFactorsAffectingSuccess2012} Kapitel 4.1 => Verschiedene Sprachen
%     lassen sich schwer miteinander vergleichen.
%     Warum JavaScript? Weil NPM die Anzahl an Downloads anzeigt (Markterfolg) und ich persönlich mich am
%     besten damit auskenne.
%     Java zum Beispiel hat Maven und Gradle. Was das Sammeln von Daten um einiges schwerer machen würde
%     zudem zeigt, Maven keine Downloads zahlen an
%     Market Success vs Technical Success == Download zahlen vs Anzahl an commits in einem Zeitraum.
%     Hierbei lohnt sich ein Blick in \cite{midhaFactorsAffectingSuccess2012} Kapitel 4.2}

% ? What data to collect?
% \begin{itemize}
%     \item Welche Daten werden erhoben
%           \begin{itemize}
%               \item Anzahl an GitHub Stars
%               \item Anzahl an Mitwirkenden
%               \item Anzahl an Commits nach Mitglied => Main Contributor
%               \item Anzahl der Nutzer
%               \item Lizenzen
%               \item Welche Programmiersprache wurde verwendet
%               \item Anzahl an Zeilen commented
%               \item Anzahl an gesamten Commits / Regelmäßigkeit der Commits
%               \item Anzahl der Tickets => Regelmäßigkeit in der Tickets abgearbeitet werden
%               \item Was für ein Projekt ist das? Library/ Framework / Modul / Applikation ... (*)
%               \item Wem gehört das Projekt? Einer Einzelnen Person, einem unabhängigen Team oder einem Unternehmen? (*)
%           \end{itemize}
%     \item evtl. wie wurde der Crawler entwickelt\dots
%     \item Analyse der Daten (Graphen etc.)
%     \item ggf. genauere Analyse von Projekten. Entweder von
%           \begin{itemize}
%               \item den Top Projekte
%               \item Top Projekten aufgeteilt nach Branchen
%           \end{itemize}
%     \item *Diese Punkte müssen ggf. von Hand erhoben werden, das passt dann zum Punkt "Genauere Analyse"
% \end{itemize}

% ? Link zu Top GitHub Projekten
% \href{https://github.com/search?l=&o=desc&q=stars%3A%3E100+stars%3A%3E1&s=stars&type=Repositories}{Link zu Github Top Repos}






