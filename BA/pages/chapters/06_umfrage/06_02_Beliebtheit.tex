%! newpage
\cleardoublepage %! \cleardoublepage

\section{Beliebtheit}\label{sec:Beliebtheit}
Im zweiten Teil der Umfrage, soll geklärt werden, welchen Einfluss die Beliebtheit, bei der Wahl
eines Projektes hat. Wie zuvor auch sollten die Teilnehmer Kriterien auf einer 5-Punkte Skala bewerten.
Die Mittelwerte der Antworten finden sich in Tabelle \ref{tab:beliebtheit} wieder.

\bigskip
\noindent
Die Aussage war \textit{"Wie sehr achten Sie bei der Auswahl von OSS auf..."}, mit den Punkten:
\begin{multicols}{2}
    \begin{itemize}
        \setlength\itemsep{0em}
        \item Anzahl der GitHub Sterne
        \item Anzahl der Mitwirkenden eines
              Projektes
        \item Anzahl von Sponsoren
        \item Trends
        \item Anzahl an Fragen und Antworten auf\\ StackOverflow
    \end{itemize}
\end{multicols}

\noindent
Diese Punkte wurden gewählt, da es klassischen Vergleichsmerkmalen sind, um die Beliebtheit von OSS
zu vergleichen. GitHub Sterne und Downloads sind eine schnelle und einfache Methode, um die aktuelle 
Beliebtheit eines Projektes zu vergleichen. Trends hingegen stellen GitHub Sterne oder Downloads 
über Zeit dar, häufig genutzte Tools für die Analyse von Trends sind
\textit{StackOverflow Trends}\footnote{\url{https://insights.stackoverflow.com/trends}},
\textit{npm trends}\footnote{\url{https://www.npmtrends.com/}} oder
\textit{Google Trends}\footnote{\url{https://trends.google.de/trends/}}.


% Daten beschreiben
Anders als bei der Dokumentation gibt es hier kein Kriterium mit starker Zustimmung. Die Downloads
sind hierbei mit einer durchschnittlich Zustimmung von 3,27 die beliebtes Vergleichsmetrik für Beliebtheit
gefolgt von Sternen (2,92). Anzahl der Mitwirkenden (2,68), StackOverflow Fragen und Antworten (2,60) sowie 
Trends (2,23) werden tendenziell weniger beachtet. Die Anzahl der Sponsoren wird mit einer durchschnittlichen
Bewertung (1,7) fast gar nicht beachtet. 


Des Weiteren gaben 46\% der Teilnehmer an, dass die geringer Popularität eines Projekts sie schon 
mal davon abgehalten hat dieses zu nutzten.  

% Die leichte Ablehnung der Kriterien deckt sich mit diesem Ergebnissen. Nutzer achten auf Beliebtheit weniger als auf die 
% Qualität der Dokumentation bei der Wahl von OSS.

\begin{table}[h]
    %\resizebox{\textwidth}{!}{%
        \begin{tabular}{cccccc}
            \hline
            Downloads & Sterne & Mitwirkende & StackOverflow Fragen/Antworten & Trends & Sponsoren \\ \hline
            3.27      & 2.92   & 2.68        & 2.60                           & 2.23   & 1.7
        \end{tabular}%
    %}
    \caption{\label{tab:beliebtheit}Einfluss der Beliebtheitsmerkmalen bei der Wahl von OSS}
\end{table}