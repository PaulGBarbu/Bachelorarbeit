%! newpage
\cleardoublepage %! \cleardoublepage

\section{Beliebtheit}\label{sec:Beliebtheit}
Im zweiten Teil der Umfrage ging es, um den Einfluss der Beliebtheit bei der Wahl eines OS-Projektes.
Wie zuvor auch sollten die Teilnehmenden Aussagen auf einer 5-Punkte Skala bewerten.
Die Fragestellung war: \textit{"Wie sehr achten Sie bei der Auswahl von OSS auf..."}.


\begin{multicols}{2}
    \begin{itemize}
        \setlength\itemsep{0em}
        \item Anzahl der GitHub Sterne
        \item Anzahl der Contributor eines
              Projektes
        \item Anzahl von Sponsoren
        \item Trends
        \item Anzahl an Fragen und Antworten auf\\ StackOverflow
    \end{itemize}
\end{multicols}

\noindent
\todoo{Diese Punkte wurden gewählt basierend auf den gestellten Hypothesen. GitHub Sterne und Downloads
    sind eine schnelle und einfache Methode die Beliebtheit eines Projektes zu....
    Trends beziehen sich auch auf GitHub Sterne oder Downloads allerdings zeigen sie die Beliebtheit über
    Zeit an, statt nur als Snapshot der aktuellen Beliebtheit. Häufig genutzte Tools hierfür sind
    \textit{StackOverflow Trends}\footnote{https://insights.stackoverflow.com/trends},
    \textit{npm trends}\footnote{https://www.npmtrends.com/} oder
    \textit{Google Trends}\footnote{https://trends.google.de/trends/}}



\todo{Describe it like you dont see the table: Downloads ganz oben, Keiner shaut auf Sponsoren...}

\bigskip
\noindent
Anders als bei der Dokumentation liegt die durchschnittliche Bewertung aller Punkte im \todoo{Mittelfeld}
die höchste Bewertung hast Downloads mit durchschnittlich 3,27.
Außerdem gab es auch hier die Ja/Nein Frage \textit{"Hat die geringe Popularität eines Projekts Sie
    jemals davon abgehalten, es zu nutzen?"}. Hier haben 46\% der Teilnehmer mit \textit{"Ja"} gestimmt.
Insgesamt wirkt sich die Beliebtheit schwächer auf den Entscheidungsprozess aus, als die Qualität der
Dokumentation.

\begin{table}[h]
    \resizebox{\textwidth}{!}{%
        \begin{tabular}{cccccc}
            \hline
            Downloads & Sterne & Contributor & StackOverflow Fragen/Antworten & Trends & Sponsoren \\ \hline
            3.27      & 2.92   & 2.68        & 2.60  \vspace{0.3cm}           & 2.23   & 1.7       \\
            +0.27     & -0.08  & -0.32       & -0.40                          & -0.77  & -1.3
        \end{tabular}%
    }
    \caption{\label{tab:beliebtheit}Einfluss von Beliebtheitsmerkmalen}
\end{table}