\section{Sponsoren}\label{sec:umfrage_sponsoren}

\todo{Fliegt evtl ganz raus?}

Im dritten Teil der Umfrage ging es darum herauszufinden welchen Einfluss Sponsoren bei der OSS Wahl
spielen. Auch hier wurde mit der 5-Punkte Skala bewertet, mit der Frage: \textit{"Bewerten Sie folgende
    Aussagen."} Die Aussagen waren:

\begin{enumerate}
    \setlength\itemsep{0em}
    \item Projekte mit Sponsoren wirken zukunftssicherer
    \item Ich bevorzuge es, wenn möglich, Projekte mit Sponsoren zu nutzten
    \item Gesponserte Projekte sind meistens qualitativ besser
\end{enumerate}

\begin{table}[h]
    \begin{tabular}{cccccc}
        \hline
        Aussage: \hspace{1cm} & 1.   & \hspace{0.75cm} & 2.   & \hspace{0.75cm} & 3.   \\ \hline
                              & 3.27 &                 & 2.92 &                 & 2.68
    \end{tabular}%
    \caption{\label{tab:sponsorens}Einfluss von Sponsoren}
\end{table}


\todoo{Umformulieren, Ausformulieren...}
\noindent
\todoo{Wie in Kapitel \ref{sec:Beliebtheit} dargelegt, wird auf das Vorhandensein von Sponsoren wenig
    geachtet. Hier soll es aber darum gehen wie das Vorhandensein von Sponsoren empfunden wird.
    Am ehesten wird die Präsenz von Sponsoren mit zukunftssicherer verbunden. Qualitativ werden werden
    gesponserte Projekte nicht als besser empfunden. \textit{Meine Vermutung liegt hier in der Tatsache, das
        nicht gesponserte Projekte ebenfalls Qualitativ hochwertig sein können/sind. Somit hängt Qualität nicht von
        Geld ab...}}

Wie in Kapitel \ref{sec:Beliebtheit} dargelegt, wird auf das Vorhandensein von Sponsoren weniger geachtet.
Hier erkennt man womöglich den Grund dafür.
Gesponserte Projekte werden weder bevorzugt, noch als qualitativ
hochwertiger betrachtet, sprich es wird auf Sponsoren nicht geachtet.
