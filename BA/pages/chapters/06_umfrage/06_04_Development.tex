\section{Development}

Im vorletzten Teil der Umfrage ging es um den Einfluss des Entwicklungsprozesses bei der Auswahl von
OSS. Die Frage war: \textit{"Wie sehr achten Sie auf die folgenden Punkte, wenn Sie ein Projekt
    wählen?"}

\begin{multicols}{2}
    \begin{itemize}
        \setlength\itemsep{0em}
        \item Anzahl aktiver Maintainer
        \item Ticket/Issue Verhältnis Open/Closed
        \item Antwortzeit der Entwickler auf Tickets/Issues
        \item Regelmäßigkeit der Commits
        \item Aktualität des letztes Commit
        \item []
    \end{itemize}
\end{multicols}

\noindent
Diese Punkte wurden gewählt, da es leicht einsehbare Daten sind, die den Entwicklungsprozess eines 
Projektes widerspiegeln.

Ähnlich wie bei der Dokumentation legen die Nutzer einen große Wert auf Aktualität 
(vgl. Tabellen \ref{tab:freitext_felder_ergebnisse})
Mit einer durchschnittlichen Zustimmung von 3,94 ist diese das wichtigste Kriterium des 
Entwicklungsprozesses, gefolgt von Anzahl der Maintainer (3,33), Regelmäßigkeit der Commits (3,28),
Verhältnis von offenen und abgeschlossenen Tickets (2,90) und Antwortzeit der Entwickler auf Tickets
(2,86).


\begin{table}[ht]
    %\resizebox{\textwidth}{!}{%
    \begin{tabular}{ccccccc}
        \hline
        Aktualität & Anzahl der Maintainer & Regelmäßigkeit & Verhältnis & Antwortzeit \\ \hline
        3.94       & 3.33                  & 3.28           & 2.90       & 2.86
    \end{tabular}%
    %}
    \caption{Einfluss des Entwicklungsprozesses bei der Wahl von OSS}
    \label{tab:development}
\end{table}

