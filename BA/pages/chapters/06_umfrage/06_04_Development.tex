\section{Development}
Im vierten Teil der Umfrage ging es um den Einfluss des Entwicklungsprozesses bei der Auswahl von
OSS. Die Frage war: \textit{"Wie sehr achten Sie auf die Folgenden Punkte, wenn Sie ein Projekt
    wählen?"}

\todo{Warum genau diese Punkte?}

\begin{multicols}{2}
    \begin{itemize}
        \setlength\itemsep{0em}
        \item Anzahl aktiver Maintainer
        \item Ticket/Issue Verhältnis Open/Closed
        \item Antwortzeit der Entwickler auf Tickets/Issues
        \item Regelmäßigkeit der Commits
        \item Aktualität des letztes Commit
        \item []
    \end{itemize}
\end{multicols}

\noindent
\todoo{Die gewählten Punkte sind auf GitHub einsehbare Daten, welche als Metriken der Entwicklung
    verwendet werden können.}

\begin{table}[ht]
    %\resizebox{\textwidth}{!}{%
    \begin{tabular}{ccccccc}
        \hline
        Aktualität & Anzahl der Maintainer & Regelmäßigkeit & Verhältnis & Antwortzeit \\ \hline
        3.94       & 3.33                  & 3.28           & 2.90       & 2.86
    \end{tabular}%
    %}
    \caption{\label{tab:development}Einfluss des Entwicklungsprozesses}
\end{table}

\noindent
Ähnlich wie bei der Dokumentation (Sieh Tabellen \ref{tab:freitext_felder_ergebnisse}) legen die Nutzer
einen großen Wert auf Aktualität. \todoo{Gefolgt von Anzahl der Maintainer und Regelmäßigkeit der
    Commits}
Auf Tickets wird hierbei weniger geachtet, weder das Verhältnis zwischen offenen und geschlossen
Tickets noch auf die Antwortzeit der Entwickler auf diese.