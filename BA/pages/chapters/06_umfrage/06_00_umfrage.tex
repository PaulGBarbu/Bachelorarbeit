\chapter{Umfrage}

\todo{Ausformulieren}

% Introduction Umfrage
Als Ergänzung zur Datenerfassung der GitHub Projekte wurde auch eine Umfrage durchgeführt.
Ziel der Umfrage war es herauszufinden worauf die Nutzer von Open Source achten, wenn sie ein Projekt
wählen. Das übergeordnete Ziel war es die Hypothesen H3, H6 und H8 zu prüfen.
An der Online-Umfrage haben Softwareentwickler der adesso SE, sowie Studenten der TH Rosenheim teilgenommen.
Insgesamt erhielt die Umfrage 308 Antworten. Zur Erstellung der Umfrage wurde \textit{cryptpad.org} 
verwendet \cite{Cryptpad_org}.

Hierbei sollten die Teilnehmer Aussagen zu den Themen Dokumentation, Beliebtheit, Sponsoren und
Entwicklung bewerten. Als Bewertungsskala wurde eine 5-Punkte Likert-Skala verwendet \cite{likertScale}.


% \todoo{Interpretation der Durchschnitte: 1 = der Teilnehmende stimmt der Aussage gar nicht zu, 5 =
%     der Teilnehmende stimmt der Aussage vollkommend zu. 3 = Undecided, doesnt matter etc.}



% Dokumentation
\section{Dokumentation}

% Was mussten die Teilnehmer machen?
Im ersten Teil der Umfrage mussten die Teilnehmenden zwei Aussagen bewerten.


\bigskip
\noindent
Die erste Aussage \textit{"Was macht gute Dokumentation aus?"}, mit den \todoo{Punkten}:
\begin{multicols}{2}
    \begin{itemize}
        \setlength\itemsep{0em}
        \item Übersichtlichkeit
        \item Einfache Sprache
        \item Live-Demos
        \item Übersetzungen (z.B. Englisch -> Deutsch)
        \item Das Vorhandsein einer Getting Started Page
        \item Code Beispiele
        \item Strukturierung der Dokumentation
        \item []
    \end{itemize}
\end{multicols}

\bigskip
\noindent
Die zweite Aussage war \textit{"Was macht schlechte Dokumentation aus?"}, mit den \todoo{Punkten}:

\begin{multicols}{2}
    \begin{itemize}
        \setlength\itemsep{0em}
        \item Keine Code Beispiele
        \item Schlecht strukturierte Dokumentation
        \item Dokumentation zu komplex
        \item Fehlende Übersetzung (z.B. fehlende deutsche Übersetzung)
        \item Fehlende Getting Started Page
    \end{itemize}
\end{multicols}


\todoo{Diese Punkte wurden gewählt basierend auf Beobachtungen vieler Dokumentationen, während des "Einsammel" prozesses
    sowie Allgemein erfahrung die ich als Entwickler hab. Manche Dokumentationen sind strukturierter als andere, Manche
    haben Code-Beispiele manche nicht, manche haben Live-Demos manche nicht. Hiermit wollte ih quasi herausfinden welche
    punkte die wichtigsten sind für die Befragen. Wie sich herasusstellt sind es Code-Beispiele......}

\todo{Etwas genauer erklären was hier gewichtet heißt}


\bigskip
\noindent
% More Detail
%! Die Referenzen sind aktuell RIP !!!!!!!!!!!!!!!!!!!!!!!!!!!!!!!!!!!!!!!!!!!!!!!!!!!!!!!!!!!!!!!!!!!!!!!!!!!!!!!!!!!!!
Die Aussagen wurden auf einer 5-Punkte Skala bewertet, hierbei entspricht 1 \textit{Stimme gar nicht zu}
und 5 \textit{Stimme vollkommend zu}. In den Tabellen \ref{tab:gute_doku} und \ref{tab:schlechte_doku}
finden sich die Antworten auf die jeweiligen Aussagen als gewichteten Mittelwerte.

% Conclusion
Die Umfrage hat gezeigt, dass Code Beispiele und Übersichtlichkeit bzw. Struktur die wichtigsten
Eigenschaften einer guten Dokumentation sind.
Die Teilnehmer bewerteten diese als wichtige Aspekte einer guten Dokumentation, während die
Abwesenheit dieser Aspekte tendenziell Nutzer motivieren würde sich nach Alternativen umzuschauen.
Eine Getting-Started Page wird tendenziell als wichtig empfunden, dessen Abwesenheit wird allerdings
weniger kritisch betrachtet als die vorher genannten. Eine Getting-Started Page bewegt sich basierend
auf diesen Daten daher eher in die Richtung \textit{Nice-To-Have}, während Code Beispiele und Struktur
\textit{Must-Have} sind.

% ---------------------------------------------------- %
% Likert to Kano (kinda)
% https://www.eric-klopp.de/texte/die-kano-methode.php
% ---------------------------------------------------- %


% ----------------------------------------- FreitextFeld ----------------------------------------- %
Zudem gab es jeweils ein Freitextfeld, wo die Teilnehmenden die Möglichkeit hatten weitere Aspekte
einer guten bzw. schlechten Dokumentation zu nennen. Diese Freitextfelder waren optional und wurden
von  95 bzw. 71 von den gesamten 308 Teilnehmenden genutzt.

\bigskip
\noindent
% Most Used Words
In den Tabellen \ref{tab:freitext_felder_ergebnisse} (a) und (b) finden sich die am häufigsten genannten
Punkte der Freitextfelder. Aktualität, Vollständigkeit und UX waren die am häufigsten genannten Punkte
bezüglich guter Dokumentation, die jeweiligen pendants, veraltete Dokumentation, unvollständige
Dokumentation und schlechte UX waren die häufigsten genannten Gründe, um ein OSS Projekt nicht zu
nutzten. Somit ergänzen diese drei Punkte die vorhin erwähnten \textit{Must-Haves}.

\todo{Diesen Fact etwas schöner einbinden}

\bigskip
\noindent
\todoo{Die letzte Frage zum Thema Dokumentation war
    \textit{"Hat eine schlechte Dokumentation Sie jemals dazu gebracht, ein alternatives Projekt zu wählen?"}.\\
    81\% der Teilnehmer haben diese Frage mit \textit{"Ja"} beantwortet.}

\subsubsection*{Anmerkung:}
Im Fall von UX wurden folgende Punkte zusammengefasst: Suchfunktion/Verlinkung, Design und
Übersichtlichkeit.

% Gutes Zitat tbh.
% "Gar keine Dokumentation ist ein Zeichen von fehlendem Engagement/Einsatz - nicht vertrauenswürdig"

\newpage
% % Tabelle gute Dokumentation
% \begin{table}[ht]
%     \resizebox{\textwidth}{!}{%
%         \begin{tabular}{ccccccc}
%             \hline
%             Code Beispiele & Übersichtlichkeit & Struktur & Getting Started & Einfache Sprache & Live-Demo & Übersetzung \\ \hline
%             4.36           & 4.36              & 4.10     & 3.98            & 3.09             & 2.91      & 1.54
%         \end{tabular}%
%     }
%     \caption{\label{tab:gute_doku}Was macht gute Dokumentation aus?}
% \end{table}

% \bigskip

% % Tabelle schlechte Dokumentation
% \begin{table}[ht]
%     \resizebox{\textwidth}{!}{%
%         \begin{tabular}{ccccccc}
%             \hline
%             Keine Code Beispiele & Schlechte Struktur & Komplexe Doku. & Keine Getting Started & Keine Übersetzung \\ \hline
%             4.08                 & 3.92               & 3.30           & 3.10                  & 1.38
%         \end{tabular}%
%     }
%     \caption{\label{tab:schlechte_doku}Was kennzeichnet schlechte Dokumentation?}
% \end{table}


% ------------------------------------------ Bar Charts ------------------------------------------ %
\begin{figure}[h]
    \centering
    \includegraphics[scale=0.05]{figures/05/GuteDoku_BarChart.png}
    \caption{Antworten: Was zeichnet Gute Dokumentation aus?}
    \label{abb:GuteDoku_BarChart}
\end{figure}

\begin{figure}[h]
    \centering
    \includegraphics[scale=0.05]{figures/05/SchlechteDoku_BarChart.png}
    \caption{Antworten: Was zeichnet Gute Dokumentation aus?}
    \label{abb:SchlechteDoku_BarChart}
\end{figure}


% Tabelle: Most Used Words
\begin{figure}[tph]
    \todo{Die zwei Tabellen unten werden als \textbf{Abbildung} gekennzeichnet ! Statt Tabellen
    Copy Paste einfach die Tabelle 4.1 Lizenzen der erfassten Projekte}
    \begin{minipage}[t]{0.4\textwidth}\vspace{0pt}%
        \subcaptionbox{\label{tab:gute_doku_freitext}Freitext: Gute Dokumentation}{
            \begin{tabular}{lc}
                \firsthline
                Aktualität          & 28 \\ \hline
                Vollständigkeit     & 21 \\ \hline
                Gute UX             & 14 \\ \hline
                Versionierung       & 10 \\ \hline
                Changelog vorhanden & 6  \\ \hline
                Gute Code Beispiele & 5  \\ \hline
            \end{tabular}% 
        }
    \end{minipage}
    %
    \hfill
    %
    \begin{minipage}[t]{0.5\textwidth}\vspace{0pt}%
        \subcaptionbox{\label{tab:schlechte_doku_freitext}Freitext: Schlechte Dokumentation}{
            \begin{tabular}{lc}
                \firsthline
                Veraltet                           & 25 \\ \hline
                Unvollständig                      & 15 \\ \hline
                Schlechte UX                       & 11 \\ \hline
                Fehlerhaft                         & 7  \\ \hline
                Keine Doku vorhanden               & 6  \\ \hline
                Code Beispiele Funktionieren nicht & 5  \\ \hline
            \end{tabular}% 

        }
    \end{minipage}
    \captionabove{\label{tab:freitext_felder_ergebnisse}Häufigste Erwähnungen beim Freitext}
\end{figure}


% Beliebtheit
\newpage
%! newpage
\cleardoublepage %! \cleardoublepage

\section{Beliebtheit}\label{sec:Beliebtheit}
Im zweiten Teil der Umfrage, soll geklärt werden, welchen Einfluss die Beliebtheit, bei der Wahl
eines Projektes hat. Wie zuvor auch sollten die Teilnehmer Kriterien auf einer 5-Punkte Skala bewerten.
Die Mittelwerte der Antworten finden sich in Tabelle \ref{tab:beliebtheit} wieder.

\bigskip
\noindent
Die Aussage war \textit{"Wie sehr achten Sie bei der Auswahl von OSS auf..."}, mit den Punkten:
\begin{multicols}{2}
    \begin{itemize}
        \setlength\itemsep{0em}
        \item Anzahl der GitHub Sterne
        \item Anzahl der Mitwirkenden eines
              Projektes
        \item Anzahl von Sponsoren
        \item Trends
        \item Anzahl an Fragen und Antworten auf\\ StackOverflow
    \end{itemize}
\end{multicols}

\noindent
Diese Punkte wurden gewählt, da es klassischen Vergleichsmerkmalen sind, um die Beliebtheit von OSS
zu vergleichen. GitHub Sterne und Downloads sind eine schnelle und einfache Methode, um die aktuelle 
Beliebtheit eines Projektes zu vergleichen. Trends hingegen stellen GitHub Sterne oder Downloads 
über Zeit dar, häufig genutzte Tools für die Analyse von Trends sind
\textit{StackOverflow Trends}\footnote{\url{https://insights.stackoverflow.com/trends}},
\textit{npm trends}\footnote{\url{https://www.npmtrends.com/}} oder
\textit{Google Trends}\footnote{\url{https://trends.google.de/trends/}}.


% Daten beschreiben
Anders als bei der Dokumentation gibt es hier kein Kriterium mit starker Zustimmung. Die Downloads
sind hierbei mit einer durchschnittlich Zustimmung von 3,27 die beliebtes Vergleichsmetrik für Beliebtheit
gefolgt von Sternen (2,92). Anzahl der Mitwirkenden (2,68), StackOverflow Fragen und Antworten (2,60) sowie 
Trends (2,23) werden tendenziell weniger beachtet. Die Anzahl der Sponsoren wird mit einer durchschnittlichen
Bewertung (1,7) fast gar nicht beachtet. 


Des Weiteren gaben 46\% der Teilnehmer an, dass die geringer Popularität eines Projekts sie schon 
mal davon abgehalten hat dieses zu nutzten.  

% Die leichte Ablehnung der Kriterien deckt sich mit diesem Ergebnissen. Nutzer achten auf Beliebtheit weniger als auf die 
% Qualität der Dokumentation bei der Wahl von OSS.

\begin{table}[h]
    %\resizebox{\textwidth}{!}{%
        \begin{tabular}{cccccc}
            \hline
            Downloads & Sterne & Mitwirkende & StackOverflow Fragen/Antworten & Trends & Sponsoren \\ \hline
            3.27      & 2.92   & 2.68        & 2.60                           & 2.23   & 1.7
        \end{tabular}%
    %}
    \caption{\label{tab:beliebtheit}Einfluss der Beliebtheitsmerkmalen bei der Wahl von OSS}
\end{table}

% Sponsoren
% \newpage
\section{Sponsoren}
Im dritten Teil der Umfrage ging es darum herauszufinden welchen Einfluss Sponsoren bei der OSS Wahl
spielen. Auch hier wurde mit der 5-Punkte Skala bewertet, mit der Frage: \textit{"Bewerten Sie folgende
    Aussagen."} Die Aussagen waren:

\begin{enumerate}
    \setlength\itemsep{0em}
    \item Projekte mit Sponsoren wirken zukunftssicherer
    \item Ich bevorzuge es, wenn möglich, Projekte mit Sponsoren zu nutzten
    \item Gesponserte Projekte sind meistens qualitativ besser
\end{enumerate}

\begin{table}[]
    \begin{tabular}{cccccc}
        \hline
        Aussage: \hspace{1cm} & 1.   & \hspace{0.75cm} & 2.   & \hspace{0.75cm} & 3.   \\ \hline
                              & 3.27 &                 & 2.92 &                 & 2.68
    \end{tabular}%
    \caption{\label{tab:sponsorens}Einfluss von Sponsoren}
\end{table}


\todoo{Umformulieren, Ausformulieren...}
\noindent
\todoo{Wie in Kapitel \ref{sec:Beliebtheit} dargelegt, wird auf das Vorhandensein von Sponsoren wenig
    geachtet. Hier soll es aber darum gehen wie das Vorhandensein von Sponsoren empfunden wird.
    Am ehesten wird die Präsenz von Sponsoren mit zukunftssicherer verbunden. Qualitativ werden werden
    gesponserte Projekte nicht als besser empfunden. \textit{Meine Vermutung liegt hier in der Tatsache, das
        nicht gesponserte Projekte ebenfalls Qualitativ hochwertig sein können/sind. Somit hängt Qualität nicht von
        Geld ab...}}

Wie in Kapitel \ref{sec:Beliebtheit} dargelegt, wird auf das Vorhandensein von Sponsoren weniger geachtet.
Hier erkennt man womöglich den Grund dafür.
Gesponserte Projekte werden weder bevorzugt, noch als qualitativ
hochwertiger betrachtet, sprich es wird auf Sponsoren nicht geachtet.


% Development
\newpage
\section{Development}
Im vierten Teil der Umfrage ging es um den Einfluss des Entwicklungsprozesses bei der Auswahl von
OSS. Die Frage war: \textit{"Wie sehr achten Sie auf die Folgenden Punkte, wenn Sie ein Projekt
    wählen?"}

\todo{Warum genau diese Punkte?}

\begin{multicols}{2}
    \begin{itemize}
        \setlength\itemsep{0em}
        \item Anzahl aktiver Maintainer
        \item Ticket/Issue Verhältnis Open/Closed
        \item Antwortzeit der Entwickler auf Tickets/Issues
        \item Regelmäßigkeit der Commits
        \item Aktualität des letztes Commit
        \item []
    \end{itemize}
\end{multicols}

\noindent
\todoo{Die gewählten Punkte sind auf GitHub einsehbare Daten, welche als Metriken der Entwicklung
    verwendet werden können.}

\begin{table}[ht]
    %\resizebox{\textwidth}{!}{%
    \begin{tabular}{ccccccc}
        \hline
        Aktualität & Anzahl der Maintainer & Regelmäßigkeit & Verhältnis & Antwortzeit \\ \hline
        3.94       & 3.33                  & 3.28           & 2.90       & 2.86
    \end{tabular}%
    %}
    \caption{\label{tab:development}Einfluss des Entwicklungsprozesses}
\end{table}

\noindent
Ähnlich wie bei der Dokumentation (Sieh Tabellen \ref{tab:freitext_felder_ergebnisse}) legen die Nutzer
einen großen Wert auf Aktualität. \todoo{Gefolgt von Anzahl der Maintainer und Regelmäßigkeit der
    Commits}
Auf Tickets wird hierbei weniger geachtet, weder das Verhältnis zwischen offenen und geschlossen
Tickets noch auf die Antwortzeit der Entwickler auf diese.


\newpage %! new page
\cleardoubleemptypage
\section{Freie Kategorisierung der Erfolgskriterien}\label{sec:umfrage_last_question}


Im letzten Teil der Umfrage sollten die Teilnehmer verschiedene Kriterien per
\textit{Drag and Drop} sortieren. Die Fragestellung lautete
\textit{Sortieren Sie nach den für Sie wichtigsten Kriterien bei der Auswahl von OSS.}
mit folgenden Punkten zum Sortieren


\begin{multicols}{2}
    \begin{itemize}
        \setlength\itemsep{0em}
        \item Gute Dokumentation
        \item Beliebtheit
        \item Anzahl von offenen Issues/Tickets
        \item Trends
        \item Projekt hat Sponsoren
        \item []
    \end{itemize}
\end{multicols}


\noindent
Aufgrund der Übersichtlichkeit hat sich die Auswahl hier auf fünf beschränkt. Es wurde versucht
alle vorherigen Themen abzudecken.
Je nach Platzierung, haben die Kriterien Punkte bekommen, 5 Punkte für den ersten Platz, 4 für den
zweiten usw. der letzte Platz bekam einen Punkt.

Das wichtigste Kriterium ist Gute Dokumentation mit 1329 Punkten, gefolgt von Beliebtheit (1165),
Anzahl der offenen Issues/Tickets (794), Trends (624), Projekt hat Sponsoren (559).
