\section{Lizenzen}


\todo{Klare Unterscheidung zwischen Restriktiv und Permissive, aus Kapitel Datenerfassung
soll man hierdrauf verlinken können.}

\todo{Offene LIzenzen => Freizügig oder Permissive Lizenz}

% Lizenzen haben signifikaten Einfluss auf Erfolg [Subramaniam]
Laut Subramaniam et al. spielen Lizenzen eine signifikante Rolle für den Erfolg von
Open Source Software.
Freie Lizenzen wie MIT oder BSD haben einen positiven Einfluss vor allem auf Software Entwickler.
Denn Entwickler nutzen OSS, um es in eigene Projekte einzubauen, gegebenenfalls zu modifizieren
und ein Endprodukt mit der OSS Komponente weiterzuverbreiten. Das ist mit restriktiven Lizenzen
wie GPL meist nicht bedingungslos umsetzbar. Restriktive Lizenzen wie GPL wirken sich daher
negativ bis neutral auf den Erfolg von OSS aus. Wenn die Software allerdings an Endnutzer
gerichtet ist, wie zum Beispiel die Chat-App Telegram\footnote{https://telegram.org/}, spielt die
Lizenz eine weniger wichtige Rolle, da Weiterverbreitung und Modifizierung für diese Nutzergruppe
keine Rolle spielen \cite{subramaniamDeterminantsOpenSource2009}.

% Stewart et al. on Licenses
Stewart et al. widerspricht der zweiten Aussage von Subramaniam et al. laut ihnen
haben nicht-restriktive-Lizenzen nicht nur auf das Entwicklerinteresse, sondern auch auf das
Nutzerinteresse einen positiven Einfluss. % Stewart et al. H1A
Während restriktive Lizenzen einen nicht signifikanten Einfluss auf
Entwickleraktivität hätten. \cite{stewartImpactsLicenseChoice2006a}

% Widerspruch von Midha et. al.
Laut Midha und Palvia wirken sich freie Lizenzen positiv auf den Markterfolg aus, allerdings
nur zu Beginn eines Projektes. %H2a Vermutung: Beliebtheit ist Wichtiger als Lizenz
Restriktive Lizenzen wiederum wirken sich negativ auf den technischen Erfolg aus %H2b
\cite{midhaFactorsAffectingSuccess2012}.

%? Stichproben Größe (Side Note)
% Midha et al. 283 [Kapitel 4.1]
% Subramaniam et al. 8627 [Kapitel 4.2]
% Steward et al. 138 [Kapitel "Sample]

% Überleitung zur Hypothese (Ich Form, aber Bro schau dir mal wie häufig "WE" in Paper vorkommt)
Meine Hypothese ist, dass sich offene Lizenzen durchgängig positiv auf ein Projekt auswirken.
Offene Lizenzen werden tendenziell eher von Unternehmen verwendet als Projekte mit restriktiven
Lizenzen, das führt zum einen dazu, dass die Beliebtheit und Bekanntheit des Projektes steigt, als
auch die Wahrscheinlichkeit das die Unternehmen zum Open Source Projekt etwas beitragen oder
Sponsoren werden.

\begin{hypothesis}
    Offene Lizenzen haben positiven Einfluss auf den Markterfolg.
    \label{H:1}
\end{hypothesis}

\noindent
Steigt die Beliebtheit eines Projekts, so steigt auch das Interesse von Open Source Entwickler an
einem renommierten Projekt mitzuwirken.

\begin{hypothesis}
    Offene Lizenzen haben positiven Einfluss auf den technischen Erfolg.
\end{hypothesis}



\todo{A word about Licenses}
Was macht permissive aus?
- Kann uneingeschränkt benutzt werden
- Kann ez verwendet werden, einfach Lizenz Text beilegen
- Source Code einer MIT Lib kann in Propreritärer Software verwendet werden usw.
- Beispiel: MIT, BSD, Apache

Was macht GPL aus?
- Kann nicht uneingeschränkt genutzt werden => one GPL Lib => Whole Programm needs to be Open Source now under gpl
- GPLv2 vs GPLv3
- Was hat es dann eigentlich mit AGPL und LGPL aufsich? irgendwas mit Cloud und Server shit

MPL
- Die Mitte zwischen GPL und MIT
So under the terms of the MPL, it allows the integration of MPL-licensed code into proprietary codebases,
but only on condition those components remain accessible
% Quelle: https://en.wikipedia.org/wiki/Mozilla_Public_License
% Sieh auch: https://www.mozilla.org/en-US/MPL/2.0/FAQ/   =>   Q11
% https://www.youtube.com/watch?v=YlKtCDJquSw