\subsection{Lizenzen}

% TODO: Weitere Quelle einbauen!
\todoBox{
    \textbf{Weitere Quelle einbauen:}

    \cite{stewartImpactsLicenseChoice2006a} Seite 140 (16 in PDF) sagt aus, 
    dass nicht-restriktive Lizenzen sich positiv auf ein Projekt auswirken (Hypothese 1A), 
    indem mehr User angezogen werden.
    
    Gleichzeitig heißt es das es \textit{keinen} Zusammenhang gibt zwischen restriktive Lizenzen
    und anziehen von mehr Entwicklern für das Projekt.
}


% Lizenzen haben signifikaten Einfluss auf Erfolg [Subramaniam]
Laut Subramaniam et al. spielen Lizenzen eine \textbf{signifikante} Rolle für den Erfolg von 
Open Source Software.
Freie Lizenzen wie MIT oder BSD haben einen positiven Einfluss vor allem auf Software Entwickler.
Entwickler die OSS nutzen tun dies, um es in eigene Projekte einzubauen, gegebenenfalls zu modifizieren
und weiterzuverbreiten. Das ist mit restriktiven Lizenzen wie GPL meist nicht bedingungslos umsetzbar.
Restriktive Lizenzen wie GPL wirken sich daher negativ bis neutral auf den Erfolg von OSS aus.
Wenn die Software allerdings an Endnutzer
gerichtet ist, wie zum Beispiel die Chat-App Telegram, spielt die Lizenz eine weniger wichtige Rolle, 
da Weiterverbreitung und Modifizierung für diese Nutzergruppe keine Rolle spielen 
\cite{subramaniamDeterminantsOpenSource2009}.

% Stewart et al. on Licenses
Laut Stewart et al. haben nicht restriktive Lizenzen
einen positiven Einfluss auf das Nutzerinteresse % Stewart et al. H1A
während restriktive Lizenzen einen nicht signifikaten Einfluss auf Nutzerinteresse
als auch Entwickleraktivität haben.

%? Sample Size von Steward et al.?

% Widerspruch von Midha et. al.
Midha und Palvia widerspricht \cite{stewartImpactsLicenseChoice2006a} und \cite{subramaniamDeterminantsOpenSource2009} 
in dem Punkt, das die Lizenz ein signifikanter Faktor für den Erfolg eines OSS Projekts sei.
Laut Midha et al. spielt die Lizenz nur zu Beginn
des Projektes eine Rolle, da sobald ein Projekt \todoo{beliebt (Synonym finden)} ist, die Beliebtheit höher gewichtet wird als
die Lizenz.
Des Weiteren heißt es auch, dass restriktive Lizenzen sich im späteren Verlauf eines Projektes positiv
auf Entwickler auswirkt \cite{midhaFactorsAffectingSuccess2012}. % Kapitel 6.2
Damit widerspricht Midha erneut \cite{stewartImpactsLicenseChoice2006a}.

Die Stichprobengröße von Midha et. al. lag allerdings nur bei 283, % Kapitel 4.1 (midhaFactorsAffectingSuccess2012)
während die Stichprobengröße von Subramaniam et. al. bei 
8627 lag \cite{subramaniamDeterminantsOpenSource2009}. % Kapitel 4.2

\todoo{
    Sample-Size in \cite{stewartImpactsLicenseChoice2006a} 138 % Kapitel "Sample"
}

% ! Hidden Quote
% \begin{quote}
%     \begin{tcolorbox}[colback=black!5!white,colframe=white!75!black,title=Direkt Zitat aus \cite{midhaFactorsAffectingSuccess2012} Kapitel 6.2]
%         The insignificance may be because consumers
%         do not wish to be bothered by the license choice when information on other extrinsic attributes is readily
%         available. This, in a way, agrees with vox populi that most of the end users do not read the
%         licensing agreements.
%     \end{tcolorbox}
% \end{quote}

Wie in \ref{ssec:Eine Community Aufbauen} später genauer erläutert wird, ist eine Community ein 
essenzieller Bestandteil für ein Open Source Projekt.
Abhängig der Lizenz zieht man unterschiedliche Personengruppen an.
Offene Lizenzen wie MIT lädt vor allem X an...
Mit eingeschränkten Lizenzen wie GPL zieht man weniger Leute/Unternehmen etc. an und hindert 
somit das Wachstum der eigenen Community \citationNeeded{\cite{stewartImpactsLicenseChoice2006a} PDF S. 16}

Unternehmen nutzten die Software, 
eingie improven die die Software und ein Teil davon gibt zur OSS Community auch wieder zurück 
\cite{bangerthWhatMakesComputational2013} % Kapitel 3.5

\bigskip

% --------------------------------- Hypothese -------------------------------- %
\begin{hypothesis}
    Lizenzen haben einen signifikanten Einfluss sowohl auf den technischen
    als auch den Markterfolg. 
    Wobei offene Lizenzen sich positiv auswirken, während restriktive einen negativen
    Einfluss auf den Erfolg haben
\end{hypothesis}
% ---------------------------------------------------------------------------- %