\subsection{Eine Community Aufbauen} \label{ssec:Eine Community Aufbauen}

\todo{Markierte Quellen evtl. nochmal anschauen}

\todo{Stichpunkte ausformulieren}

\todo{Bogen zu Lizenzen ggf. schließen sieh .tex}

\noindent
Ein Open Source Projekt braucht eine Community von Benutzern und Mitwirkende. Ohne kann ein Projekt
nicht wachsen. Einige Nutzer werden selbst zu Mitwirkenden und tragen ihren Teil zur Verbesserung
des Projektes bei. Angefangen von Bug Reports bis hin zu Pull Requests. Dieses Verhalten der Community
muss von den Projektleitern ermutigt werden
\cite{bangerthWhatMakesComputational2013,GitHubBuildingWelcomingCommunities2022}.

\bigskip
\noindent
Was muss man dafür machen?

\begin{itemize}
    \item Make people feel welcome => How
    \item Document Everything
    \item Be Responsive (Time to Close a Ticket)
    \item Share ownership of your project (Encourage Users/Contributors to contribute)
    \item Good README
\end{itemize}

\bigskip
\noindent
Womöglich gibt es hier noch etwas praktische infos \\
\url{https://opensource.guide/code-of-conduct/} \\
\url{https://opensourcesurvey.org/2017/} \\
\url{https://www.sciencedirect.com/science/article/abs/pii/S0024630108000836} \\
\cite{midhaFactorsAffectingSuccess2012}

\bigskip

%TODO: Bezug auf Lizenz hier aufnehmen
Abhängig der Lizenz zieht man unterschiedliche Personengruppen an.
Offene Lizenzen wie MIT lädt vor allem X an...
Mit eingeschränkten Lizenzen wie GPL zieht man weniger Leute/Unternehmen etc. an und hindert
somit das Wachstum der eigenen Community \citationNeeded{\cite{stewartImpactsLicenseChoice2006a} PDF S. 16}
%
Unternehmen nutzten die Software,
eingie improven die die Software und ein Teil davon gibt zur OSS Community auch wieder zurück
\cite{bangerthWhatMakesComputational2013} % Kapitel 3.5



\begin{hypothesis}
    Bemühungen darin eine Community aufzubauen führen zu einem hohen Markt Erfolg.
\end{hypothesis}