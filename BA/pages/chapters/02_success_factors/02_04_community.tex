\subsection{Eine Community Aufbauen} \label{ssec:Eine Community Aufbauen}

\todo{Hier braucht es noch etwas Literatur Recherche.}

\todoo{There are two types of community, User Community and Developer Community...}

\todoo{Code of Conduct passt hier rein}

\noindent
Ein Open Source Projekt braucht eine Community.
Eine Community von Benutzern und eine Community von Contributor. Ohne eine Community kann ein Projekt
nicht wachsen. Contributer werden gebraucht um das Projekt kontinuierlich zu verbessern, Benutzer um
es natürlich zu nutzten (aka Goal of the " success ") aber auch um Bugs zu finden und zu reporten,
dies muss allerdings auch aktive encouraged werden. \cite{bangerthWhatMakesComputational2013} 
Sprich die Entwickler müssen sich um die
Community kümmern, bzw aktiv dafür sorgen, dass die Community wächst. 

\todoBox{
    Weitere Quellen: \cite{midhaFactorsAffectingSuccess2012}

    Mögliche weitere Quelle \link{How do Firms Make Use of Open Source Communities}{https://www.sciencedirect.com/science/article/abs/pii/S0024630108000836}
}

%! Copy Paste aus dem Lizenz Kapitel ursprünglich
%TODO: Bezug auf Lizenz hier aufnehmen
\textcolor{red}{
    Wie in \ref{ssec:Eine Community Aufbauen} später genauer erläutert wird, ist eine Community ein 
    essenzieller Bestandteil für ein Open Source Projekt.
    Abhängig der Lizenz zieht man unterschiedliche Personengruppen an.
    Offene Lizenzen wie MIT lädt vor allem X an...
    Mit eingeschränkten Lizenzen wie GPL zieht man weniger Leute/Unternehmen etc. an und hindert 
    somit das Wachstum der eigenen Community \citationNeeded{\cite{stewartImpactsLicenseChoice2006a} PDF S. 16}
    %
    Unternehmen nutzten die Software, 
    eingie improven die die Software und ein Teil davon gibt zur OSS Community auch wieder zurück 
    \cite{bangerthWhatMakesComputational2013} % Kapitel 3.5
    }


% \begin{hypothesis}
%     Bemühungen darin eine Community aufzubauen führen zu einem hohen Markt Erfolg.
% \end{hypothesis}