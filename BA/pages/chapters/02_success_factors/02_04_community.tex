\section{Die Community und die Projektentwicklung}\label{sec:community}


% ----------------------------------------- New Approach ----------------------------------------- %

% Was will ich in diesem Kapitel überhaupt aussagen?
Ein weiterer Erfolgsfaktor der OSS ist die Community, die sowohl aus Nutzern als auch aus OSS-Entwickler
besteht. Es liegt in der Verantwortung der Projektleiter bzw. Projekteigentümer die Community aufzubauen
und zu pflegen \cite{bangerthWhatMakesComputational2013,GitHubBuildingWelcomingCommunities2022}.


In einem Leitfaden von GitHub wird die Bedeutung der Community und wie diese aufgebaut wird näher erklärt.
Hierbei ist vor allem die Rede von einem Code of Conduct und Contributing Guide
\cite{GitHubBuildingWelcomingCommunities2022}. Das Ziel ist es eine offene und wachsende Community
aufzubauen, die dem technischen Erfolg dienen soll.



% Erklärung zu: Code of Conduct
Das \textit{Code of Conduct} ist ein Dokument, welches die Erwartungen an das Verhalten der Projektteilnehmer
festlegt. Das Übernehmen und Durchsetzten des Code of Conducts kann dazu beitragen, eine positive
und soziale Atmosphäre für alle zu schaffen \cite{GitHubYourCodeOfConduct2022}. % Section 1
Häufig findet sich hier im Hauptverzeichnis des Projektes die Datei \texttt{CODE\_OF\_CONDUCT.md}.
Als Vorlagen dient in der Regel das \textit{Contributor Covenant}\footnote{https://www.contributor-covenant.org/}.


% Erklärung zu: Contributing
Der \textit{Contributing Guide} ist eine Einführung, für neue Mitwirkende,
die sich an dem jeweiligen Projekt beteiligen wollen. Hier finden sich die Anleitungen und Vorlagen für
Bug Reports, Feature Requests, vorgehen bei Pull Requests, sowie Richtlinien bezüglich
Coding Styles und Testabdeckung \cite{GitHubStartingAProject2022}. % Unter #writing-your-contributing-guidelines
Einige Projekte beginnen ihre \texttt{CONTRIBUTING.md} mit einem Dank an den Leser und künftigen
Mitwirkenden, wie beispielsweise Chakra-UI\footnote{https://github.com/chakra-ui/chakra-ui},
mit den Worten
\textit{"Thanks for showing interest to contribute to Chakra UI, you rock!"}.
Somit wird die Hypothese aufgestellt, dass Projekte mit einem Code of Conduct bzw. Contributing
Guide eine höheren technischen Erfolg haben.

% Hypothese: Code of Conduct => technischer Erfolg
\begin{hypothesis}
    Das Vorhandensein eines \textit{Code of Conduct} führt zu einem höheren technischen Erfolg.
    \label{H:4}
\end{hypothesis}


% Hypothese: Contributing Guides => technischen Erfolg
\begin{hypothesis}
    Das Vorhandensein eines \textit{Contributing Guides} führt zu einem höherem technischen Erfolg.
    \label{H:5}
\end{hypothesis}


% \noindent
% Da der Entwicklungsprozess in einem Open Source Projekt offen einsehbar ist, kann dieser von potenziellen
% Nutzern auch bewertet werden. Metriken wie Aktualität des letzten Commits, Regelmäßigkeit der Commits,
% Anzahl der aktiven Maintainer oder die Antwortzeit der Entwickler auf Tickets
% könnten hierbei für einen Nutzer interessant sein.
% Die sechste Hypothese ist daher, dass Projekte mit hohem technischen Erfolg einen positiven Einfluss auf 
% die Wahl der OSS haben.


% \begin{hypothesis}
%     Technischer Erfolg führt zu Markterfolg.
%     \label{H:6}
% \end{hypothesis}