\section{Eine Community Aufbauen} \label{ssec:Eine Community Aufbauen}


% ----------------------------------------- New Approach ----------------------------------------- %

\todo{Was bringt CoC und Contributing Guide überhaupt. Mehr oder weniger evtl GitHub Guides hier zitieren}

% Was will ich in diesem Kapitel überhaupt aussagen?
Ein weiterer Erfolgsfaktor in OSS ist die Community, sowohl die Community von Nutzern, als auch die
Community von OSS-Entwickler. Es liegt in der Verantwortung der Projektleiter die Community aufzubauen
und zu pflegen. \cite{bangerthWhatMakesComputational2013,GitHubBuildingWelcomingCommunities2022}.
Auf GitHub finden sich Standardmäßig zwei Dateien, welche auf die Wahrnehmung dieser Verantwortung
hindeutet. Die \texttt{CODE\_OF\_CONDUCT.md} und \texttt{CONTRIBUTING.md}.

\bigskip

% Erklärung zu: Code of Conduct
Das \texttt{CODE\_OF\_CONDUCT.md} ist ein Dokument, welches die Erwartungen an das Verhalten der
Projektteilnehmer festlegt.
Das Übernehmen und Durchsetzten des Code of Conducts kann dazu beitragen, eine positive und soziale
Atmosphäre für alle zu schaffen \cite{GitHubYourCodeOfConduct2022}. % Section 1
Häufig werden hierfür Vorlagen von der \textit{Contributor Covenant} 
Website\footnote{https://www.contributor-covenant.org/} verwendet.

% Hypothese: Code of Conduct => Markterfolg
\begin{hypothesis}
    Das Vorhandensein eines \textit{Code of Conduct} führt zu einem höheren Markterfolg.
    \label{H:4}
\end{hypothesis}

% Erklärung zu: Contributing
Die \texttt{CONTRIBUTING.md} Datei ist eine kurze Einführung, für potenzielle neue Contributor,
wie man am jeweiligen Projekt mitwirken kann. Hier finden sich Anleitungen und Vorlagen für 
Bug Reports, Feature Requests, vorgehen bei Pull Requests, sowie Richtlinien bezüglich 
Coding Styles und Testabdeckung \cite{GitHubStartingAProject2022}. % Unter #writing-your-contributing-guidelines
Einige Projekte beginnen ihre \texttt{CONTRIBUTING.md} mit einem dank an den Leser und künftigen
Contributor, wie beispielsweise Chakra-UI\footnote{https://github.com/chakra-ui/chakra-ui},
mit den Worten 
\textit{"Thanks for showing interest to contribute to Chakra UI, you rock!"}.
Somit wird die Hypothese aufgestellt, dass Projekte mit einem Contributing Guide eine höhere
Chance für technischen Erfolg haben.

% Hypothese: Contributing Guides => technischen Erfolg
\begin{hypothesis}
    Das Vorhandensein eines \textit{Contributing Guides} führt zu einem höherem technischen Erfolg.
    \label{H:5}
\end{hypothesis}