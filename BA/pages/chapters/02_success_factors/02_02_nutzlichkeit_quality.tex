\subsection{Qualität}

\rawidea


% Sprich, gibt es eine gute Dokumentation, ist das Package etc. Customizable nach den Wünschen
% des Nutzers etc. Zu der Qualität gehört auch Software die "Bug-Frei" läuft, oder zumindest die Funktionalität
% die Bugs überwiegt \cite{bangerthWhatMakesComputational2013}. % Kapitel 2.1. Utility and Quality

\todoBox{
    Schwer zu definieren und zu Messen, hier braucht es noch ein bisschen Literatur Recherche. Was und wie haben andere
    gemessen? \\ Evtl. die Daten nur mit Umfrage erfassen.
}

Welche Kriterien müssen erfüllt werden damit ein Projekt als Qualitativ hochwertig betrachtet wird?


% \begin{quote}
%     \begin{tcolorbox}[colback=black!5!white,colframe=white!75!black,title=Direkt Zitat aus \cite{bangerthWhatMakesComputational2013} Kapitel 2.1]
%         [W]ithout a focus on fixing bugs as soon as they are
%         identified will never be of high quality.
%         Thus, quality needs to be an important aspect of
%         development from the start.
%     \end{tcolorbox}
% \end{quote}

% Die \textit{First Time Experience} spielt eine wichtige Rolle, ist ein Tool
% schwer aufzusetzten / installieren (bei npm trifft das nicht ganz zu?)
% beziehungsweise die Dokumentation nicht schlüssig genug, gibt es meist eine (Hand voll/Menge...) alternative Tools
% die ein User stattdessen einfach hernehmen kann, statt sich mit Tool \textit{X} herumzuschlagen
% \cite{bangerthWhatMakesComputational2013}. % Kapitel 2.1 Utility and Quality

Wenn Software von hoher Qualität ist, führt es zu einem höheren Markterfolg, da die Software 
weiter empfohlen als auch von großen IT-Unternehmen eingesetzt wird, was zur mehr Bekanntheit
führen kann.

% TODO: Muss etwas genauer und vorallem Messbarer formuliert werden!
\begin{hypothesis}
    Hohe Software Qualität führt zu zufriedenen Benutzern => mehr Markterfolg
\end{hypothesis}