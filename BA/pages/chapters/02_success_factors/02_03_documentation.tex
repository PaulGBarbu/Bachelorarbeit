\subsection{Gute Dokumentation}

\todo{Blauen Text ausformulieren} \\
\todo{Quelle für raussuchen sieh unten} \\
\todo{Weitere Quelle einbauen \cite{dagenaisDeveloperDocumentation}}

Dokumentationen spielen eine entscheidende Rolle für den Erfolg eines Projekts.
Ohne eine gute Dokumentation ist die Software schwer zugänglich für die Benutzer und damit teils
unbrauchbar.
Mailing Listen und StackOverflow können eine gute Ergänzung zur Dokumentation sein, allerdings kann
diese dadurch nicht ersetzt werden.\\ \citationNeeded{Könnte aus \cite{bangerthWhatMakesComputational2013} stammen}


Mit einem Crawler ist es schwer zu beurteilen, ob eine Dokumentation gut ist oder nicht oder ob
eine Dokumentation überhaupt existiert, da sich diese häufig auch auf der Homepage des Projekts befinden.
Man kann aber Dokumentation mit als Punkt in die Umfrage mit aufnehmen.

\begin{quote}
    GitHub's 2017 Open Source Survey showed incomplete or confusing documentation is the biggest
    problem for open source users. Good documentation invites people to interact with your project. 
    Eventually, someone will open an issue or pull request. Use these interactions as opportunities 
    to move them down the funnel.

    \tiny
    Source: \url{https://opensource.guide/building-community/#make-people-feel-welcome}
    \normalsize

\end{quote}

Weitere hilfreiche Quellen: \link{GitHub Open Source Survey}{https://opensourcesurvey.org/2017/},
\link{Blog Post}{https://opensource.guide/building-community/\#document-everything} über Dokumentieren

\bigskip

\todoo{
    \noindent
    \textbf{"Wie wichtig ist eine gute Dokumentation bei der Auswahl einer OSS für Sie?"} eignet
    sich als hervorragende Frage in der Umfrage.
    Datenerfassung aus GitHub muss ggf. händisch stattfinden und soll wie folgt kategorisiert werden:
    \begin{itemize}
        \item 0 = keine Dokumentation
        \item 1 = Basis Dokumentation in der GitHub README
        \item 2 = Ausführliche Dokumentation (Eigene Website / Code Beispiele etc.)
    \end{itemize}
}

% TODO: Hypothese schöner formulieren
\todo{Hypothese schöner formulieren}

\begin{hypothesis}
    Dokumentation ist wichtig vorallem wenn die Zielgruppe Entwickler sind.
    (Gute Doku => Hoher Technischer Erfolg)
\end{hypothesis}