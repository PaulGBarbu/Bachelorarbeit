\subsection{Gute Dokumentation}

% Einleitung
Dokumentationen spielen eine entscheidende Rolle für den Erfolg eines Projekts.
Ohne eine gute Dokumentation ist die Software für die Benutzer als auch Contributor schwer zugänglich. 
% Quelle 1: StackOverflow et al. not enough
Mailing Listen und StackOverflow können als eine gute Ergänzung zur Dokumentation dienen, allerdings kann
diese dadurch nicht ersetzt werden \cite{bangerthWhatMakesComputational2013}. % 2.2.3

% Quelle 2: Gute Dokus ziehen Nutzer an
Eine gute Dokumentation ist auch ein Mittel, um neue Nutzer zu gewinnen. In Dagenais et al. heißt es,
dass schon das Vorhandensein eines \textit{Getting Started} Tutorials, den Nutzer beim 
Entscheidungsprozess positiv beeinflussen kann. \cite{dagenaisDeveloperDocumentation}. % 4.2 Consequences.
Ist man als Nutzer die ersten Schritte mit einer neuen Programmiersprache, Framework oder Bibliothek
gegangen, steigt die Wahrscheinlichkeit, dieses Produkt auch zu nutzten.
% Beispiel: React
Beispielsweise hat die Web-Bibliothek ReactJS\footnote{https://reactjs.org/} auf der Homepage simple 
Beispiele, die man live editieren kann, ohne sich vorher etwas downloaden zu müssen, um einen
ersten Eindruck von React zu gewinnen.
Des Weiteren, gibt es ein sehr ausführliches Tutorial\footnote{https://reactjs.org/docs/getting-started.html}
welches über 5 Kapitel mit 21 Unterkapitel alle Grundbausteine von React abdeckt, um das gesamte 
Feature-Set in Kürze zu präsentieren und neue Entwickler von React zu überzeugen.

% Quelle 3: GitHub, Bad Doku => Misserfolg / Good Doku => More Contribution
Laut einer GitHub Umfrage im Jahr 2017 sind unvollständige oder verwirrende Dokumentationen das größte
Problem für Open Source Nutzer. Eine Gute Dokumentation hingegen lädt nicht nur neue Nutzer ein,
sondern kann auch Nutzer zu Contributor machen.
Sei es durch das Erstellen von Issues oder eines ersten Pull Requests.
Hierfür spielen vor allem Contributing Guides und ein Code of Conduct eine wichtige Rolle, hierzu
mehr im nächsten Kapitel \cite{GitHubOpenSourceSurvey2017}.

% Überleitung zur Hypothese
Dokumentationen spielen eine wichtige Rolle, um neue Nutzer als auch neue Mitwirkende für ein
Projekt zu gewinnen. 
Daher wird die Hypothese aufgestellt, dass Projekte mit guter Dokumentationen, erfolgreicher sind.
Sowohl im Bereich des Markt- als auch technischen Erfolgs.


\begin{hypothesis}
    Gute Dokumentationen ziehen mehr Nutzer an und führen so zu einem höheren Markt Erfolg.
\end{hypothesis}

\begin{hypothesis}
    Vorhanden sein von Contributing Guides und/oder Code of Conduct führt zu einem höheren technischen Erfolg.
\end{hypothesis}