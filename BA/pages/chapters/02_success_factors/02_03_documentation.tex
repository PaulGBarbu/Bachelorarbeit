\section{Dokumentation} \label{sec:Documentation}


% Einleitung
Dokumentationen spielen eine entscheidende Rolle für den Erfolg eines Projekts.
Ohne eine gute Dokumentation ist die Software für die Benutzer als auch Mitwirkenden schwer zugänglich.
% Quelle 1: StackOverflow et al. not enough
Mailing Listen und StackOverflow können als eine Ergänzung zur Dokumentation dienen, allerdings kann
diese dadurch nicht ersetzt werden \cite{bangerthWhatMakesComputational2013}. % 2.2.3

\bigskip
\noindent
% Was ist Gute Dokumentation?
%Laut Adam Scott gehören folgende Punkte zu einer Guten Dokumentation:

% \begin{multicols}{2}
%     \begin{itemize}
%         \setlength\itemsep{0em}
%         \item verständlich und einfach zu lesen
%         \item vollständig
%         \item hat wenn möglich Beispiele
%         \item ist aktuell
%         \item überschaubar
%         \item auffindbar
%     \end{itemize}
% \end{multicols}

\noindent
Laut Adam Scott muss eine gute Dokumentation vollständig und einfach zu lesen sein, damit Entwickler
jeglichen Erfahrungshintergrunds in der Lage sind die Software zu nutzten. Außerdem sollte 
die Dokumentation, wenn möglich Beispiele beinhalten, um Nutzern zu ermöglichen Frameworks und Bibliotheken 
schneller zu verstehen und einzubinden. Des Weiteren sollte die Dokumentation immer auf dem aktuellen 
Stand der Software sein, da diese sonst für Verwirrung sorgen könnte.
\cite{scottEightRulesGood2018}


% Quelle 2: Gute Dokus ziehen Nutzer an
Eine gute Dokumentation ist auch ein Mittel, um neue Nutzer zu gewinnen. Im Artikel von Dagenais et al. heißt es,
dass schon das Vorhandensein eines \textit{Getting Started} Tutorials, den Nutzer beim
Entscheidungsprozess positiv beeinflussen kann \cite{dagenaisDeveloperDocumentation}. % 4.2 Consequences.
Sind Nutzer die ersten Schritte mit einer neuen Programmiersprache, Framework oder Bibliothek
gegangen, steigt die Wahrscheinlichkeit, dieses Produkt auch zu nutzen.
% Beispiel: React
Beispielsweise hat die Web-Bibliothek ReactJS\footnote{https://reactjs.org/} auf der Homepage simple
Beispiele, die live editiert werden können, ohne sich vorher etwas herunterladen zu müssen, um einen
ersten Eindruck von React zu gewinnen.
Ferner gibt es ein sehr ausführliches Tutorial\footnote{https://reactjs.org/docs/getting-started.html}
welches über 5 Kapitel mit 21 Unterkapitel alle Grundbausteine von React abdeckt, um das gesamte
Feature-Set in Kürze zu präsentieren und neue Entwickler von React zu überzeugen \cite{ReactTutorial}.

% Quelle 3: GitHub, Bad Doku => Misserfolg / Good Doku => More Contribution
Laut einer GitHub Umfrage im Jahr 2017 sind unvollständige oder verwirrende Dokumentationen das größte
Problem für Open Source Nutzer. Eine gute Dokumentation hingegen lädt nicht nur neue Nutzer ein,
sondern kann auch Nutzer zu Mitwirkenden machen.
Sei es durch das Erstellen von Issues oder eines ersten Pull Requests.
Hierfür spielen vor allem Contributing Guides und ein Code of Conduct eine wichtige Rolle, hierzu
mehr im nächsten Kapitel \cite{GitHubOpenSourceSurvey2017}.

% Überleitung zur Hypothese
Dokumentationen spielen eine wichtige Rolle, um neue Nutzer als auch neue Mitwirkende für ein
Projekt zu gewinnen.
Daher wird die Hypothese aufgestellt, dass Projekte mit guter Dokumentation, einen höheren
Markterfolg haben.


\begin{hypothesis}
    Gute Dokumentationen ziehen mehr Nutzer an und führen so zu einem höheren Markterfolg.
    \label{H:3}
\end{hypothesis}