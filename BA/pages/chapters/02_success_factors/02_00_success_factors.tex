\chapter{Erfolgsfaktoren}


\todo{ggf. besseren Namen finden als \textit{Haupterfolgsfaktoren} und \textit{weitere Faktoren} }
\bigskip

\noindent
Es gibt viele Faktoren, die Einfluss auf den Erfolg eines Projektes haben. 
In diesem Kapitel werden die Faktoren in zwei Klassen aufgeteilt und näher erklärt.
Die \textit{Haupterfolgsfaktoren} werden in dieser Bachelorarbeit mittels Datenerhebung und
Umfrage analysiert. 
\textit{Weitere Faktoren} werden zusätzlich auf Basis von Literatur betrachtet und diskutiert.
% Hauptfaktoren Erkläuterung
Zu den Haupterfolgsfaktoren gehören Eigenschaften wie \textit{Lizenzen, Qualität, Dokumentation, Community} 
und \textit{Network Effekt} \todoo{(ggf. noch anpassen)}.
% Weitere Faktoren
Diese Faktoren alleine schaffen allerdings kein Gesamtbild des Erfolgs, deshalb wird im Kapitel 
\textit{\ref{sec:Weitere Faktoren}} auf zusätzlich Aspekte eingegangen.
Diese werden allerdings nicht mittels Datenerhebung oder Umfrage analysiert, da der Aufwand
diese empirisch zu erfassen sehr viel höher ist. 
Hierzu gehören Faktoren wie \textit{Das richtige Timing, Modularität und Komplexität, Responsibility
Assignment} \todoo{(ggf. noch anpassen)}


\section{Haupterfolgsfaktoren} \label{sec:Haupterfolgsfaktoren}

\todo{Einleitender Satz für Kapitel \ref{sec:Haupterfolgsfaktoren}}


% TODO: Community Standards als möglichen weiteren Faktor
% \todoBox{
%     Man könnte einen weiteren Punkt aufnehmen: \textit{Community Standards}

%     Sieh hierfür \link{ein Beispiel}{https://github.com/facebook/react/community}
%     die Werte können sowohl von der API erfasst werden, sieh Kapitel Crawler, als auch in der Umfrage.
%     Ich schätze den User ist es egal ob sowas wie ein \textit{Code of Conduct} oder ein \textit{Pull request template}
%     dabei ist.
%     % ABER meine Hypothese wäre:
%     % \textbf{
%     %     mehr Entwickler mitarbeiten => mehr Features => Höhere Qualität =>
%     %     besseres Produkt => mehr Erfolg
%     % }
%     Aber das kann man nur dann aufnehmen, wenn man von der API auch rausbekommt, was jeweils von den \textit{Community Standards}
%     eingehalten werden. Nur mit den Prozentzahlen kann man wohl leider nicht viel Anfangen.
% }



\subsection{Lizenzen}

% TODO: Weitere Quelle einbauen!
\todoBox{
    \textbf{Weitere Quelle einbauen:}

    \cite{stewartImpactsLicenseChoice2006a} Seite 140 (16 in PDF) sagt aus, 
    dass nicht-restriktive Lizenzen sich positiv auf ein Projekt auswirken (Hypothese 1A), 
    indem mehr User angezogen werden.
    
    Gleichzeitig heißt es das es \textit{keinen} Zusammenhang gibt zwischen restriktive Lizenzen
    und anziehen von mehr Entwicklern für das Projekt.
}


% Lizenzen haben signifikaten Einfluss auf Erfolg
Laut Subramaniam et. al. haben Lizenzen einen \textbf{signifikanten} Einfluss auf den Erfolg von 
Open Source Software vor allem dann, wenn die Zielgruppe Entwickler sind.
Freie Lizenzen wie MIT oder BSD haben einen positiven Einfluss auf Software Entwickler die OSS nutzten,
während restriktive Lizenzen wie GPL sich negativ auf den Erfolg von OSS auswirken.
Die Erklärung von Subramaniam ist, dass Entwickler die OSS nutzen es tun, um diese zu
modifizieren, in eigene Projekte einzubauen und weiterzuverbreiten. Das ist mit
restriktiven Lizenzen meist nicht bedingungslos umsetzbar. Wenn die Software allerdings an Endnutzer
gerichtet ist, wie zum Beispiel die Chat-App Telegram, spielt die Lizenz eine weniger wichtige Rolle, 
da Weiterverbreitung und Modifizierung
für den normalen Nutzer keine Rolle spielen \cite{subramaniamDeterminantsOpenSource2009}.

% Widerspruch von Midha et. al.
Midha und Palvia widerspricht allerdings dieser Aussage, laut ihnen spielt die Lizenz nur zu Beginn
des Projektes eine Rolle, da sobald ein Projekt beliebt ist, die Beliebtheit höher gewichtet wird als
die Lizenz, so behaupten sie.
Es heißt allerdings auch, dass restriktive Lizenzen sich im späteren verlauf eines Projektes positiv
auf Entwickler auswirkt \cite{midhaFactorsAffectingSuccess2012}. % Kapitel 6.2
Die Stichprobengröße von Midha et. al. lag allerdings nur bei 283, % Kapitel 4.1 (midhaFactorsAffectingSuccess2012)
während die Stichprobengröße von Subramaniam et. al. bei 
8627 lag \cite{subramaniamDeterminantsOpenSource2009}. % Kapitel 4.2

% ! Hidden Quote
% \begin{quote}
%     \begin{tcolorbox}[colback=black!5!white,colframe=white!75!black,title=Direkt Zitat aus \cite{midhaFactorsAffectingSuccess2012} Kapitel 6.2]
%         The insignificance may be because consumers
%         do not wish to be bothered by the license choice when information on other extrinsic attributes is readily
%         available. This, in a way, agrees with vox populi that most of the end users do not read the
%         licensing agreements.
%     \end{tcolorbox}
% \end{quote}

Wie in \ref{ssec:Eine Community Aufbauen} später genauer erläutert wird, ist eine Community ein 
essenzieller Bestandteil für ein Open Source Projekt.
Abhängig der Lizenz zieht man unterschiedliche Personengruppen an.
Offene Lizenzen wie MIT lädt vor allem X an...
Mit eingeschränkten Lizenzen wie GPL zieht man weniger Leute/Unternehmen etc. an und hindert 
somit das Wachstum der eigenen Community \citationNeeded{\cite{stewartImpactsLicenseChoice2006a} PDF S. 16}

Unternehmen nutzten die Software, 
eingie improven die die Software und ein Teil davon gibt zur OSS Community auch wieder zurück 
\cite{bangerthWhatMakesComputational2013} % Kapitel 3.5

\bigskip

% --------------------------------- Hypothese -------------------------------- %
\begin{hypothesis}
    Lizenzen haben einen signifikanten Einfluss sowohl auf den technischen
    als auch den Markterfolg. 
    Wobei offene Lizenzen sich positiv auswirken, während restriktive einen negativen
    Einfluss auf den Erfolg haben
\end{hypothesis}
% ---------------------------------------------------------------------------- %
\subsection{Qualität}

\rawidea

\todoo{\textbf{Was genau gehört zur Qualität?} Abarbeitung von Tickets/Bug Fixing, Gute UX und DX,
Gute Dokumentation, ...}


% Sprich, gibt es eine gute Dokumentation, ist das Package etc. Customizable nach den Wünschen
% des Nutzers etc. Zu der Qualität gehört auch Software die "Bug-Frei" läuft, oder zumindest die Funktionalität
% die Bugs überwiegt \cite{bangerthWhatMakesComputational2013}. % Kapitel 2.1. Utility and Quality

\begin{quote}
    \begin{tcolorbox}[colback=black!5!white,colframe=white!75!black,title=Direkt Zitat aus \cite{bangerthWhatMakesComputational2013} Kapitel 2.1]
        [W]ithout a focus on fixing bugs as soon as they are
        identified will never be of high quality.
        Thus, quality needs to be an important aspect of
        development from the start.
    \end{tcolorbox}
\end{quote}

Die \textit{First Time Experience} spielt eine wichtige Rolle, ist ein Tool
schwer aufzusetzten / installieren (bei npm trifft das nicht ganz zu?)
beziehungsweise die Dokumentation nicht schlüssig genug, gibt es meist eine (Hand voll/Menge...) alternative Tools
die ein User stattdessen einfach hernehmen kann, statt sich mit Tool \textit{X} herumzuschlagen
\cite{bangerthWhatMakesComputational2013}. % Kapitel 2.1 Utility and Quality

\begin{hypothesis}
    Die Qualität spielt eine sehr wichtige Rolle, vor allem dann, wenn es Alternativen gibt. 
    (Gute Qualität => Guter Technischer/Markt Erfolg) 
\end{hypothesis}
\subsection{Gute Dokumentation}

\rawidea
Dokumentationen spielen eine entscheidende Rolle beim Erfolg eines Projekts.
Ohne eine gute Dokumentation ist die Software schwerer zugänglich für die Benutzer und damit teils
unbrauchbar, ausgenommen Projekte mit intuitiven User Interfaces.
Mailing Listen und StackOverflow können eine gute Ergänzung zur Dokumentation sein, allerdings kann
diese dadurch nicht ersetzt werden.
Mit einem Crawler ist es schwer zu beurteilen, ob eine Dokumentation gut ist oder nicht oder ob
eine Dokumentation überhaupt existiert, da sich diese häufig auch auf der Homepage des Projekts befinden.
Man kann aber Dokumentation mit als Punkt in die Umfrage mit aufnehmen.
\citationNeeded{Könnte aus \cite{bangerthWhatMakesComputational2013} stammen}

\bigskip

\todoo{
    \noindent
    \textbf{"Wie wichtig ist eine gute Dokumentation bei der Auswahl einer OSS für Sie?"} eignet
    sich als hervorragende Frage in der Umfrage.\\
    Alternative könnte man diese Daten auch erfassen, allerdings nur von Hand.
    Da zum einen die Dokumentationen nicht immer in der README.md sind, sondern auf anderen Website
    und der Crawler nicht beurteilen kann, ob eine Dokumentation gut ist oder nicht. Daher könnte man
    quasi eine Liste zum Abhacken durchgehen
    Beispielsweise wie folgt:
    \begin{itemize}
        \item Hat das Projekt eine Dokumentation?
        \item Hat die Doku Anwendungsbeispiele?
        \item Ist ein Sandbox-Modus für dieses Projekt möglich? Wenn ja, gibt es einen?
    \end{itemize}
}

\begin{hypothesis}
    Dokumentation ist wichtig vorallem wenn die Zielgruppe Entwickler sind.
    (Gute Doku => Hoher Technischer Erfolg)
\end{hypothesis}
\subsection{Eine Community Aufbauen} \label{ssec:Eine Community Aufbauen}

\rawidea

\todoo{There are two types of community, User Community and Developer Community...}

\noindent
Ein Open Source Projekt braucht eine Community.
Eine Community von Benutzern und eine Community von Contributor. Ohne eine Community kann ein Projekt
nicht wachsen. Contributer werden gebraucht um das Projekt kontinuierlich zu verbessern, Benutzer um
es natürlich zu nutzten (aka Goal of the " success ") aber auch um Bugs zu finden und zu reporten,
dies muss allerdings auch aktive encouraged werden. \cite{bangerthWhatMakesComputational2013} 
Sprich die Entwickler müssen sich um die
Community kümmern, bzw aktiv dafür sorgen, dass die Community wächst. 

\todoBox{
    Weitere Quellen: \cite{midhaFactorsAffectingSuccess2012}

    Mögliche weitere Quelle \link{How do Firms Make Use of Open Source Communities}{https://www.sciencedirect.com/science/article/abs/pii/S0024630108000836}
}


% \begin{hypothesis}
%     Bemühungen darin eine Community aufzubauen führen zu einem hohen Markt Erfolg.
% \end{hypothesis}

\subsection{Sponsoren}

\subsection{Schneeball Effekt / Network Effekt}


\todo{Network Effekt recherchieren}

% Eine zweite Quelle bestätigt das Vox Populi. In \cite{subramaniamDeterminantsOpenSource2009} % Kapitel 5.1. 
% wird vom sogenannten \textit{Network Effekt} gesprochen. Dieser wirkt sich laut
% Subraminam et. al. positiv auf den Erfolg von OSS aus. \todoo{Network Effekt in 2. Worten erklären... den Aktiven als auch den Passiven }

% \begin{quote}
%     \begin{tcolorbox}[colback=black!5!white,colframe=white!75!black,title=Direkt Zitat aus \cite{subramaniamDeterminantsOpenSource2009} Kapitel 5.1.]
%         The results from our study show the important role played by
%         network effects of OSS
%     \end{tcolorbox}
% \end{quote}

% \begin{hypothesis}
%     Erfolgreiche Projekte werden noch erfolgreicher. => neue Projekte müssen raus stechen
%     sonst haben sie keine Chance gegen die bestehenden OSS Alternativen
% \end{hypothesis}

% \todo{Wird von Umfrage erfasst: Achten Sie auf Downloads/Sterne o.ä.?}

\section{Weitere Faktoren} \label{sec:Weitere Faktoren}

\todo{Einleitender Satz für Kapitel \ref{sec:Haupterfolgsfaktoren}}


\subsection{Der richtige Zeitpunkt}


\todoo{Zitat aus Kapitel 3.1 hernehmen und Paraphrasieren/einbauen, siehe auch Schluss von 3.1}

\begin{quote}
    \begin{tcolorbox}[colback=black!5!white,colframe=white!75!black,title=Direkt Zitat aus \cite{bangerthWhatMakesComputational2013} Kapitel 3.1]
        An interesting point made in Malcolm Gladwells book Outliers: The Story of Success [24] is that people
        are successful if their skills support products in a marketplace that is just maturing and where there is,
        consequently, still little competition. The same is certainly true for open source software projects as well:
        Projects that pick up a trend too late will have a difficult time thriving in a market that already supports other,
        large and mature projects
    \end{tcolorbox}
\end{quote}
\subsection{Modularität und Komplexität}

Bezüglich Modularität gibt es hier etwas mehr \cite{marganSuccessOpenSource2015} sieh Kapitel IV. A)
beziehungsweise die Quellen [42,43,49] in \cite{marganSuccessOpenSource2015}

\subsection{Responsibility Assigment}
\subsection{More ?}

