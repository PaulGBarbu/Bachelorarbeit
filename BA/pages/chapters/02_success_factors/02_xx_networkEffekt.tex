\subsection{Schneeball Effekt / Network Effekt \colorbox{yellow}{WIP}}


\bigskip
\noindent
% Quelle: Midha
Midha et al. fand eine starke Korrelation zwischen der vergangenen Beliebtheit eines Projekts,
und der aktuellen Beliebtheit. Der Grund hierfür sei, dass die Beliebtheit
als Entscheidungskriterium bei der Auswahl eines Projektes verwendet wird 
\cite{midhaFactorsAffectingSuccess2012}. % Kapitel 6.3

% Quelle: Subramaniam 
Subramaniam et al. spricht von vom sogenannten \textit{Network Effekt},
dieser wirkt sich laut \cite{subramaniamDeterminantsOpenSource2009} % Kapitel 5.1. 
positiv auf den Erfolg von OSS aus.
% TODO \todoo{Network Effekt in 2. Worten erklären... den Aktiven als auch den Passiven }

% \begin{quote}
%     \begin{tcolorbox}[colback=black!5!white,colframe=white!75!black,title=Direkt Zitat aus \cite{subramaniamDeterminantsOpenSource2009} Kapitel 5.1.]
%         The results from our study show the important role played by
%         network effects of OSS
%     \end{tcolorbox}
% \end{quote}

Das kann erfassen mittels Umfrage. Durch Fragen wie: "Wie wichtig sind Downloads und GitHub 
Sterne oder ähnliche Metriken bei der Auswahl eines Projektes, im Vergleich zu anderen Metriken."

% Oder: Sortieren der wichtigkeiten, zur Auswahl wird dann sowas gegeben wie:
% \begin{itemize}
%     \item Anzahl der Contributor
%     \item Regelmäßigkeit der Commits
%     \item Anzahl der offenen Issues (bzw Tickets o.ä.)
%     \item Beliebtheit des Projekts, anhand Downloads oder GitHub Sterne.
%     \item Lizenz
% \end{itemize}

\bigskip
\noindent
\textbf{Hypothese in etwa:} Erfolgreiche Projekte werden noch erfolgreicher. => neue Projekte müssen raus stechen
sonst haben sie keine Chance gegen die bestehenden OSS Alternativen


% \begin{hypothesis}
%     Erfolgreiche Projekte werden noch erfolgreicher. => neue Projekte müssen raus stechen
%     sonst haben sie keine Chance gegen die bestehenden OSS Alternativen
% \end{hypothesis}

% \todo{Wird von Umfrage erfasst: Achten Sie auf Downloads/Sterne o.ä.?}