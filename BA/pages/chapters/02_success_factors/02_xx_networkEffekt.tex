\subsection{Schneeball Effekt / Network Effekt}


\todo{Network Effekt recherchieren}

\cite{midhaFactorsAffectingSuccess2012} in Kapitel 6.3 bzw die Hypothese H3a.
\begin{quote}
    \begin{tcolorbox}[colback=black!5!white,colframe=white!75!black,title=Direkt Zitat aus \cite{midhaFactorsAffectingSuccess2012} Kapitel 6.3.]
        As hypothesized in H3a, OSS projects that have a larger previous
        user base are more popular. This was true at all stages.
    \end{tcolorbox}
\end{quote}

% Eine zweite Quelle bestätigt das Vox Populi. In \cite{subramaniamDeterminantsOpenSource2009} % Kapitel 5.1. 
% wird vom sogenannten \textit{Network Effekt} gesprochen. Dieser wirkt sich laut
% Subraminam et. al. positiv auf den Erfolg von OSS aus. \todoo{Network Effekt in 2. Worten erklären... den Aktiven als auch den Passiven }

% \begin{quote}
%     \begin{tcolorbox}[colback=black!5!white,colframe=white!75!black,title=Direkt Zitat aus \cite{subramaniamDeterminantsOpenSource2009} Kapitel 5.1.]
%         The results from our study show the important role played by
%         network effects of OSS
%     \end{tcolorbox}
% \end{quote}

% \begin{hypothesis}
%     Erfolgreiche Projekte werden noch erfolgreicher. => neue Projekte müssen raus stechen
%     sonst haben sie keine Chance gegen die bestehenden OSS Alternativen
% \end{hypothesis}

% \todo{Wird von Umfrage erfasst: Achten Sie auf Downloads/Sterne o.ä.?}