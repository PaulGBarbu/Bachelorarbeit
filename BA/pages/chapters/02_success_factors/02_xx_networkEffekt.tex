\section{Beliebtheit}\label{sec:beliebtheit_erfolgsfaktor}

Des Weiteren kann die Beliebtheit eines Projekts für die Nutzer eine wichtige Rolle spielen. 
% Midha zu Beliebtheit
Midha et al. fanden eine starke Korrelation zwischen der vergangenen Beliebtheit 
eines Projekts, und der aktuellen Beliebtheit. Der Grund hierfür sei, dass die Beliebtheit
als Entscheidungskriterium bei der Auswahl eines Projektes verwendet wird
\cite{midhaFactorsAffectingSuccess2012}. % Kapitel 6.3
%In dieser Arbeit wird die Beliebtheit anhand der Anzahl der GitHub Sterne und Downloads gemessen.

% Quelle: Subramaniam 
Subramaniam et al. spreche vom sogenannten \textit{Netzwerkeffekt},
dieser wirkt sich laut \cite{subramaniamDeterminantsOpenSource2009} % Kapitel 5.1. 
positiv auf den Erfolg von OSS aus.
% Network Effekt Erklären
Der Begriff Netzwerkeffekt kommt aus der Volkswirtschaftslehre und beschreibt ein Phänomen,
bei dem ein Produkt oder eine Dienstleistung einen zusätzlichen Wert erhält, wenn mehr Menschen
diesen nutzten.
Im Softwareumfeld wird der Netzwerkeffekt in, z.B. Anzahl an Tutorials
oder Beiträge auf StackOverflow zeigen.
%Ist ein Projekt erfolgreich werden Entwickler mehr Fragen stellen und Tutorials verfassen,
Das wiederum schafft einen stärkeren Anreiz das beliebtere Tool zu nutzten.
Im Vergleich von ReactJS und Svelte wie in der Tabelle \ref{tab:react_vs_svelte} dargestellt wird, 
zeigt sich, dass ReactJS etwas mehr als 3-mal so viele Sterne auf GitHub hat, aber über 100-mal so viele 
Beiträge auf StackOverflow. 
Daraus ergibt sich die Hypothese:


\begin{hypothesis}
    Beliebte Projekte werden von Nutzern bei der Wahl von OSS bevorzugt.
    \label{H:8} % H9 => H8
\end{hypothesis}


% Edit with: https://www.tablesgenerator.com/latex_tables
\begin{table}[h]
    \begin{tabular}{lcccc}
        \hline
                        & \multicolumn{1}{l}{\textbf{GitHub Stars}} & \multicolumn{1}{l}{\textbf{npm downloads}} & \multicolumn{1}{l}{\textbf{Tutorials}\footnote{Anzahl der Google Ergebnisse}} & \multicolumn{1}{l}{\textbf{StackOverflow Fragen}} \\ \hline
        \textbf{React}  & 186k                                      & 14.6 mio                                   & 438 mio                                                                       & 380k                                              \\
        \textbf{Svelte} & 57k                                       & 278k                                       & 2.3 mio                                                                       & 3k
    \end{tabular}%
    \caption{React vs Svelte}
    \label{tab:react_vs_svelte}
\end{table}