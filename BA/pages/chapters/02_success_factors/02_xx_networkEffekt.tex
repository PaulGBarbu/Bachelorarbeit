\section{Netzwerkeffekt}


% Einleitung: Network Effekt Erklären
Der Begriff \textit{Netzwerkeffekt} kommt aus der Volkswirtschaftslehre und beschreibt ein Phänomen,
bei dem ein Produkt oder eine Dienstleistung einen zusätzlichen Wert erhält, wenn mehr Menschen
diesen nutzten.
Im Software Umfeld würde sich der Netzwerkeffekt in, z.B. Anzahl an Tutorials
oder Fragen/Antworten auf StackOverflow zeigen.

Ist ein Projekt erfolgreich werden Entwickler mehr Fragen stellen und Tutorials verfassen,
was wiederum einen stärkeren Anreiz schafft das beliebtere Tool zu nutzten.
Vergleicht man beispielsweise ReactJS und Svelte wird man feststellen das React etwas mehr als 3-mal
so viele Sterne auf GitHub hat, aber über 100-mal so viele Fragen auf StackOverflow.

% Edit with: https://www.tablesgenerator.com/latex_tables
\begin{table}[h]
    \begin{tabular}{lcccc}
        \hline
                        & \multicolumn{1}{l}{\textbf{GitHub Stars}} & \multicolumn{1}{l}{\textbf{npm downloads}} & \multicolumn{1}{l}{\textbf{Tutorials}\footnote{Anzahl der Google Ergebnisse}} & \multicolumn{1}{l}{\textbf{StackOverflow Fragen}} \\ \hline
        \textbf{React}  & 186k                                      & 14.6 mio                                   & 438 mio                                                                       & 380k                                              \\
        \textbf{Svelte} & 57k                                       & 278k                                       & 2.3 mio                                                                       & 3k
    \end{tabular}%
\end{table}

\todo{Schöner Formulieren}


% Midha zu Beliebtheit
Midha et al. fand eine starke Korrelation zwischen der vergangenen Beliebtheit eines Projekts,
und der aktuellen Beliebtheit. Der Grund hierfür sei, dass die Beliebtheit
als Entscheidungskriterium bei der Auswahl eines Projektes verwendet wird
\cite{midhaFactorsAffectingSuccess2012}. % Kapitel 6.3

% Quelle: Subramaniam 
Subramaniam et al. spricht von vom sogenannten \textit{Network Effekt},
dieser wirkt sich laut \cite{subramaniamDeterminantsOpenSource2009} % Kapitel 5.1. 
positiv auf den Erfolg von OSS aus.

% Überleitung zur Hypothese
Projekte die bereits beliebt sind haben einen dadurch einen Vorteil noch mehr neue Nutzer anzuziehen,
also welche die neu auf dem Markt sind.

\todo{Hyoothese ausformulieren}
\begin{hypothesis}
    Erfolgreiche Projekte werden noch erfolgreicher. => neue Projekte müssen raus stechen
    sonst haben sie keine Chance gegen die bestehenden OSS Alternativen
\end{hypothesis}