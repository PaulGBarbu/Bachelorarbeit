\subsection{Sponsoren}


%? Beispiel ?

% \begin{quotation}
%     Most of the core team members do this open source work in their free time. If you use recharts
%     for a important work, and you'd like us to invest more time on it, please donate. Thanks!
%     Quelle: https://github.com/recharts/recharts\#sponsoring
% \end{quotation}

% Einleitung
Ein weiterer Faktor, der Nutzer davon überzeugen kann ein gewisses Framework oder Bibliothek 
gegenüber einer anderen zu nutze, kann das Vorhandensein von Sponsoren sein. 
Den Sponsoren wird meist ein Abschnitt in der README.md oder auf der Website des Projektes
gewidmet, hier bedanken sich die Maintainer bei den jeweiligen Sponsoren. Häufig werden Logos
der Firmen oder Profilbilder der Einzelpersonen aufgelistet. Somit ist es schnell ersichtlich
ob ein Projekt gesponsert ist oder nicht. 

% Gesponsertes Projekt => Sieht fancier aus => Markterfolg
Wenn Nutzer sehen, dass ein gewisses Framework oder eine Bibliothek von Unternehmen oder 
Einzelpersonen gesponsert werden, kann die Wahrnehmung der Qualität 
des Produktes verstärkt wird, was sich im Endeffekt dann im Markterfolg widerspiegelt. % customer perceived value

\begin{hypothesis}
    Sponsoren haben einen positiven Einfluss bei der Auswahl von Open Source Software auf die Nutzern. Gesponserte Projekte haben
    eine höhere Wahrscheinlichkeit auf Markterfolg.
\end{hypothesis}

Sponsoren garantieren auch, dass ein Projekt auch noch morgen existiert, weiter aktiv gewartet wird
und neue Features entwickelt werden können. Somit wirken sich Sponsoren auch positiv auf den 
technischen Erfolg aus.

\begin{hypothesis}
    Gesponserte Projekte haben einen höheren technischen Erfolg
\end{hypothesis}

% Disclaimer
\subsubsection*{Anmerkung}
Man muss allerdings anmerken, dass einige Projekte von Unternehmen entwickelt werden. Diese finanzieren 
ihre Projekte meist selbst und sind somit nicht abhängig von Sponsoren. 
React beispielsweise kommt aus dem Hause Facebook, während VueJS unabhängig ist und auf Sponsoren angewiesen ist.
In diesem Kontext zählen Projekte von Unternehmen auch als gesponsert.