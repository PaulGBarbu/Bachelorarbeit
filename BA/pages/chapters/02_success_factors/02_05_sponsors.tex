\section{Finanzierung und Sponsoren von Open Source}\label{sec:sponsors}

% Einleitung
Ein weiterer Faktor, der Nutzer davon überzeugen kann ein gewisses Framework oder eine Bibliothek 
gegenüber einer anderen zu nutzen, ist die finanzielle Sicherheit eines Projekts. Es wird vermutet,
dass Projekte mit Sponsoren beliebter bei Nutzern sind als Projekte ohne.
Die Information, ob ein Projekt Sponsoren hat, ist für einen neuen potenziellen Nutzer hierbei in der
Regel sehr leicht ersichtlich. 
VueJS beispielsweise macht sowohl auf der Website als auch in der README.md auf die Sponsoren aufmerksam
und bedankt sich für ihre Unterstützung, wie in Abbildung \ref{abb:VueJS_Sponsors} zu erkennen ist.

Daraus folgt die Hypothese:

\begin{hypothesis}
    Sponsoren haben einen positiven Einfluss auf die OSS-Auswahl, demnach haben gesponserte Projekte
    einen höheren Markterfolg.
    \label{H:6} % H7 => H6
\end{hypothesis}

\noindent
Gleichzeitig kann die finanzielle Unterstützung der Sponsoren die Produktivität fördern.
Die Projekte werden schneller weiterentwickelt und sorgfältiger gewartet. Daraus folgt die 
Hypothese:

\begin{hypothesis}
    Gesponserte Projekte haben einen höheren technischen Erfolg.
    \label{H:7} % H8 => H7
\end{hypothesis}

\begin{figure}[h]
    \centering
    \includegraphics[scale=0.6]{figures/02/VueSponsoren.JPG}
    \caption{VueJS Top Sponsoren}
    \label{abb:VueJS_Sponsors}
\end{figure}
