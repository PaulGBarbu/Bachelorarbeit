\chapter{Diskussion}


Das Ziel dieser Bachelorarbeit war es, zu untersuchen welche Faktoren Einfluss auf den Erfolg
eines Open Source Projektes haben. Um die Hypothesen dieser Arbeit zu prüfen wurden 108 Projekte
analysiert sowie 308 Studenten und adesso Mitarbeiter befragt.
Anhand der Daten wurden sieben Hypothesen belegt und eine wurde verworfen.
In den folgenden Unterkapiteln werden diese Ergebnisse diskutiert.



% ---------------------------------------- Hypothese 1 & 2 --------------------------------------- %
\section{Der Einfluss von permissiven Lizenzen auf den Erfolg}


% --------------------- Hypothese 1 --------------------- %
%~ 1. Um welche Hypothese geht es
In Kapitel \ref{sec:Lizenzen} wurde die Hypothese aufgestellt, dass permissive Lizenzen einen positiven
Effekt auf den Markterfolg haben.
%~ 2. Mit welchen Mittel wurde diese Hypothese geprüft
Um diese Hypothesen zu prüfen wurde eine Datenerhebung mit 108 Projekten durchgeführt. Als Metrik des
Markterfolges wurden GitHub Sterne verwendet.
%~ 3. Welche Daten belegen/widerlegen die Hypothese + Interpretation
Wie die Ergebnisse aus Kapitel \ref{sec:Ergebnisse_Datenerhebung_Lizenzen} zeigen, haben Projekte mit
permissiven Lizenzen 27,4\% mehr Sterne zusätzlich sind die Top 32\% Projekte permissiv lizenziert.
Und das trotz expliziter Suche nach restriktiven Projekten, wie in Kapitel \ref{sec:manuelle_datenerfassung}
beschrieben. Bei der manuellen Datenerfassung war es zusätzlich notwendig explizit nach restriktiven
Projekten zu suchen, um den Anteil dieser Projekte zu erhöhen. Das zeigt zum einen, dass
permissive Lizenzen allgemein bevorzugt werden und permissiv lizenzierte Projekte auch allgemein
erfolgreicher sind. Somit bestätigt sich \hyperref[H:1]{Hypothese H1}.
%~ Beschränkungen
Ursprünglich sollten auch die NPM Downloads als Beliebtheitsmerkmalen verwendet und verglichen
werden, allerdings sind nur 4 der 10 restriktiven Projekte auf NPM.


% --------------------- Hypothese 2 --------------------- %
\bigskip
\noindent
%~ 1. Um welche Hypothese geht es
Des Weiteren wurde die Hypothese aufgestellt, das permissive Lizenzen einen positiven Einfluss auf den
technischen Erfolg haben.
%~ 2. Mit welchen Mittel wurde diese Hypothese geprüft
Zum Prüfen der Hypothese wurden die Anzahl der Commits in den letzten 12 Monaten und Anzahl der Mitwirkenden
verglichen.
%~ 3. Welche Daten belegen/widerlegen die Hypothese + Interpretation
Projekte mit permissiven Lizenzen haben im Median sowohl mehr Commits (252\%) als auch mehr Mitwirkende
(144\%). Somit bestätigt sich die \hyperref[H:2]{Hypothese H2}, permissive Projekte sind technisch
erfolgreicher.


\bigskip
\noindent
% Limitations 2.0
Dieser Arbeit hat sich nur mit den Projekten beschäftigt, die hauptsächlich in JavaScript/Typescript
programmiert sind. Das JS/TS Umfeld ist hierbei dominiert von permissiven Lizenzen, dies hat zur
Folge gehabt, das von den 108 Projekten nur 10 Projekte restriktiv lizenziert sind. Diese
Ungleichverteilung erschwert generalisierbare Aussagen bezüglich der Lizenzen.

% TODO: Weiteres wäre: 
% - Commits zählen nur auf Main und GH-Page branch. 
% - Die genannten Punkte decken representieren natürlich nicht in Fülle Markt und technischen Erfolg sondern
% sind nur versuche diese anzunähren

\bigskip
\noindent
%~ Bezug zur Literatur, bestätigt/widerlegt Erkenntnisse aus der Literatur (Außer bei Explorativen)
Die Ergebnisse bestätigen die Erkenntnisse von Subramaniam et al. in dem Punkt, dass permissive Lizenzen
einen positiven Effekt auf den Erfolg von Projekten haben \cite{subramaniamDeterminantsOpenSource2009}.
Über den Effekt von restriktiven Lizenzen lassen sich hier keine äquivalenten Aussagen treffen, aufgrund
kleiner Datenmengen mit restriktiven Lizenzen.

% Subramaniam et al. 
%       - Freie Lizenzen haben positiven Einfluss 
%       - Restriktive Lizenzen kann ich aufgrund der Beschränkung nicht bestätigen.

% Steward et al.
%       - idk...

% Midha et al.
%       - Freizügige Lizenzen nur zu beginn bedeutend, würde ich nicht zustimmen


% Limitations
% \todoo{
%       In 2015 hat GitHub in einem Blogeintrag die Aufteilung der Lizenznutzung auf ihrer Platform veröffentlicht
%       \cite{OpenSourceLicense2015}. Wie man in Tabelle \ref{tab:GitHub_Blog_Lizenznutzung2015} sehen kann,
%       haben 64\% aller Projekte permissive Lizenzen und 24\% restriktive. 15\% der Projekte haben keine
%       Standardlizenz und werden von GitHub daher als \textit{other} kategorisiert.
% }
%? What about this text? 
% Allerdings muss man einräumen,
% dass für die Beliebtheit nur GitHub Sterne verwendet wurden. Der Linux Kernel ist GPLv2 lizenziert und
% gehört zu den bekanntesten und erfolgreichsten Open Source Projekten.
% Ein Grund weshalb freizügige Lizenzen beliebter sein könnten ist, dass es sich bei restriktiven
% Projekten meist, um Applikationen handelt. Telegram, Signal und ähnliche Open Source Chat Apps
% werden von Entwicklern sowie nicht Entwicklern verwendet, die eventuell gar keine GitHub Accounts haben.
% GitHub Sterne erfassen somit nicht das gesamte Bild \textit{Markterfolg}.





% ------------------------------------------ Hypothese 3 ----------------------------------------- %
%\newpage %! new page
\section{Der Einfluss einer guten Dokumentation auf den Markterfolg}


%~ 1. Um welche Hypothese geht es
Im Kapitel \ref{sec:Documentation} wurde die Hypothese aufgestellt, dass gute Dokumentationen mehr
Nutzer anziehen und somit zu einem höheren Markterfolg führen würde.
%~ 2. Mit welchen Mittel wurde diese Hypothese geprüft
Um diese Hypothese zu prüfen wurde eine Umfrage durchgeführt. Die Umfrage hatte 308 Teilnehmer.
%~ 3. Welche Daten belegen/widerlegen die Hypothese + Interpretation
Die Teilnehmenden sollten auf einer Likert Skala Aussagen zustimmen oder ablehnen, hier hatte das
Thema Dokumentation die höchste Zustimmung verglichen mit den anderen Themengebieten der Umfrage,
dies zeigt ein allgemein hohes Interesse an der Dokumentation. Diese Aussage wird von der Tatsache
untermauert, das wie in Kapitel \ref{sec:umfrage_last_question} gezeigt wird, die Dokumentation
das wichtigste Kriterium, bei der Auswahl von OSS ist. Des Weiteren gaben 81\% der Befragten an,
aufgrund schlechter Dokumentation schon mal zu einem alternativen Projekt gewechselt zu haben.
% Hypothese ist Proven.
Das zeigt die Wichtigkeit einer guten Dokumentation und bestätigt somit die
\hyperref[H:3]{Hypothese H3}.


%~ Beschränkungen & more
Die Freitextfelder haben gezeigt das die Kriterien der geschlossenen Fragen, die es zu bewerten 
galt nicht vollständig ware. Vor allem Aktualität und Vollständigkeit sind weitere wichtige 
Eigenschaften der Dokumentationen, die die Befragten angaben.
Für weitere Forschung wäre daher sinnvoll alle als wichtig bewerteten Kriterien, zu untersuchen.
Ein weiterer Aspekt der in Bezug auf die Dokumentation von den Befragten häufig erwähnt wurde,
war die UI/UX. Auch diese Kriterien könnten für weitere Untersuchungen relevant sein.

%~ Bezug zur Literatur, bestätigt/widerlegt Erkenntnisse aus der Literatur (Außer bei Explorativen)
In einer Umfrage der Open Source Survey aus 2017 stellte sich heraus, dass unvollständige bzw.
verirrende Dokumentation als größtes Problem in Open Source gelten \cite{GitHubOpenSourceSurvey2017}.
Diese Arbeit bestätigt diese Ergebnisse, und erweitert sie. Neben der Komplexität und Unvollständigkeit
sind die wichtigste Punkte einer guten Dokumentation, Code Beispiele und Aktualität.




% Die Ergebnisse der Umfrage bezüglich Code Beispiele decken mit den Aussagen aus Dagenais et al. 
% \cite{dagenaisDeveloperDocumentation}.




% ---------------------------------------- Hypothese 4 & 5 --------------------------------------- %
\section{Der Einfluss vom Code of Conduct und Contributing Guide auf den technischen Erfolg}

%~ 1. Um welche Hypothese geht es
Im Kapitel \ref{sec:community} wurden Hypothesen H4 und H5 aufgestellt, basierend auf den Open Source
Guide von GitHub \cite{GitHubBuildingWelcomingCommunities2022}.
Die \hyperref[H:4]{Hypothese H4} beschäftigt sich mit dem Effekt des \textit{Code of Conducts} auf
den technischen Erfolg. \hyperref[H:5]{Hypothese H5} hingegen beschäftigt sich mit dem Effekt des
\textit{Contributing Guides} auf den technischen Erfolg.
%~ 2. Mit welchen Mittel wurde diese Hypothese geprüft
Zum Prüfen beider Hypothesen wurden die Daten aus der GitHub Datenerhebung verwendet.
%~ 3. Welche Daten belegen/widerlegen die Hypothese + Interpretation
Ein Code of Conduct haben 53\% der 108 analysierten Projekte, einen Contributing Guide 86\%.
Gemessen wurde der technische Erfolg anhand der Commits und Anzahl der Mitwirkenden. Projekte mit
einem Code of Conduct haben deutlich mehr Commits (717\%) und Mitwirkende (201\%).
Projekte mit einem Contributing Guide haben 463\% mehr Commits und 412\% mehr Mitwirkende als Projekte
ohne.
% Hypothese belegt.
Damit sind beide Hypothesen belegt.

%~ Begründung für die Daten
Eine mögliche Erklärung für die Größe der Effekte, kann damit begründet werden, dass es sich für
kleine Projekte nicht lohnt, eine Code of Conduct einzuführen, da die Größe der Teams zu klein ist.
Im Fall des Contributing Guides, könnte die ungleiche Verteilung in den Daten die Ursache sein.

%~ Bezug zur Literatur, bestätigt/widerlegt Erkenntnisse aus der Literatur (Außer bei Explorativen)
Es zeigt sich allerdings deutlich, dass die Empfehlungen von GitHub auch durchgesetzt werden, 
vor allem in großen Projekten.



% ------------------------------------------ Hypothese 6 ----------------------------------------- %
% \section{Technischer Erfolg führt zu Markterfolg}

% \red{Ich glaub diese Hypothese schmeiß ich wieder aus da ich die nicht ordentlich belegen/wiederlegen kann}

% %~ 1. Um welche Hypothese geht es
% Im Kapitel \ref{sec:community} wurde die Hypothese aufgestellt, dass technischer Erfolg positiven
% Einfluss auf den Markterfolg hat.
% %~ 2. Mit welchen Mittel wurde diese Hypothese geprüft
% Die Hypothese wurde mittels Umfrage geprüft.
% %~ 3. Welche Daten belegen/widerlegen die Hypothese + Interpretation
% %~ Begründung für die Daten
% %~ Bezug zur Literatur, bestätigt/widerlegt Erkenntnisse aus der Literatur (Außer bei Explorativen)




% ---------------------------------------- Hypothese 6 & 7 --------------------------------------- %


\section{Der Einfluss der Sponsoren auf den Erfolg}

%~ 1. Um welche Hypothese geht es
Im Kapitel \ref{sec:sponsors} wurde die Hypothese aufgestellt, dass Sponsoren einen positiven Einfluss
bei der Wahl von OSS hätten.
%~ 2. Mit welchen Mittel wurde diese Hypothese geprüft
Die Hypothese wurde mittels Umfrage getestet.
%~ 3. Welche Daten belegen/widerlegen die Hypothese + Interpretation
Wie in Kapitel \ref{sec:Beliebtheit} bereits gezeigt wurde, achten die Nutzer sehr wenig auf das
Vorhandensein von Sponsoren.
In Kapitel \ref{sec:umfrage_sponsoren} ist ein möglicher Grund dafür erkennbar.
Gesponserte Projekte werden weder bevorzugt noch als qualitativ
hochwertiger betrachtet, das heißt Nutzer haben wenig Anlass um auf Sponsoren zu achten.
\hyperref[H:6]{Hypothese 6} wurde somit verworfen.

\bigskip
\noindent
%~ 1. Um welche Hypothese geht es
Bezüglich Sponsoren wurde zusätzlich noch die Hypothese aufgestellt, dass Sponsoren sich positiv auf
den technischen Erfolg auswirken würden.
%~ 2. Mit welchen Mittel wurde diese Hypothese geprüft
Diese Hypothese wurde mittels der GitHub Datenerhebung geprüft.
%~ 3. Welche Daten belegen/widerlegen die Hypothese + Interpretation
Wie in Kapitel \ref{sec:sponsoren_datenerhebung} dargelegt, haben Projekte mit Sponsoren 72\% mehr
Commits und 86\% mehr Mitwirkende als Projekte ohne Sponsoren.
Entsprechend wird die \hyperref[H:7]{Hypothese 7} belegt.


%~ Diskussion
Zwar hat die Umfrage gezeigt, dass die Nutzer nicht auf die Sponsoren achten, allerdings achten sie
auf die Aspekte des technischen Erfolgs, welcher wiederum durch die Anwesenheit der Sponsoren 
unterstützt wird. 
Die Beziehung zwischen all diesen Faktoren sollte in künftigen Studien untersucht werden, die genaue 
Effektstärke der Sponsoren im Projekt aufzuklären.



% ------------------------------------------ Hypothese 8 ----------------------------------------- %
\section{Der Einfluss von Beliebtheit auf den Markterfolg}

%~ 1. Um welche Hypothese geht es
In Kapitel \ref{sec:beliebtheit_erfolgsfaktor} wurde die Hypothese aufgestellt, dass Beliebtheit zu
mehr Markterfolg führt.
%~ 2. Mit welchen Mittel wurde diese Hypothese geprüft
Die Hypothese wurde als Teil der Umfrage getestet.
%~ 3. Welche Daten belegen/widerlegen die Hypothese + Interpretation
Die Beliebtheit ist laut Umfrage das zweit wichtigste Kriterium bei der Wahl von OSS.
Insgesamt gaben 46\% der Befragten an, aufgrund der geringer Popularität eines Projekts es nicht
genutzt zu haben.
Es wurde erwartet, das die Nutzer die Beliebtheit höher gewichten würden.
Dennoch führt die geringe Popularität bei fast der Hälfte der Befragten dazu ein Projekt nicht 
zu verwenden. Angesichts diesen hohen Anteils gilt die \hyperref[H:8]{Hypothese H8} als belegt.

%~ Diskussion
Dass die Nutzer die Beliebtheit als nur zweitwichtigstes Kriterium eingestuft haben kann vor allem
durch höhere Gewichtung der Dokumentation in einem Projekt erklärt werden.
Weitere Forschung könnte die Unterkategorien der Beliebtheit umfangreicher analysieren, um genauer
feststellen zu können welche Aspekte der Beliebtheit für die Nutzer größere Rollen spielen.



\section{Weitere Beschränkungen der Arbeit}
Zu den wesentlichen Limitationen der Untersuchung zählt die mangelnde Überprüfung der Beziehung der
einzelnen Erfolgskriterien zueinander. Es wurde nicht aufgeklärt ob es mögliche Überlappungen zwischen
den jeweiligen Variablen gibt. Des Weiteren wurden nur Projekte in zwei Programmiersprachen (JS/TS) 
zur Analyse verwendet. Aus diesem Grund wäre es für die weitere Forschung betrachtenswert auch 
Projekte in anderen Programmiersprachen zu untersuchen und mit Erkenntnissen dieser Arbeit zu vergleichen.

% \newpage
% --------------------------------------------------------------------



% %! Allgemeine Limitation
% Gleiches war bei den Contributor nicht umsetzbar, da die GitHub API keinen
% Endpoint hierfür hat, somit werden Contributor über den gesamten Zeitraum des Projektes verglichen. (Commits nur der letzten 12 Monate)


% \begin{itemize}
%       \item Umfrage hatte nur den Umfang von 308 Teilnehmern und kann deshalbt nicht generalisiert werden...
%       \item Die Daten aus dem \textit{Crawler} sind nur ein Snapshot der aktuellen Lage, sprich man kann
%             nur sagen. Ja es gibt eine Korrelation zwischen z.B. Sponsoren und User Interest, aber nicht was was hervorruft.
%             Hat das Projekt Sponsoren weil es viele Nutzer hat \textit{oder} hat das Projekt viele Nutzer weil es Sponsoren hat?
%             Dafür bräuchte man eine Datenerhebung over time.
%             Diese ist allerdings im kleinen Zeitfenster einer Bachelorarbeit nicht machbar.
%             \cite{midhaFactorsAffectingSuccess2012} hat beispielsweise in einem Zeitfenster von 3 Jahren die
%             Datenerhebung gemacht. % Abstract
% \end{itemize}


% \subsection{Disclaimer}

% https://www.npmjs.com/package/npm-downloads-increaser

% npm downloadzahlen sind manipulierbar, ich geh zwar davon aus, dass es allgemein nicht häufig gemacht wird
% und da ja sowieso vielmehr in die rechnung rein geht als nur npm downloads. Werden diese daten trotzdem mit erfasst.
% \link{How to exploit NPM}{https://dev.to/andyrichardsonn/how-i-exploited-npm-downloads-and-why-you-shouldn-t-trust-them-4bme}.

% In einem
% \link{Blogpost}{https://blog.npmjs.org/post/92574016600/numeric-precision-matters-how-npm-download-counts-work.html}
% erklärt NPM wie es zu den Download zahlen zustande kommt.