\chapter{Diskussion}


\todoo{Ziel dieser Arbeit war es....}


% ---------------------------------------- Hypothese 1 & 2 --------------------------------------- %
\section{Einfluss von permissiven Lizenzen auf den Erfolg}


% --------------------- Hypothese 1 --------------------- %
%~ 1. Um welche Hypothese geht es
In Kapitel \ref{sec:Lizenzen} wurde die Hypothese aufgestellt, dass permissive Lizenzen einen positiven
Effekt auf den Markterfolg haben.
%~ 2. Mit welchen Mittel wurde diese Hypothese geprüft
Um diese Hypothesen zu Prüfen wurde eine Datenerhebung mit 108 Projekten durchgeführt. Als Metrik des
Markterfolges wurden GitHub Sterne verwendet.
%~ 3. Welche Daten belegen/widerlegen die Hypothese + Interpretation
Wie die Ergebnisse aus Kapitel \ref{sec:Ergebnisse_Datenerhebung_Lizenzen} zeigen, haben Projekte mit
permissiven Lizenzen 27,4\% mehr Sterne zusätzlich sind die Top 32\% Projekte permissiv lizenziert.  
Und das trotz expliziter suche nach restriktiven Projekten, wie in Kapitel \ref{sec:manuelle_datenerfassung}
beschrieben. Bei der Manuellen Datenerfassung war es zusätzlich notwendig explizit nach restriktiven 
Projekten zu suchen, um den Anteil von restriktiver Projekte zu erhöhen. Das zeigt zum einen, dass 
permissive Lizenzen allgemein bevorzugt werden und permissiv lizenzierte Projekte auch allgemein 
erfolgreicher sind. Somit bestätigt sich \hyperref[H:1]{Hypothese H1}.
%~ Beschränkungen
Ursprünglich sollten auch die NPM Downloads als Beliebtheitsmerkmalen verwendet und vergleichen
werden. Allerdings sind nur 4 der 10 restriktiven Projekte auf NPM.


% --------------------- Hypothese 2 --------------------- %
\bigskip
\noindent
%~ 1. Um welche Hypothese geht es
Des Weiteren wurde die Hypothese aufgestellt, das permissive Lizenzen einen positiven Einfluss auf den
Technischen Erfolg haben.
%~ 2. Mit welchen Mittel wurde diese Hypothese geprüft
Zum Prüfen der Hypothese wurden wie in Hypothese 1 vorgegangen, allerdings wurden hier 
Anzahl der Commits in den letzten 12 Monaten und Anzahl der Mitwirkenden
verglichen.
%~ 3. Welche Daten belegen/widerlegen die Hypothese + Interpretation
Projekte mit permissiven Lizenzen haben im Median sowohl mehr Commits (252\%) als auch mehr Mitwirkende
(144\%). Somit bestätigt sich die \hyperref[H:2]{Hypothese H2}, permissive Projekte sind technisch 
erfolgreicher.


\todo{Sieh .tex}

\bigskip
\noindent
% Limitations 2.0
Dieser Arbeit hat sich nur mit Projekten beschäftigt, die hauptsächlich in JavaScript/Typescript
programmiert sind. Das JS/TS Umfeld ist hierbei dominiert von permissiven Lizenzen, dies hat zur
Folge gehabt, das von den 108 Projekten nur 10 Projekte restriktiv Lizenziert sind. Diese
Ungleichverteilung erschwert Aussagen bezüglich Lizenzen.

% TODO: Weiteres wäre: 
      % - Commits zählen nur auf Main und GH-Page branch. 
      % - Die genannten Punkte decken representieren natürlich nicht in Fülle Markt und technischen Erfolg sondern
      % sind nur versuche diese anzunähren

\bigskip
\noindent
%~ Bezug zur Literatur, bestätigt/widerlegt Erkenntnisse aus der Literatur (Außer bei Explorativen)
Die Ergebnisse bestätigen die Studie von Subramaniam et al. in dem Punkt, dass permissive Lizenzen 
einen positiven Effekt auf den Erfolg von Projekten haben \cite{subramaniamDeterminantsOpenSource2009}. 
Über den Effekt von restriktiven Lizenzen lassen sich hier keine starken Aussagen treffen, aufgrund 
kleiner Datenmengen mit restriktiven Lizenzen.

% Subramaniam et al. 
%       - Freie Lizenzen haben positiven Einfluss 
%       - Restriktive Lizenzen kann ich aufgrund der Beschränkung nicht bestätigen.

% Steward et al.
%       - idk...

% Midha et al.
%       - Freizügige Lizenzen nur zu beginn bedeutend, würde ich nicht zustimmen


% Limitations
% \todoo{
%       In 2015 hat GitHub in einem Blogeintrag die Aufteilung der Lizenznutzung auf ihrer Platform veröffentlicht
%       \cite{OpenSourceLicense2015}. Wie man in Tabelle \ref{tab:GitHub_Blog_Lizenznutzung2015} sehen kann,
%       haben 64\% aller Projekte permissive Lizenzen und 24\% restriktive. 15\% der Projekte haben keine
%       Standardlizenz und werden von GitHub daher als \textit{other} kategorisiert.
% }
%? What about this text? 
% Allerdings muss man einräumen,
% dass für die Beliebtheit nur GitHub Sterne verwendet wurden. Der Linux Kernel ist GPLv2 lizenziert und
% gehört zu den bekanntesten und erfolgreichsten Open Source Projekten.
% Ein Grund weshalb freizügige Lizenzen beliebter sein könnten ist, dass es sich bei restriktiven
% Projekten meist, um Applikationen handelt. Telegram, Signal und ähnliche Open Source Chat Apps
% werden von Entwicklern sowie nicht Entwicklern verwendet, die eventuell gar keine GitHub Accounts haben.
% GitHub Sterne erfassen somit nicht das gesamte Bild \textit{Markterfolg}.





% ------------------------------------------ Hypothese 3 ----------------------------------------- %
%\newpage %! new page
\section{Der Einfluss einer guten Dokumentation auf den Markterfolg} 
% TODO: Kann der Titel eigentlich sowas sein wie? 

%~ 1. Um welche Hypothese geht es
Im Kapitel \ref{sec:Documentation} wurde die Hypothese aufgestellt, dass gute Dokumentationen mehr 
Nutzer anziehen würden und somit zu einem höheren Markterfolg führen würde. 
%~ 2. Mit welchen Mittel wurde diese Hypothese geprüft
Um diese Hypothese zu prüfen wurde eine Umfrage durchgeführt. Die Umfrage hatte 308 Teilnehmer. 
%~ 3. Welche Daten belegen/widerlegen die Hypothese + Interpretation
Die Teilnehmenden sollten auf einer Likert Skala Aussagen zustimmen oder ablehnen, hier hatte das 
Thema Dokumentation die höchste Zustimmung verglichen mit den anderen Themengebieten der Umfrage, 
dies zeigt ein allgemein hohes Interesse am Thema Dokumentation. Diese Aussage wird vom Fakt 
untermauert, das wie in Kapitel \ref{sec:umfrage_last_question} gezeigt wird, die Dokumentation 
das wichtigste Kriterium, bei der Auswahl von OSS ist. Des Weiteren gaben 81\% der befragten an, 
aufgrund schlechter Dokumentation schon mal zu einem alternativen Projekt gewechselt zu haben.
% Hypothese ist Proven.
Das zeigt die Wichtigkeit einer guten Dokumentation und bestätigt somit die 
\hyperref[H:3]{Hypothese H3}.


%~ Beschränkungen
Die Freitext Felder haben gezeigt das die Kriterien die es zum bewerten gab nicht vollständig waren. 
Vor allem Aktualität und Vollständigkeit sind weitere wichtige Eigenschaften für Dokumentationen.
Für weitere Forschung wäre daher sinnvoll die Kriterien die es zu bewerten gilt zu erweitern.
Dies würde auch dazu führen das man die Wichtigkeit dieser Kriterien vergleichbarer macht.

%~ Bezug zur Literatur, bestätigt/widerlegt Erkenntnisse aus der Literatur (Außer bei Explorativen)
In einer Umfrage der Open Source Survey aus 2017 stellte sich heraus, dass unvollständige bzw. 
verirrende Dokumentation als größtes Problem in Open Source gelten \cite{GitHubOpenSourceSurvey2017}.
Diese Arbeit bestätigt diese Ergebnisse, und erweitert diese. Neben Komplexität und Unvollständigkeit
sind die wichtigste Punkte einer Guten Dokumentation, Code Beispiele und Aktualität.


% Die Ergebnisse der Umfrage bezüglich Code Beispiele decken mit den Aussagen aus Dagenais et al. 
% \cite{dagenaisDeveloperDocumentation}.




% ---------------------------------------- Hypothese 4 & 5 --------------------------------------- %
\section{Der Einfluss vom Code of Conduct und Contributing Guide auf den technischen Erfolg}

%~ 1. Um welche Hypothese geht es
Im Kapitel \ref{sec:community} wurden zwei Hypothesen aufgestellt, basierend auf den Open Source 
Guide von GitHub aufgestellt \cite{GitHubBuildingWelcomingCommunities2022}.
Die \hyperref[H:4]{Hypothese H4} beschäftigt sich mit dem Effekt des \textit{Code of Conducts} auf
den technischen Erfolg. \hyperref[H:5]{Hypothese H5} hingegen beschäftigt sich mit dem Effekt des
\textit{Contributing Guides} auf den technischen Erfolg.
%~ 2. Mit welchen Mittel wurde diese Hypothese geprüft
Zum prüfen beider Hypothesen wurden die Daten aus der Datenerhebung verwendet.
%~ 3. Welche Daten belegen/widerlegen die Hypothese + Interpretation
Das Code of Conduct haben 53\% der 108 analysierten Projekte. Einen Contributing Guide 86\%.

Gemessen wurde der technische Erfolg anhand Commits und und Anzahl der Mitwirkenden. Projekte mit 
einem Code of Conduct haben deutlich mehr Commits (717\%) und Mitwirkende (201\%).
Projekte mit einem Contributing Guide haben 463\% mehr Commits und 412\% mehr Mitwirkende als Projekte 
ohne.
% Hypothese belegt.
Damit sind beide Hypothese eindeutig belegt.

%~ Begründung für die Daten
Eine plausible Erklärung für den starken unterschied, kann damit begründet werden, dass es sich für 
kleine Projekte nicht lohnt eine Code of Conduct einzuführen, da die Größe der Teams zu klein ist.

Im Fall des Contributing Guides, könnte die ungleiche Verteilung in den Daten die Ursache sein.

%~ Bezug zur Literatur, bestätigt/widerlegt Erkenntnisse aus der Literatur (Außer bei Explorativen)
Es zeigt sich allerdings deutlich, dass die Empfehlungen von GitHub auch durchgesetzt werden, vor allem
in großen Projekten. 



% ------------------------------------------ Hypothese 6 ----------------------------------------- %
\section{Technischer Erfolg führt zu Markterfolg}

\red{Ich glaub diese Hypothese schmeiß ich wieder aus da ich die nicht ordentlich belegen/wiederlegen kann}

%//% %~ 1. Um welche Hypothese geht es
%// Im Kapitel \ref{sec:community} wurde die Hypothese aufgestellt, dass technischer Erfolg positiven 
%// Einfluss auf den Markterfolg hat.
%// %~ 2. Mit welchen Mittel wurde diese Hypothese geprüft
%// Die Hypothese wurde mittels Umfrage geprüft.
%// %~ 3. Welche Daten belegen/widerlegen die Hypothese + Interpretation
%// %~ Begründung für die Daten
%// %~ Bezug zur Literatur, bestätigt/widerlegt Erkenntnisse aus der Literatur (Außer bei Explorativen)




% ---------------------------------------- Hypothese 7 & 8 --------------------------------------- %


\section{Einfluss der Sponsoren auf den Erfolg}

%~ 1. Um welche Hypothese geht es
Im Kapitel \ref{sec:sponsors} wurde die Hypothese aufgestellt, dass Sponsoren einen positiven Einfluss
bei der Wahl von OSS hätten.
%~ 2. Mit welchen Mittel wurde diese Hypothese geprüft
Die Hypothese wurde mittels Umfrage getestet.
%~ 3. Welche Daten belegen/widerlegen die Hypothese + Interpretation
Wie in Kapitel \ref{sec:umfrage_sponsoren} und \ref{sec:sponsoren_datenerhebung} bereits gezeigt, 
spielen Sponsoren für die Nutzer von OSS fast gar keine Rolle.

Somit wird \hyperref[H:6]{Hypothese 6} nicht unterstützt.


%// %~ Bezug zur Literatur, bestätigt/widerlegt Erkenntnisse aus der Literatur (Außer bei Explorativen)
% Die \hyperref[H:7]{Hypothese H7}, Sponsoren hätten einen positiven Einfluss bei der Auswahl von OSS
% wird nicht unterstützt. In Kapitel \ref{sec:Beliebtheit} \todoo{zeigt sich das die Anzahl der Sponsoren
%       garnicht beachtet werden}


%~ 1. Um welche Hypothese geht es
Bezüglich Sponsoren wurde zusätzlich noch die Hypothese aufgestellt, dass Sponsoren sich positiv auf
den technischen Erfolg auswirken würden. 
%~ 2. Mit welchen Mittel wurde diese Hypothese geprüft
Diese Hypothese wurde mittels der Datenerhebung geprüft.
%~ 3. Welche Daten belegen/widerlegen die Hypothese + Interpretation
Wie sich herausstellt haben Projekte mit Sponsoren 72\% mehr Commits und 86\% mehr Mitwirkende.
%~ Limitations für die Daten
% Manche Projekte haben keine Sponsoren sondern Unternehmen.

%~ Bezug zur Literatur, bestätigt/widerlegt Erkenntnisse aus der Literatur (Außer bei Explorativen)
%TODO: Kein Bezug zur Literatur

%%% Disclaimer
%\subsubsection*{Anmerkung}
% Man muss allerdings anmerken, dass einige Projekte von Unternehmen entwickelt werden. Diese finanzieren
% ihre Projekte meist selbst und sind somit nicht abhängig von Sponsoren.
% React beispielsweise kommt aus dem Hause Facebook, während VueJS unabhängig ist und auf Sponsoren angewiesen ist.
% In diesem Kontext zählen Projekte von Unternehmen auch als gesponsert.
% Es sind aber auch nur 10 Projekte von Unternehmen, also fällt das auch nicht so sehr ins Gewicht tatsächlich.





% ------------------------------------------ Hypothese 9 ----------------------------------------- %
\section{Zu Hypothese 9}
Die \hyperref[H:9]{Hypothese H9}, Beliebte Projekte werden von Nutzern bei der Wahl von OSS bevorzugt.
Wird nur schwach/Nicht unterstützt? (54\% Nein etc....)

In Kapitel \ref{sec:Beliebtheit} sieht man das fast die hälfte schon auf Beliebtheit achte, aber halt
erstens nur die hälfte und zweites kommt es auf die Vergleichsmetrik an. Mehr als die hälfte zumindest
achtet auf Downloads.... (Schwach unterstützte Hypothese? Garnicht? idk)




% TODO: Nach allen Hypothesen, eine zusammenfassung, 4 von 9 sind bestätigt etc...  und eine teil bestätigt.....

% TODO: Bezug auf LIteratur, anders oder gleich...

% TODO: Ausblick, ...




%! Allgemeine Limitation
Gleiches war bei den Contributor nicht umsetzbar, da die GitHub API keinen
Endpoint hierfür hat, somit werden Contributor über den gesamten Zeitraum des Projektes verglichen. (Commits nur der letzten 12 Monate)


\begin{itemize}
      \item Umfrage hatte nur den Umfang von 308 Teilnehmern und kann deshalbt nicht generalisiert werden... 
      \item Die Daten aus dem \textit{Crawler} sind nur ein Snapshot der aktuellen Lage, sprich man kann
            nur sagen. Ja es gibt eine Korrelation zwischen z.B. Sponsoren und User Interest, aber nicht was was hervorruft.
            Hat das Projekt Sponsoren weil es viele Nutzer hat \textit{oder} hat das Projekt viele Nutzer weil es Sponsoren hat?
            Dafür bräuchte man eine Datenerhebung over time.
            Diese ist allerdings im kleinen Zeitfenster einer Bachelorarbeit nicht machbar.
            \cite{midhaFactorsAffectingSuccess2012} hat beispielsweise in einem Zeitfenster von 3 Jahren die
            Datenerhebung gemacht. % Abstract
\end{itemize}


% \subsection{Disclaimer}

% https://www.npmjs.com/package/npm-downloads-increaser

% npm downloadzahlen sind manipulierbar, ich geh zwar davon aus, dass es allgemein nicht häufig gemacht wird
% und da ja sowieso vielmehr in die rechnung rein geht als nur npm downloads. Werden diese daten trotzdem mit erfasst.
% \link{How to exploit NPM}{https://dev.to/andyrichardsonn/how-i-exploited-npm-downloads-and-why-you-shouldn-t-trust-them-4bme}.

% In einem
% \link{Blogpost}{https://blog.npmjs.org/post/92574016600/numeric-precision-matters-how-npm-download-counts-work.html}
% erklärt NPM wie es zu den Download zahlen zustande kommt.