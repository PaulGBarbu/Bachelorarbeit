\chapter{Fazit}


%? Was wurde gemacht?
Das Ziel dieser Arbeit war es den Einfluss von \textbf{Lizenzen, Dokumentation, Code of Conduct,
    Contributing Guide, Beliebtheit und Vorhandensein von Sponsoren} zu erforschen. Um dieses Ziel zu
erreichen wurden insgesamt acht Hypothesen aufgestellt. Mithilfe einer Umfrage mit 308 Teilnehmern
sowie einer Datenerhebung von 108 Github Projekten konnten sieben diese Hypothesen belegt werden.

%? Erkenntnisse (1)
Die Umfrage hat gezeigt, dass Dokumentation das wichtigste Entscheidungskriterium der Nutzer bei
der Auswahl von Open Source Software ist. Die wichtigsten Aspekte einer guten Dokumentation sind
hierbei Übersichtlichkeit, Code Beispiele sowie eine Getting Started Seite. Weitere Eigenschaften
sind Aktualität und Vollständigkeit. Die Umfrage kam auf ähnliche Ergebnisse wie \cite{GitHubOpenSourceSurvey2017},
und bestätigt die Aussagen von \cite{scottEightRulesGood2018}
%? Erkenntnisse (2)
Die Umfrage hat auch gezeigt, dass Beliebtheit das zweit wichtigste Entscheidungskriterium ist.
Anzahl der Downloads und die der GitHub Sterne werden hierbei am häufigsten verwendet. Diese Erkenntnis
deckt sich mit der von \cite{midhaFactorsAffectingSuccess2012}.
%? Erkenntnisse (3)
Die Datenerhebung der 108 GitHub Projekte hat gezeigt, dass permissive lizenzierte Projekte einen
höheren Markt- als auch technischen Erfolg zu verzeichnen haben.
Aufgrund der starken Ungleichverteilung der Lizenzen Gruppen, lässt sich allerdings keine allgemeingültigen
Schlussfolgerungen ziehen. 
%? Erkenntnisse (4)
Wie die Datenerhebung und Umfrage gezeigt haben, sind Projekte mit Sponsoren technisch erfolgreicher, als
Projekte ohne. Allerdings haben Sponsoren keinen Einfluss auf den Markterfolg. Sponsoren werden von 
potenziellen Nutzern fast nicht beachtet.

%? Ausblick & Empfehlung für weitere Forschung
Weitere Forschung ist nötig, um den Zusammenhang der von Lizenzen und Erfolg zu erforschen. Hierbei
muss darauf geachtet werden, dass restriktive und permissive Projekte besser gleichverteilt sind.
Da Dokumentation sich als wichtigstes Kriterium des Open Source Erfolgs erwiesen hat, wäre es empfehlenswert
diesen Erfolgsfaktor in weiterer Studien zu untersuchen.