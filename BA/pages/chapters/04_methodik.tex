\chapter{Methodik}

% -------------------------------------------- Umfrage ------------------------------------------- %
\section{Umfrage}


% ----------------------------------- Datenerhebung / "Crawler" ---------------------------------- %
\section{Datenerhebung / "Crawler"}
Hierfür wurden X viele Projekte Handpicked ausgewählt, von großen mit über hundert tausend GH-Stars, bis hin
zu kleinen mit wenigen hundert GH-Stars. Hierbei wurden nur welche ausgewält die....


% ----------------------------------------- Dokumentation ---------------------------------------- %
\section{Dokumentation}

Der Punkt \textit{Gute Dokumentation} (kann/wird) auf zwei verschiedene Arten und Weisen erfasst.
Einmal durch die Umfrage und einmal mittels Crawler. In der Umfrage gibt es konkrete Fragen Bezüglich
Dokumentation wie:
\begin{itemize}
    \item Wie wichtig ist eine gute Dokumentation bei der Auswahl einer OSS für Sie?
    \item Würden Sie wegen einer schlechten Dokumentation ein Stück Software NICHT nutzten?
    \item Etc.
\end{itemize}

Bezüglich Datenerfassung mittels Crawler muss man das wohl eher händisch machen. Da man mit einem
Crawler nicht bewerten kann, ob eine Dokumentation gut/vorhanden ist vor allem, weil sich diese
meist auf anderen Websites befinden. Hierbei könnte man pro Projekt folgendermaßen vorgehen:
Man geht händisch durch die Projekte durch und bewertet diese nach folgendem Schema

\begin{itemize}
    \item 0 = keine Dokumentation
    \item 1 = Basis Dokumentation in der GitHub README (Nur ein Getting Started)
    \item 2 = Ausführliche Dokumentation (Eigene Website / Code Beispiele etc.)
\end{itemize}
