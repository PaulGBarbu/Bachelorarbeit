\chapter{Methodik}

\todo{Ausformulieren}
\bigskip


% TODO: Hinschreiben genau wie viele Datensätze das sind.
Für die Datenerhebung wurden 101 Open Source Projekte auf GitHub ausgewählt.
% Rahmenbedingungen
Laut Midha et al. lassen sich unterschiedliche Programmiersprache schlecht miteinander vergleichen
\cite{midhaFactorsAffectingSuccess2012}. Um eine vergleichbare Basis zu schaffen wurde sich bei der
Datenerhebung auf die Programmiersprachen JavaScript und TypeScript beschränkt.


\todo{zusammenhang zwischen JS und TS und warum auch TS Projekte mitaufgenommen werden}

\todo{evtl. weiter Beschränkung erwähnen: Nur Bibliotheken Framekworks etc.}

\todo{evtl. erwähnen warum gerade JavaScript}

% % %  Why JavaScript  % % %
% JavaScript ist aktuell die beliebteste Programmiersprache auf GitHub, wenn man nach Anzahl der
% Repositories geht außerdem lassen sich Projekte auch über Downloadzahlen auf NPM vergleichen.



% ------------------------------------------------------------------------------------------------ %
%                                     Händische Datenerfassung                                     %
% ------------------------------------------------------------------------------------------------ %
\section{Händische Datenerfassung}

\todo{Ausformulieren}

\todo{ggf. mehr Projekte mit nicht MIT Lizenzen in Datenerhebung aufnehmen}

\bigskip
\noindent
Mithilfe einer selbstgeschrieben GUI wurden zunächst Eckdaten händisch erfasst, dazu gehören Besitzer und Name
der ausgewählten GitHub Projektes sowie der Name des Projektes auf NPM. Des Weiteren wurden Daten erfasst, die
man nicht automatisiert hätte erfassen können. Darunter gehören Daten wie: Qualität der Dokumentation, ob ein
Projekt Sponsoren hat, wer hinter den Projekten steckt und um was für ein Typ an Projekt es sich handelt.

Mit der GitHub Suchfunktion\footnote{\url{https://github.com/search/advanced}} wurde nach JS/TS Projekten
gesucht. Die Projekte werden absteigend nach Sternen wiedergegeben und auch so auch erfasst.
Allerdings gibt es hier viele Projekte die keine Bibliotheken, Frameworks oder ähnliches sind,
wie beispielsweise
\textit{30-seconds/30-seconds-of-code}\footnote{\url{https://github.com/30-seconds/30-seconds-of-code}}.
Dieses und ähnliche Projekte wurden entsprechen für die Datenerfassung nicht berücksichtigt.

Während der Datenerfassung wurde festgestellt, dass die meisten Projekte eine MIT Lizenz haben,
sodass später explizit nach nicht MIT lizenzierten Projekten gesucht wurde.

Alle Einträge wurden in einer CSV gespeichert und später weiter automatisiert verarbeitet, mehr dazu in
Kapitel \ref{sec:automatisierte_datenerfassung}.


\bigskip
\noindent
In den nächsten Unterkapiteln soll näher erklärt wie und nach welchen Kriterien die von Hand erfassten
Daten zu Stande kommen und welche Bedeutung die Werte in der CSV haben.

% ----------------------------------------- Dokumentation ---------------------------------------- %
\subsection{Dokumentation}
Bei der Qualität der Dokumentation wurde nach folgenden Kriterien bewertet

\begin{itemize}[noitemsep]
    \item[0 =] keine Dokumentation
    \item[1 =] basis Dokumentation, rein textuelle Dokumentation
    \item[2 =] Dokumentationen mit Demos oder Live Beispielen,
        Einführungsvideos oder ähnliches
\end{itemize}

\noindent
Hierfür wurde die Dokumentation auf GitHub selbst bzw. auf den externen Websiten überflogen und
kategorisiert.


% ------------------------------------------- Sponsoren ------------------------------------------ %
\subsection{Sponsoren}
Hier wurde für jedes Projekt geprüft, ob es Sponsoren hat. Hierfür gibt es oft auf der GitHub Page
eine Link zur Sponsor Seite oder in den meisten Fällen findet sich in der \textit{README.md} bzw.
auf der Website des Projektes eine kurze Danksagung an die Top Sponsoren.

Die Anzahl an Sponsoren bzw. die Einnahmen durch Sponsoren wurden nicht beachtet, nur ob Sponsoren
vorhanden sind.

\begin{itemize}[noitemsep]
    \item[0 =] hat keine Sponsoren
    \item[1 =] hat Sponsoren
\end{itemize}


% ------------------------------------------- Backed by ------------------------------------------ %
\subsection{Backed By}
Im nächsten Schritt wurde vermerkt wer hinter einem Projekt steckt, hierbei gab es die Möglichkeiten:

\begin{itemize}[noitemsep]
    \item[0 =] Eines reinen Community Projektes wie zum Beispiel VueJS,
    \item[1 =] ein Projekt wie NodeJS welches von einer Stiftung wie der OpenJS\footnote{\url{https://openjsf.org/}}
        entwickelt oder unterstützt wird oder
    \item[2 =] Projekte die von Unternehmen entwickelt werden, wie React von Facebook beispielsweise.
\end{itemize}

Diese Information fand sich entweder auf der GitHub bzw. Homepage des Projektes oder das Unternehmen
ist der Besitzer des Repositories.

% ------------------------------------------- Kategorie ------------------------------------------ %
\subsection{Kategorie}
Im letzten Schritt wurde das Projekt nach Typ Kategorisiert, folgende Kategorien standen zur Auswahl.

\begin{table}[h]
    \begin{tabular}{ll}
        \hline
        \textbf{Kategorie} & \textbf{Anzahl}        \\ \hline
        Utility            & \multicolumn{1}{c}{40} \\
        UI                 & \multicolumn{1}{c}{28} \\
        Application        & \multicolumn{1}{c}{12} \\
        Library            & \multicolumn{1}{c}{7}  \\
        Framework          & \multicolumn{1}{c}{6}  \\
        Test-Framework     & \multicolumn{1}{c}{5}  \\
        Open-Core          & \multicolumn{1}{c}{2}  \\
        API                & \multicolumn{1}{c}{1}
    \end{tabular}%
\end{table}


Utility




\begin{itemize}[noitemsep]
    \item \textbf{Utility}, wird nicht direkt in Projekte eingebaut sondern als Tool verwendet.
          Beispiel: \texttt{shelljs/shelljs}\footnote{\url{https://github.com/shelljs/shelljs}}
    \item UI
    \item Application
    \item Library
    \item Framework
    \item Test-Framework
    \item Open-Core
    \item API
\end{itemize}



\todo{Unterscheidung zwischen Utility/Library etc.}


\todo{Zahlen Ausfüllen, wv Projekte in welcher Kategorie}



% ------------------------------------------------------------------------------------------------ %
%                                                                                                  %
%                                   Automatisierte Datenerfassung                                  %
%                                                                                                  %
% ------------------------------------------------------------------------------------------------ %
\section{Automatisierte Datenerfassung}\label{sec:automatisierte_datenerfassung}

Im zweiten Teil der Datenerfassung wurden weitere Daten automatisiert erfasst. Hierfür wurde die
GitHub API\footnote{\url{https://docs.github.com/en/rest}}, NPM
API\footnote{\url{https://github.com/npm/registry}} und für einige Daten Web-Scraping
verwendet.


% ----------------------------------- Daten aus der GitHub API ----------------------------------- %
\subsection{Daten aus der GitHub API}
Mittels der GitHub API wurden folgende Daten gesammelt:

\begin{itemize}[noitemsep]
    \item Anzahl der GitHub-Sternen
    \item Erstellungsdatum des Repositories
    \item Vorhandensein einer \textit{CODE\_OF\_CONDUCT.md}
    \item Vorhandensein einer \textit{CONTRIBUTING.md}
    \item Lizenz
    \item Anzahl der Commits in den letzten 12 Monaten
    \item Anzahl der Issues in den letzten 12 Monaten
\end{itemize}

% * GitHub_generalInfo
% * projectHealth
% * commitActivity
% * issues_PRs

% ------------------------------------- Daten aus der NPM API ------------------------------------ %
\subsection{Daten aus der NPM API}
Mit der NPM API wurden die Downloads der letzten 7 Tage abgefragt.

% * npm_downloads


\subsection{Daten aus dem Web-Scraping}
Mithilfe von Web-Scraping wurden folgende Daten erfasst:

\begin{itemize}[noitemsep]
    \item Die Anzahl von \textit{UsedBy} auf der GitHub Page des Projektes.
    \item Anzahl der Gesamt-Commits auf dem default Branch
    \item Anzahl der Contributor
\end{itemize}

\noindent
Ein Contributor zählt nur dann als solche, wenn dessen Commit entweder auf dem \textit{default}
oder \textit{gh-pages} Branch liegt. Commits auf anderen Branches werden nur dann gezählt, wenn ein
merge auf einen der vorherig erwähnten Branches stattfindet. Gleiches gilt für Commits \cite{GHapiDocsCommits}.

% * GitHub_usedBy
% * GitHub_Page
% * npm_dependants