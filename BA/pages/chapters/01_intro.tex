\chapter{Einleitung}

\todo{Überleitung zur Fragestellung überarbeiten.}

% TODO: Kind of like: In der BA wird Projekt allgemein verwendet für, Framework, Bibliothek, Software etc.
% TODO: Disclaimer bezüglich so Sachen wie Entwickler:in ... cause you know. We live in 2022


% ? Einschränken nur auf Packages. Grund: 
%?    a) npm 
%?    b) "used by" Kategorie auf GitHub  
%?    c) kleinerer Umfang für BA


Open Source ist heutzutage ein fester Bestandteil der Softwareindustrie.
Von Frontend-Entwicklung über Datenbanken bis hin zu Machine Learning, überall kommen Open Source
Bibliotheken, Frameworks und Programme zum Einsatz.
Im Frontend werden verschiedene Frameworks bzw. Bibliotheken wie Angular oder React verwendet,
auch im Fall von Datenbankmanagementsysteme gibt eine Vielzahl an Optionen
bei der Auswahl von Datenbankmanagementsysteme steht einem eine Vielzahl von Optionen zur Auswahl
wie PostgreSQL, MySQL oder MongoDB
und im Bereich Machine Learning werde Frameworks wie TensorFlow, Keras oder SciKit-Learn genutzt.

% TODO: Überleitung braucht noch feinschliff
% Überleitung 
Doch worin unterscheiden sich die oben genannten von vielen weiteren erfolgreiche Open Source
Projekte, von denen die nicht als erfolgreich definiert werden können.
% Fragestellung
Mit dieser Bachelorarbeit soll die Frage beantwortet werden, \textit{welche Faktoren Einfluss auf
    den Erfolg von Open Source Projekten, speziell in der JavaScript/TypeScript Umgebung haben.}


% Ziel der Arbeit
Mittels einer Datenerhebung von ausgewählten Open Source Projekten sowie einer
Umfrage soll herausgefunden werden, welche Faktoren zum Erfolg eines Open Source Projektes beitragen.
% Abgrenzung
Ein zentraler Punkt dieser Ausarbeitung sind die extrinsischen sowie intrinsischen Anreize, % Extrinsisch == Downloads, Intrinsisch == Qualität
die Nutzer zur Auswahl eines Produktes motivieren \cite{midhaFactorsAffectingSuccess2012}. % Chapter 3.1 & 3.2
Warum React und nicht Angular? Warum Debian und nicht Ubuntu?
Aspekte wie die interne Führung und Organisation der Projekte wird hierbei nicht thematisiert.




% ------------------------------------------------------------------------------------------------ %
%                                    Definition von Open Source                                    %
% ------------------------------------------------------------------------------------------------ %
\section{Definition von Open Source}

\todo{Hier kann ne ganze Seite hin....}
Diese Arbeit folgt der gleichen Definition für Open Source wie von der
\textit{Open Source Initiative}\footnote{https://opensource.org/osd} definiert wird. Entsprechend
werden nur Projekte betrachtet die eine von der
OSI genehmigten Lizenzen\footnote{https://opensource.org/licenses/category}
nutzt.


% ------------------------------------------------------------------------------------------------ %
%                                         Erfolg definieren                                        %
% ------------------------------------------------------------------------------------------------ %
\section{Erfolg definieren}


% Types of Success
Der Erfolg wird in der Literatur häufig in verschiedene Bereiche unterteilt.
Im Artikel von Midha und Palvia wird zwischen \textit{Markterfolg} und \textit{technischem Erfolg}
unterschieden.
% Define Markterfolg
Midha et al. definieren Markterfolg als
als Grad des Nutzerinteresses an einem Projekt, %? Direkt Zitat: Market Success of an OSS project, a measure of project popularity, is defined as the level of interest displayed in the project by its consumers
welches sich in der Beliebtheit des Projektes widerspiegelt.
% Define Technischer Erfolg
Den technischen Erfolg definieren Midha et al. durch die Entwickleraktivität, d.h. durch den Aufwand,
den die Entwickler für das Projekt betreiben %? Direkt Zitat: Technical Success is defined in terms of developer activity, i.e.,the level of effort expended by developers of the project
beispielsweise die Häufigkeit und Frequenz von Updates und neuen Versionen
\cite{midhaFactorsAffectingSuccess2012}. % 2.1


% Nutzerinteresse & Entwickleraktivität
Im Artikel von Steward et al. wird zwischen \textit{Nutzerinteresse} und \textit{Entwickleraktivität} unterschieden
\cite{stewartImpactsLicenseChoice2006a}. % Defining Success in OSS
Subramaniam et al. gehen sogar weiter und unterteilen den Erfolg in
\textit{Nutzerinteresse, Entwicklerinteresse und Projektaktivität}
\cite{subramaniamDeterminantsOpenSource2009}. % 3.2

% Synthese
Diese Bereiche werden getrennt betrachtet, da verschiedene Faktoren unterschiedlichen Einfluss
auf den Erfolg eines Projektes haben.
Laut Midha et al. und Steward et al. wirkt sich wachsendes Interesse der Nutzern positiv auf den
Markterfolg aus, während sich die Entwickleraktivität positiv den technischen Erfolg auswirken
\cite{midhaFactorsAffectingSuccess2012, % 6.3. Hypothosis H3c (aber nur teils unterstützt)
    stewartImpactsLicenseChoice2006a}. % Defining Success in OSS (letzter Absatz)


\todo{Warum genau diese Metriken? Sterne, Downloads etc.???}

\todoo{- Kein Copy Paste von Autoren, weil die nicht GH verwendet haben (Nicht aktuell)
- Beliebte vergleichsmetriken (I guess lol)}

% Ziel dieser Arbeit
In dieser Arbeit werden die Erfolgsfaktoren betrachtet, die zum Markt- bzw. technischen Erfolg
eines Projektes beitragen.
Der Markterfolg wird anhand folgender Metriken gemessen:
Downloads und Sterne auf GitHub.
Der technische Erfolg wird anhand von Metriken wie:
Anzahl der Commits und Anzahl der Mitwirkenden am Projekt gemessen.

% \subsection{Markterfolg}

% Markterfolg wird durch Charakteristika wie Beliebtheit gekennzeichnet.
% Diese Eigenschaft spiegelt sich beispielsweise in der Anzahl der Nutzer, GitHub Sterne oder
% Downloads wider. Einige dieser Metriken finden sich auf den GitHub Seiten
% der jeweiligen Projekte wieder und werden vom Webcrawler erfasst.
% \cite{midhaFactorsAffectingSuccess2012}.

% \subsection{Technischer Erfolg}

% Unter dem technischen Erfolg zählen Eigenschaften wie Frequenz von Updates,
% sowie Anzahl an Commits und Mitwirkenden. Diese Daten lassen sich ebenfalls auf GitHub erfassen.
% \cite{midhaFactorsAffectingSuccess2012}. % Kapitel 2.1 Success

% ? \subsection{Ökosystem}
% Ein weiterer Indikator für Erfolg ist, wenn ein Projekt ein Ökosystem, um sich entwickelt.
% Beispielsweise hat das Web-Framework React eine große Community die eine Vielzahl von Erweiterungen
% unabhängig aber für React entwickeln.
% % React-Router compared to React / vs VueJS or Angular
