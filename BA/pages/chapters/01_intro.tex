\chapter{Einleitung}

Open Source ist heutzutage ein fester Bestandteil der Softwareindustrie.
Von Frontend-Entwicklung über Datenbanken bis hin zu Machine Learning, überall kommen Open Source
Bibliotheken, Frameworks und Programme zum Einsatz.

Im Frontend werden verschiedene Frameworks bzw. Bibliotheken wie Angular oder React verwendet,
im Fall von Datenbankmanagementsysteme gibt es ebenfalls eine Vielzahl an Optionen
wie PostgreSQL, MySQL oder MongoDB.
Im Bereich Machine Learning werde Frameworks wie TensorFlow, Keras oder SciKit-Learn genutzt.

% Überleitung
Doch was macht diese Projekte erfolgreich?
% Überleitung 
%Doch worin unterscheiden sich die oben genannten von vielen weiteren erfolgreichen, Open Source
%Projekten, von denen die nicht als erfolgreich definiert werden können.
% Fragestellung
Mit dieser Bachelorarbeit soll die Frage beantwortet werden, 
\textit{welche Faktoren haben Einfluss auf den Erfolg von Open Source Projekten}, insbesondere in 
der JavaScript / TypeScript Umgebung haben.
In dieser Arbeit soll der Einfluss von \textbf{Lizenzen, Dokumentation, Code of Conduct, Contributing
Guide, Beliebtheit und Vorhandensein von Sponsoren}, auf den Erfolg von Open Source Projekten 
erforscht werden. 


% Ziel der Arbeit
Mittels einer Datenerhebung von ausgewählten Open Source Projekten sowie einer
Umfrage soll dies herausgefunden werde.
% Abgrenzung
Ein zentraler Punkt dieser Ausarbeitung sind die extrinsischen sowie intrinsischen Anreize, % Extrinsisch == Downloads, Intrinsisch == Qualität
die Nutzer zur Auswahl eines Produktes motivieren \cite{midhaFactorsAffectingSuccess2012}. % Chapter 3.1 & 3.2
Warum React und nicht Angular? Warum Debian und nicht Ubuntu?
Aspekte wie die interne Führung und Organisation der Projekte wird hierbei nicht thematisiert.




% ------------------------------------------------------------------------------------------------ %
%                                    Definition von Open Source                                    %
% ------------------------------------------------------------------------------------------------ %
\section{Definition von Open Source} \label{sec:definition_OpenSource}

Diese Arbeit folgt der Definition für Open Source wie sie von der \textit{Open Source Initiative}
(OSI) vorgegeben wird \cite{OpenSourceDefinition}. Entsprechend werden nur Projekte betrachtet die 
von der OSI genehmigten sind. Des Weiteren wird im Laufe dieser Arbeit zwischen restriktiven und 
permissiven Lizenzen unterschieden.

% Permissive
Eine permissive Lizenz ist eine sehr freizügige Lizenz, diese erlaubt die Nutzung, Modifikation
und Weiterverbreitung ohne weitere Einschränkungen. Permissive Lizenzen können in proprietärer
Software verwendet werden. Beispiele für permissive Lizenzen sind MIT, BSD und Apache 
\cite{OpenSourceDefinition}.

% Restriktiv
Restriktive Lizenzen erlauben ebenfalls die Nutzung, Modifikation und Weiterverbreitung der Software.
Allerdings mit der Einschränkungen, dass Modifikationen und Weiterverbreitungen ebenfalls restriktiv 
lizenziert werden müssen. Wird beispielsweise eine GPL lizenzierte Bibliothek in einem Software Projekt
verwendet, muss die gesamte Software ebenfalls GPL lizenziert werden. Dies gilt allerdings nur dann
wenn die Software auch verbreitet wird, die Regel greift also nicht für private und interne Zwecke. 
Restriktive Bibliotheken und Frameworks können somit nicht in proprietärer Software eingesetzt werden.
Beispiele für restriktive Lizenzen sind GPL, AGPL und LGPL \cite{OpenSourceDefinition}


% \xodoo{
%     Was macht permissive aus?
%     - Kann uneingeschränkt benutzt werden
%     - Kann ez verwendet werden, einfach Lizenz Text beilegen
%     - Source Code einer MIT Lib kann in Propreritärer Software verwendet werden usw.
%     - Beispiel: MIT, BSD, Apache
%     %
%     Was macht GPL aus?
%     - Kann nicht uneingeschränkt genutzt werden => one GPL Lib => Whole Programm needs to be Open Source now under gpl
%     - GPLv2 vs GPLv3
%     - Was hat es dann eigentlich mit AGPL und LGPL aufsich? irgendwas mit Cloud und Server shit
%     %
%     MPL
%     - Die Mitte zwischen GPL und MIT
%     So under the terms of the MPL, it allows the integration of MPL-licensed code into proprietary codebases,
%     but only on condition those components remain accessible  
% }

% Quelle: https://en.wikipedia.org/wiki/Mozilla_Public_License
% Sieh auch: https://www.mozilla.org/en-US/MPL/2.0/FAQ/   =>   Q11
% https://www.youtube.com/watch?v=YlKtCDJquSw


% ------------------------------------------------------------------------------------------------ %
%                                         Erfolg definieren                                        %
% ------------------------------------------------------------------------------------------------ %
\newpage
\section{Erfolg definieren}


% Types of Success
Der Erfolg wird in der Literatur häufig in verschiedene Bereiche unterteilt.
Im Artikel von Midha und Palvia wird zwischen \textit{Markterfolg} und \textit{technischem Erfolg}
unterschieden.
% Define Markterfolg
Midha et al. definieren Markterfolg
als Grad des Nutzerinteresses an einem Projekt, %? Direkt Zitat: Market Success of an OSS project, a measure of project popularity, is defined as the level of interest displayed in the project by its consumers
welches sich in der Beliebtheit des Projektes widerspiegelt.
% Define Technischer Erfolg
Den technischen Erfolg definieren Midha et al. durch die Entwickleraktivität, d.h. durch den Aufwand,
den die Entwickler für das Projekt betreiben %? Direkt Zitat: Technical Success is defined in terms of developer activity, i.e.,the level of effort expended by developers of the project
beispielsweise die Häufigkeit und Frequenz von Updates und neuen Versionen
\cite{midhaFactorsAffectingSuccess2012}. % 2.1


% Nutzerinteresse & Entwickleraktivität
Im Artikel von Steward et al. wird zwischen \textit{Nutzerinteresse} und \textit{Entwickleraktivität} unterschieden
\cite{stewartImpactsLicenseChoice2006a}. % Defining Success in OSS
Subramaniam et al. gehen sogar weiter und unterteilen den Erfolg in
\textit{Nutzerinteresse, Entwicklerinteresse und Projektaktivität}
\cite{subramaniamDeterminantsOpenSource2009}. % 3.2

% Synthese
Diese Bereiche werden getrennt betrachtet, da verschiedene Faktoren unterschiedlichen Einfluss
auf den Erfolg eines Projektes haben.
Laut Midha et al. und Steward et al. wirkt sich wachsendes Interesse der Nutzern positiv auf den
Markterfolg aus, während sich die Entwickleraktivität positiv den technischen Erfolg auswirken
\cite{midhaFactorsAffectingSuccess2012, % 6.3. Hypothosis H3c (aber nur teils unterstützt)
    stewartImpactsLicenseChoice2006a}. % Defining Success in OSS (letzter Absatz)


% Ziel dieser Arbeit
In dieser Arbeit werden die Erfolgsfaktoren betrachtet, die zum Markt- bzw. technischen Erfolg
eines Projektes beitragen.
In Midha et al. wurde der Markterfolg mittels der Downloadzahlen gemessen \cite{midhaFactorsAffectingSuccess2012}. % Kapitel 4.2
Diese Arbeit erweitert die Erfolgsmetrik zusätzlich, um GitHub Sterne, da sich diese Metrik ebenfalls gut
eignet, um die Beliebtheit eines Projektes zu messen. Die Downloadzahlen werden hierbei von NPM bezogen.
Für den technischen Erfolg verwenden Midha et al. die Anzahl der Commits \cite{midhaFactorsAffectingSuccess2012}. 
Zusätzlich werden in dieser Arbeit noch die Anzahl der Mitwirkenden als technischen Erfolg gewertet.


% \subsection{Markterfolg}

% Markterfolg wird durch Charakteristika wie Beliebtheit gekennzeichnet.
% Diese Eigenschaft spiegelt sich beispielsweise in der Anzahl der Nutzer, GitHub Sterne oder
% Downloads wider. Einige dieser Metriken finden sich auf den GitHub Seiten
% der jeweiligen Projekte wieder und werden vom Webcrawler erfasst.
% \cite{midhaFactorsAffectingSuccess2012}.

% \subsection{Technischer Erfolg}

% Unter dem technischen Erfolg zählen Eigenschaften wie Frequenz von Updates,
% sowie Anzahl an Commits und Mitwirkenden. Diese Daten lassen sich ebenfalls auf GitHub erfassen.
% \cite{midhaFactorsAffectingSuccess2012}. % Kapitel 2.1 Success

% ? \subsection{Ökosystem}
% Ein weiterer Indikator für Erfolg ist, wenn ein Projekt ein Ökosystem, um sich entwickelt.
% Beispielsweise hat das Web-Framework React eine große Community die eine Vielzahl von Erweiterungen
% unabhängig aber für React entwickeln.
% % React-Router compared to React / vs VueJS or Angular
