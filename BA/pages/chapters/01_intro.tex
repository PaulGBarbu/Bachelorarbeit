\chapter{Einleitung}

% \todo{Wird im Laufe der Arbeit noch genauer spezifizieren.}

% ? Einschränken nur auf Packages. Grund: 
%?    a) npm 
%?    b) "used by" Kategorie auf GitHub  
%?    c) kleinerer Umfang für BA

% Intro
In der IT ist Open Source mittlerweile ein fester Bestandsteil der gesamten Infrastruktur.
Mehr als die Hälfte aller Web Server laufen unter Open-Source-Lizenzen \cite{W3Techs_WebServer}.
Die meistgenutzten Frontend Frameworks sind ebenfalls alle Open Source. \cite{StackOverflowSurvey2021}


% Open Source Erklärung
Der Begriff Open Source ist den meisten Softwareentwicklern wahrscheinlich bekannt,
aber was genau steckt dahinter? Die Antwort ist weitaus mehr als \textit{nur} quelloffener Code
und kostenlose Software.
Die \textit{Open Source Initiative} hat eine klare Definition für Open Source.
Wie der Name schon sagt, muss der Quellcode offen liegen, des Weiteren gelten allerdings auch
Voraussetzungen, wie beispielsweise, dass Nutzer den Quellcode verändern und weitergeben
dürfen \cite{OpenSourceDefinition}.


% Ziel der Arbeit
Mit dieser Arbeit soll, basierend auf ausgewählten Open-Source-Projekten und einer Umfrage,
herausgefunden werden welche Faktoren entscheidend zum Erfolg eines Projektes beitragen.
Hierbei wird hauptsächlich von der Nutzerperspektive ausgegangen, wobei mit Nutzer nicht nur die
Endnutzer der Software, sondern auch Softwareentwickler gemeint sind, die Open Source Produkte wie
Bibliotheken etc. in eigenen Projekten einbauen.

% Abgrenzung
Ein zentraler Punkt dieser Ausarbeitung sind die extrinsischen sowie intrinsischen Anreize,
die Nutzer zur Auswahl eines Produktes motivieren \cite{midhaFactorsAffectingSuccess2012}. % Chapter 3.1 & 3.2
Aspekte wie die interne Führung und Organisation der Projekte wird hierbei nicht thematisiert.

% --------------------------------------- Erfolg definieren -------------------------------------- %
\section{Erfolg definieren}

% TODO: Weitere Ideen: Sponsoren
% Sponsoren als Erfolgsmetrik (?)
% -------------------------------
% Sponsoren: Erfolgreiche Projekte haben Sponsoren, kann man als Indikator für Erfolg hernehmen
% aber auch als Faktor was es teils braucht um Erfolgreich zu werden. (Mehr Geld => Mehr Zeit fürs Projekt)
% Beziehungsweise wenn die Software aus einem Unternehmen kommt React => Facebook / Google => Tensorflow etc.
% Das ist zwar kein Garant für Erfolg, allerdings könnte man sich dann anschauen
% Hypothesen:
%     - "Are Company backed OSS more successful?"
%     - "Are Sponsor backed OSS more sucessful?"} mehr in \cite{stewartImpactsLicenseChoice2006a} Hypothese 2A

% -------------------------------


% Types of Success
In der Literatur wird häufig in verschiedene Bereichen des Erfolgs unterteilt.
Im Artikel von Midha und Palvia wird zwischen \textit{Markterfolg} und \textit{technischem Erfolg} unterschieden.
% Define Markterfolg
Markterfolg definiert Midha et al.
als Grad des Nutzerinteresses an ein Projekt, %? Direkt Zitat: Market Success of an OSS project, a measure of project popularity, is defined as the level of interest displayed in the project by its consumers
welches sich in der Beliebtheit des Projektes widerspiegelt.
% Define Technischer Erfolg
Den technischen Erfolg definiert Midha et al.
durch die Entwickleraktivität, d.h. durch den Aufwand, den die Entwickler für das Projekt betreiben %? Direkt Zitat: Technical Success is defined in terms of developer activity, i.e.,the level of effort expended by developers of the project
Beispielsweise die Häufigkeit und Frequenz von Updates und neuen Versionen.
\cite{midhaFactorsAffectingSuccess2012}. % 2.1


% Nutzerinteresse & Entwickleraktivität
In Steward et al. wird zwischen \textit{Nutzerinteresse} und \textit{Entwickleraktivität} unterschieden
\cite{stewartImpactsLicenseChoice2006a}. % Defining Success in OSS
Subramaniam et al. geht sogar weiter und unterteilt in \textit{Nutzerinteresse, Entwicklerinteresse und Projektaktivität}
\cite{subramaniamDeterminantsOpenSource2009}. % 3.2

% Synthese
Diese Bereiche werden getrennt betrachtet, da verschiedene Faktoren unterschiedlichen Einfluss
auf den Erfolg eines Projektes haben.
Während sich wachsendes Interesse bei Nutzern positiv auf den Markterfolg auswirkt,
wirkt sich die Entwickleraktivität positiv den technischen Erfolg aus
\cite{midhaFactorsAffectingSuccess2012, % 6.3. Hypothosis H3c (aber nur teils unterstützt)
    stewartImpactsLicenseChoice2006a}. % Defining Success in OSS (letzter Absatz)

% Ziel dieser Arbeit
In dieser Arbeit werden die Erfolgsfaktoren betrachtet, die zum Markt- bzw. technischen Erfolg
eines Projektes beitragen.
Der Markterfolg wird anhand Metriken gemessen wie:
Downloads, Sterne auf GitHub sowie die Anzahl der Nutzer.
Der technische Erfolg wird anhand von Metriken wie:
Anzahl der Commits, Anzahl der Mitwirkenden am Projekt,
Geschwindigkeit in der Tickets abgearbeitet werden und Verhältnis zwischen offenen/abgeschlossenen Issues gemessen.

% \subsection{Markterfolg}

% Markterfolg wird durch Charakteristika wie Beliebtheit gekennzeichnet.
% Diese Eigenschaft spiegelt sich beispielsweise in der Anzahl der Nutzer, GitHub Sterne oder
% Downloads wider. Einige dieser Metriken finden sich auf den GitHub Seiten
% der jeweiligen Projekte wieder und werden vom Webcrawler erfasst.
% \cite{midhaFactorsAffectingSuccess2012}.

% \subsection{Technischer Erfolg}

% Unter dem technischen Erfolg zählen Eigenschaften wie Frequenz von Updates,
% sowie Anzahl an Commits und Mitwirkenden. Diese Daten lassen sich ebenfalls auf GitHub erfassen.
% \cite{midhaFactorsAffectingSuccess2012}. % Kapitel 2.1 Success

% ? \subsection{Ökosystem}
% Ein weiterer Indikator für Erfolg ist, wenn ein Projekt ein Ökosystem, um sich entwickelt.
% Beispielsweise hat das Web-Framework React eine große Community die eine Vielzahl von Erweiterungen
% unabhängig aber für React entwickeln.
% % React-Router compared to React / vs VueJS or Angular
